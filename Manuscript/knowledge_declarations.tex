% \knowledgestyle*{notion}{style={color=black!80!},intro style={color=black!80!,emphasize}}

\knowledgestyle{notion}{color=black}
\knowledgestyle{notion intro}{color=LightOrange,emphasize}
\knowledgedirective*{notion}{autoref,style=notion,intro style=notion intro}

\knowledgestyle{ignore intro}{color=LightOrange,emphasize}
\knowledgedirective*{ignore}{intro style=ignore intro}

\knowledgedirective*{title}{}


%Chapter 1

\knowledge{ignore}
  | type
  | types

\knowledge{notion}
  | type dépendant
  | types dépendants
  | théorie des types dépendants
  | dépendant@typ
  | dépendants@typ

\knowledge{notion}
  | assistant à la preuve
  | assistants à la preuve

\knowledge{notion}
  | Coq

\knowledge{notion}
  | Agda

\knowledge{notion}
  | correspondance de Curry-Howard

\knowledge{notion}
  | Gallina

\knowledge{notion}
  | noyau

\knowledge{notion}
  | critère de De Bruijn

\knowledge{notion}
  | MetaCoq

\knowledge{notion}
  | bidirectionnel
  | typage bidirectionnel

\knowledge{notion}
  | graduel
  | graduels
  | typage graduel
  | type graduel
  | types graduels

% Chapter 2

\knowledge{notion}
  | dependent type
  | dependent@typ

\knowledge{notion}
  | Curry-Howard correspondence

\knowledge{notion}
  | kernel

\knowledge{notion}
  | De Bruijn’s criterion

\knowledge{notion}
  | proof assistant

% Chapter 3

% Section 3.2
\knowledge{notion}
  | Calculus of Constructions
  | CCω

\knowledge{title}
  | CCω@tit
  | Calculus of Constructions@tit

\knowledge{notion}
  | CIC
  | Calculus of Inductive Constructions

\knowledge{title}
  | CIC@tit
  | Calculus of Inductive Constructions@tit

\knowledge{notion}
  | Polymorphic, Cumulative Calculus of Inductive Constructions
  | PCUIC

\knowledge{title}
  | PCUIC@tit

\knowledge{notion}
  | α-equality
  | α-equal 

\knowledge{notion}
  | Church-style

\knowledge{notion}
  | Curry-style

\knowledge{notion}
  | universe level
  | universe levels
  | level
  | levels

\knowledge{notion}
  | typical ambiguity


% Section 3.3 
\knowledge{notion}
  | conversion

\knowledge{notion}
  | typed conversion
  | Typed conversion
  | typed@conv

\knowledge{notion}
  | untyped conversion
  | Untyped conversion
  | untyped@conv

\knowledge{notion}
  | Pure Type Systems
  | PTS

\knowledge{notion}
  | declarative conversion
  | Declarative conversion
  | declarative@conv
  | conversion@decl

\knowledge{notion}
  | algorithmic conversion
  | algorithmic@conv

\knowledge{notion}
  | reduction

\knowledge{notion}
  | top-level reduction
  | Top-level reduction
  | top-level reductions
  | top-level@red

\knowledge{notion}
  | one-step reduction
  | one-step@red

\knowledge{notion}
  | weak-head reduction
  | weak-head@red

\knowledge{notion}
  | full reduction
  | full@red

% Section 3.4
\knowledge{notion}
  | stability under renaming
  | Stability under renaming
  | stable under renaming

\knowledge{notion}
  | weakening
  | Weakening

\knowledge{notion}
  | stability under substitution
  | Stability under substitution
  | stable under substitution

\knowledge{notion}
  | conditional stability under renaming
  | Conditional stability under renaming


\knowledge{notion}
  | strengthening
  | Strengthening

\knowledge{notion}
  | uniqueness of types up to
  | uniqueness of types
  | Uniqueness of types
  | uniqueness@typ

\knowledge{notion}
  | validity
  | Validity

\knowledge{notion}
  | Subject reduction
  | subject reduction
  | preservation

\knowledge{notion}
  | injectivity of function types
  | Injectivity of function types
  | injectivity of type constructors

\knowledge{notion}
  | confluence
  | Confluence

\knowledge{notion}
  | normal form
  | normal forms

\knowledge{notion}
  | neutral form
  | neutral forms
  | neutral
  | neutrals

\knowledge{notion}
  | canonical form
  | canonical forms

\knowledge{notion}
  | progress
  | Progress

\knowledge{notion}
  | safety
  | Safety

\knowledge{notion}
  | accessible
  | Accessibility

\knowledge{notion}
  | Normalization
  | normalization
  | normalizing

\knowledge{notion}
  | logical consistency
  | logically consistent
  | Logical consistency

\knowledge{notion}
  | canonicity
  | Canonicity

% Section 3.5 
\knowledge{notion}
  | inductive type
  | inductive types

\knowledge{notion}
  | boolean
  | booleans

\knowledge{notion}
  | constructor
  | constructors

\knowledge{notion}
  | recursor
  | recursors
  | induction principle

\knowledge{notion}
  | scrutinee
  | predicate
  | branches

\knowledge{notion}
  | fully applied

\knowledge{notion}
  | indexed@ind
  | indexed inductive type
  | indexed inductive types

\knowledge{notion, text ={CIC\textsuperscript{-}}}
  | CIC-


% Section 3.6
\knowledge{notion}
  | cumulativity

\knowledge{notion}
  | α-pre-order

\knowledge{notion}
  | impredicativity

\knowledge{notion}
  | proof irrelevance

\knowledge{notion}
  | local definitions
  | local definition

\knowledge{notion}
  | environment

\knowledge{notion}
  | guard condition

\knowledge{notion}
  | record type
  | record types
  | record@type

\knowledge{notion}
  | projection
  | projections

\knowledge{notion}
  | co-inductive type
  | co-inductive types

% Bidirectional part

\knowledge{notion}
  | mode
  | modes
  | subject
  | inputs
  | input
  | outputs
  | output

\knowledge{notion}
  | undirected typing
  | undirected@typ

% Chapter 4

\knowledge{notion}
  | well-formed
  | well-typed

\knowledge{notion}
  | inference

\knowledge{notion}
  | checking

\knowledge{notion}
  | Matita

\knowledge{notion}
  | boundary

\knowledge{notion}
  | constrained inference

\knowledge{notion}
  | full

\knowledge{notion}
  | type-level enforcing

\knowledge{notion}
  | Correctness
  | correctness@bidir
  | correct@bidir
  | Correctness@bidir

\knowledge{notion}
  | Completeness
  | completeness@bidir
  | complete@bidir
  | Completeness@bidir

% Chapter 5

\knowledge{notion}
  | principal type
  | principal@type

% MetaCoq

\knowledge{notion}
  | GitHub

% Gradual CIC

\knowledge{notion}
  | gradual@typ
  | Gradual typing
  | gradual typing

\knowledge{notion}
  | consistency@grad

\knowledge{notion}
  | Fire Triangle of Graduality

\knowledge{notion}
  | Gradual Calculus of Inductive Constructions
  | GCIC

% Chapter 9

\knowledge{notion,text={CIC+$\axiom$}}
  | CICax

\knowledge{notion,text={CIC+$\err$}}
  | CICerr

\knowledge{notion}
  | ExTT