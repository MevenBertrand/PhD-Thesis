\chapter{Building a Certified Kernel}
\label{chap:kernel-correctness}

\margintoc

With the meta-theory set down, we can turn to building a kernel – and proving that it is
correct. The first step (\cref{sec:kernel-bidir})
is to move from the declarative specification of
\cref{chap:metacoq-general} to a bidirectional presentation,
closer to the kernel we wish to implement.
Once this specification is set down, we can get to the kernel itself.
\cref{sec:kernel-subroutines} goes over the implementation of the
\kl{cumulativity} check, and \cref{sec:kernel-typing} describes the type-checker.
Finally, \cref{sec:kernel-beyond-typing} describes two extra functions belonging to the safe
\kl{kernel}: re-typing, and checking of \kl{global environment}.

I personally contributed the formalizations of \cref{sec:kernel-bidir},
the completeness part of \cref{sec:kernel-typing} –
by modifying the pre-existing proven-correct type-checker –,
and heavily modified re-typing – part of \cref{sec:kernel-beyond-typing}.

\section{Formalized Bidirectional Typing}
  [Bidirectional Typing, Formalized]
\label{sec:kernel-bidir}

We already saw the important theoretical ideas around our approach to bidirectional typing in
\arefpart{bidir}, so let us get to their implementation in the formalization, file by file.

\subsection{Definitions}

Before we can get to the definition of typing, we must go through the small
\pcuicfile[Bidirectional][BD]{EnvironmentTyping}, which is dedicated to refining a few
definitions on contexts in the bidirectional setting. Indeed,
in the case of a definition \coqe{Γ,, vdef na b T},
\coqe{wf_local} enforces that \coqe{T} is a well-formed type, and that
\coqe{b} has type \coqe{T}. In the bidirectional setting we want to use constrained inference
$\pity{\uni}$ for the first, and checking for the second, but the generic definition
on which \coqe{wf_local} is built%
\sidenote{Called \coqe{All_local_env}.}
only allows for a single parameter – instantiated with \coqe{typing} in the case
of \coqe{wf_local}.
Similarly, we need a definition expression that a context $\Delta$ is well-formed
\emph{over another context $\Gamma$}, but which does not enforce $\Gamma$ to be well-formed
\textit{a priori} – \eg something more precise than simply $\vdash \Gamma , \Delta$. This
allows to stay faithful to McBride’s discipline%
\sidenote{Taken from \textcite{McBride2018}, see the explanation in \cref{sec:bidir-mcbride}.}%
\margincite{McBride2018}
when typing context extensions,
by only demanding that the extension is well-formed, but not the initial segment.%
\sidenote{Which is an input, and thus should not be re-checked.}

Bidirectional typing is given in
\pcuicfile[Bidirectional][BD]{Typing}, as a set mutual defined inductive predicates:
one for inference, one for checking, and three for constrained inference, respectively for
sorts, Π-types and inductive types. The definition of the predicates themselves
is very close to that of \cref{fig:ccw-bidir-infer,fig:bidir-ccw-other} for the
\kl{CCω} fragment, the main innovation being the fact that constrained inference – 
written \coqe{Σ ;;; Γ |- t |>Π (na,A,B)} for products –
takes the variable name, domain and codomain of the inductive type as three separate arguments,
which our pen-and-paper notation did not make explicit.
The predicates defined in \pcuicfile[Bidirectional][BD]{EnvironmentTyping} are used in the
definition of inference for the \coqe{tCase} node, where we want to ensure that the
context extensions used to type the \kl{predicate} and \kl{branches} are well-formed.

Regarding the notions of computation, the definitions of \kl{constrained inference} use
\kl{full reduction} rather than \kl{weak-head reduction}, mainly because MetaCoq currently
lacks an adequate treatment of the latter for our needs.%
\sidenote{In particular, a standardization theorem which would be needed to show the analogous
of \cref{lem:conv-red-tycons} for weak-head reduction.}
As for \kl{cumulativity}, the \kl(conv){algorithmic} variant is used in the \kl{checking} rule, but
this is relatively irrelevant, since the equivalence between both presentations of \kl{cumulativity} is
appears much earlier in the development than bidirectional typing.

Maybe more interesting from the formalization point of view is how we obtain a usable
induction principle. This is a common issue in \kl{MetaCoq}: while
\kl{Coq} is able to detect that our inductive definitions are well-founded,
the default generation is often unable to derive a sensible induction principle,
and neither are the \coqe{Scheme} specialized commands. This is due to their nested
character, \ie the presence of lists and records containing recursive instances
of the inductive types as arguments to the type constructors.
The bidirectional typing predicate is the paroxistic example of this, as it
reaches the limit of expressiveness offered by
\kl{Coq}’s inductive types: it is not only nested, but also mutual.
We thus have to prove our desired induction principle by
hand. To do so, we introduce a notion of “generic” typing object
\pcuicline[Bidirectional][BD]{Typing}{typing\_sum}{256},
together with a notion of size for such a typing object,
and finally show the induction principle
\pcuicline[Bidirectional][BD]{Typing}{bidir\_ind\_env}{312}
by well-founded induction on that size.

This induction principle is not as strong as we might expect, as it does not provide the extra induction hypothesis on inputs that would go with McBride's discipline.
Ideally, we could use this discipline
in order to thread the well-formation invariants, giving stronger induction
hypotheses. I did not try to take this path and prove such a strong induction principle,
as it did not seem so easy: it would effectively correspond to an inline proof
of \kl{validity}.
Instead, the discipline is reflected in the choice of the predicates proven by induction.
For instance, in the case of correctness, the mutually-proven predicate for inference
is \coqe{wf_local Σ Γ -> Σ ;;; Γ |- t : T}, and more generally
assumptions are added as pre-condition for all inputs.
Still, I conjecture that such a strong induction principle should be provable, if the need
would arise, and might be nice in order to factor proofs, by showing once and for all that
the rules follow the discipline correctly.

\subsection{Equivalence with Undirected Typing}

\kl(bidir){Correctness}%
\sidenote{Bidirectional typing implies undirected typing, akin to \cref{thm:corr-ccomega}.}
is shown in \pcuicfile[Bidirectional][BD]{ToPCUIC}.
The main proof is by induction on the derivation,
its key point being to show that well-formation invariants are preserved, and in particular
that all contexts that are used are valid.

There is one particular difficulty linked to the \coqe{tCase} constructor, and the question
of its representation evoked at the very end of \cref{sec:bidir-pcuic-inductives}.
More precisely, the issue is related to the fact that case nodes store the universe
instance and parameters of the inductive type being matched upon, in order to be able to
construct the context in which the \kl{predicate} and \kl{branches} are typed.
In undirected typing, the hypothesis on the \kl{scrutinee} is that it should be of some type
\coqline|mkApps (tInd ci.(ci_ind) p.(puinst)) (p.(pparams) ++ indices)|
where \coqe{ci.(ci_ind)}, \coqe{p.(puinst)} and \coqe{p.(pparams)}
are respectively the inductive type being matched upon, its universe instance, and its
parameters – all stored in the case node –, and \coqe{indices}
are free. From the point of view of bidirectional typing, this rule is invalid:%
\sidenote{It is not mode correct \cite{Dunfield2021}.}%
\margincite{Dunfield2021}
because \coqe{indices} is free, this cannot be turned into a checking premise, but it also
cannot be directly turned into inference, or even constrained inference, because it is
not free enough.
The solution is still to turn it into an inference premise
\coqe/Σ ;;; Γ |- c |>{ci} (u,args)/, and to compare the inferred universe instance \coqe{u}
and parameters – the first part of the list \coqe{args}. But it requires some reasoning to
show that this relation between both lists of parameters is enough to use the second part
of \coqe{args}, the inferred indices, in place of \coqe{indices} above.

In the opposite direction, \kl(bidir){completeness}%
\sidenote{Undirected typing implies bidirectional typing, akin to \cref{thm:compl-ccomega}.}
is also by a direct induction, once we have used
the injectivity properties of \pcuicfile{Conversion} to relate inference of a type
in \kl{cumulativity} with a sort, Π- or inductive type and constrained inference.
In order to simplify proofs in the case of projections, \kl(bidir){correctness} is used in conjunction
with \kl{validity}, but this could probably be avoided, making the two proofs independent.

\subsection{Properties of Bidirectional Typing}

As we did in \cref{thm:unique-inf}, we show that two inferred types have a common reduct in
\pcuicfile[Bidirectional][BD]{Unique}. While the proof requires some playing with
well-scoping predicates,%
\sidenote{
  To relate the reduction used to defined \kl{constrained inference} and the one on which most lemmas
  around confluence are stated, which is defined directly on well-scoped terms.}
it is conceptually \emph{much} simpler than the direct proof of \pcuicfile{Principality},
which shows the existence of \kl{principal types} without going through bidirectional typing.
Indeed, for quite some time only the weaker version saying that if $T$ and $T'$ are both types
for the same term $t$ then there exists a third $T''$ which is both a type for $t$ and smaller than 
$T$ and $T'$ for cumulativity.

Finally, \pcuicfile[Bidirectional][BD]{Strengthening} shows strengthening.
Its first important property is that if $\Gamma \vdash t \ity T$ then
$T$ can only use variables appearing in either $t$ or in the types of $\Gamma$
(\pcuicline[Bidirectional][BD]{Strengthening}{infering\_on\_free\_vars}{487}),
which is \emph{not} true in general in undirected typing.%
\sidenote{Consider for instance $n : \Nat \vdash 0 : (\l x : \Nat.\ \Nat)\ n$: $n$ appears in the
type, but neither in the body of the term nor any types in the context.}
It then goes on with the proof that bidirectional typing is stable under any renaming, while
\pcuicfile{RenameTyp} only shows stability of undirected typing under unconditional renaming.
Finally, we get to the proof of \pcuicline[Bidirectional][BD]{Strengthening}{strengthening}{989}
\textit{per se}, once we have shown that strengthening is indeed a well-formed renaming.

\section{Before Typing: Environment Querying and Cumulativity Checking}[Before Typing]
\label{sec:kernel-subroutines}

Before we can get to typing, we need to look at its two main sub-routines:
querying the \kl{global environment}, and \kl{cumulativity} check.

\subsection{Abstract Environment}

The type and \kl{cumulativity} checking algorithms both need to query the \kl{global environment},
for two main purposes: retrieving information about previous definitions of constants and inductive
types, and checking that (in)equalities between universes hold.

While this might seem anecdotal,
a surprisingly important amount of time is spent in the actual checker on the second problem, which
requires a form of shortest-path algorithm on a graph obtained from the universe constraints, in order
to detect the presence of negative cycles.
These correspond to violations of the universe stratification.
\kl{MetaCoq} implements such an algorithm, with a proof that it is correct and complete, meaning that
the algorithm answers “yes” exactly when there is a mapping from universe levels to integers satisfying
all constraints declared in the environment.
More details on this can be found in \sidetextcite[][Section 3.3]{Sozeau2020}.

Sadly, said algorithm is too naive to be actually run on reasonable examples: it is currently the main
performance bottleneck of the extracted program. Similarly, the representation of the
\kl{global environment} as a list of definitions is too naive to allow for efficient look-ups –
\kl{Coq} uses hash maps instead.
While we hope to replace that naive implementation with a more efficient but still certified one,
for the moment it is convenient to be able to plug an uncertified but efficient implementation
into the extracted type-checker. To do so, we rely on abstract interfaces for the
\kl{global environment}, containing all the possible queries we need to perform,
presented in \safefile{WfEnv}. The naive implementation is shown to be a valid implementation of
that interface in \safefile{WfEnvImpl}.

\subsection{Cumulativity Checking}

The most important sub-routine of the type-checker is the test of \kl{cumulativity} between two terms.
The naive way to perform this, since we assume \kl{normalization}, would be to brutally normalize terms,
and compare normal forms up to \coqe{leq_term}.%
\sidenote{The extension of \kl{α-equality} to handle \kl{cumulativity}.}
But this strategy does not scale as soon as definitions are present, because it
eagerly unfolds all of these, resulting in a very inefficient test.
\kl{MetaCoq} implements a more practical strategy, which coarsely does the following:
\begin{enumerate}
  \item reduce both terms to compare to \kl(red){weak-head} normal form
    \emph{without unfolding any definition};
  \item if the two heads match, recursively compare sub-terms;
  \item if the two heads do not match, or if the recursive sub-term comparison failed, check if
    an unfolding is possible which would unblock one of the terms, and if yes, unfold it and
    go back to the first step.
\end{enumerate}
This means that the \kl{cumulativity} test must itself resort to a \kl{weak-head reduction}
function.

The difficulty with those functions is that they do not operate by a simple structural induction on
terms. Rather, they are defined using a complex abstract machine, operating on terms decomposed into
a sub-term and a stack. The termination of that abstract machine is shown using a
dependent lexicographic pre-order, which handles both the well-founded reduction order given by
\kl{normalization}, a structural order on sub-terms and stacks corresponding to a given term, and
the different phases of the algorithm.
A detailed description of this algorithm and its formalization%
\sidenote{In files \safefile{SafeReduce} for the implementation of \kl{weak-head reduction}, and
\safefile{SafeConversion} for the cumulativity checker.} is given by Théo Winterhalter,
who implemented it, in his PhD thesis \sidecite[][Chapters 21-24]{Winterhalter2020}.

An interesting point that the test of \kl{cumulativity} and that of typing have in
common, is the way they handle their propositional content. First, because
we do not want at that point to bump into issues having to do with proof-irrelevance,
in most of the formalization definitions are in \coqe{Type},
including the reduction, conversion and typing relation. But in the verified \kl{kernel} we want that
separation between propositional and relevant content. Thus, we use explicit squashing —
written $\| T \|$ – to cast a type into a proposition.%
\sidenote{Technically,
  $\| T \|$ is defined as a record of type $\Prop$ with a single field of type $T$.}
The only elimination of propositional content into the relevant world we rely on is of course that
of accessibility, so that we can define \kl{reduction} and \kl{cumulativity} by well-founded induction.

Second, we write code using the \kl{Equations} plugin, which lets us write the
relevant part of the definition in direct style, but to leave proofs to be filled-in using the
proof-mode. The definitions are given in monadic style, relying on what looks like the error monad:%
\sidenote{Given a type of errors $E$, the functor associated with that monad is $T \mapsto T + E$,
its unit is the left injection, and its bind $x >>= f$ either propagates $x$ if it is an error, or applies $f$ otherwise.}
the \kl{cumulativity}- and type-checker return a valid output, or an error message. However, since
we wish the functions to be correct by constructions, they must also return a proof, either a
witness for the positive answer, or a proof of impossibility in the negative case.
This means that the bind of the monad must actually perform a proof when re-raising the error,
which propagates the impossibility proof.%
\sidenote{For instance, if we are typing some \coqe{tProd na A B} and the call to typing for \coqe{A} 
fails we must transform a proof that \coqe{A} cannot be well-typed into a proof that \coqe{tProd na A B}
as a whole also cannot.}
At function definition, this is hidden by notations, so this feels like we are actually using a
monad, but under the hood proof obligations are generated each time we use a bind.

\section{Correct and Complete Inference}
\label{sec:kernel-typing}

Given the work already done in \cref{sec:kernel-bidir}, the definition of a
type checking algorithm \safefile{TypeChecker}
itself is rather straightforward: it follows closely
the structure laid out by the mutually defined bidirectional judgments, and
poses no termination issue as \kl{cumulativity} does, since it
operates by induction on the structure of the term.
Actually, rather than a type-checker, the main function we define is
\safeline{SafeChecker}{infer}{1323}, which performs type inference.%
\sidenote{
  Which is decidable due to our \kl{Church-style} syntax.},
from which we can easily define type-checking.

\begin{figure}[h]
  \coqfile[firstline=1,lastline=4]{./code/PCUICSafeTyping.v}
  \caption{The “error monad” used for \safeline{SafeChecker}{infer}{1323}’s return type}
  \label{fig:meta-error-mon}
\end{figure}

In more details, the function takes as inputs:
\begin{enumerate}
  \item an abstract environment implementation;
  \item a \kl{global environment} following that implementation;
  \item a context, and a squashed proof that it is well-formed;
  \item a term
\end{enumerate}
and it returns either a type and a (squashed) proof that the term infers that type, or
an error and a proof that the term cannot infer any type, using the inductive type presented in
\cref{fig:meta-error-mon}.
Thus, the function is correct and complete by construction.
In fact, we cannot separate
the definition from the correctness proof, since the conversion checker
expects a well-typed term as input in order to be terminating when it is called.

\begin{figure*}
  \coqfile[firstline=6]{./code/PCUICSafeTyping.v}
  \caption{Definition of \safeline{SafeChecker}{infer}{1323} (excerpt)}
  \label{fig:meta-infer}
\end{figure*}

\cref{fig:meta-infer} describes (an excerpt of) the algorithm.
%
For a variable \coqe{tRel n}, it checks that the variable is bound in
\coqe{Gamma}, returns its type when it is, and fails otherwise.
%
For sorts \coqe{tSort u}, it checks that the universe is well-formed in the current
environment, and returns a sort at level above when it is.
%
For dependent products \coqe{tProd na A B}, it computes the sort of
\coqe{A} and \coqe{B} – in the context extended by
\coqe{na:A}–  using the \coqe{infer_type} sub-routine,%
\sidenote{Defined beforehand in an “open recursion” flavour, \eg as a function taking \coqe{infer}
as argument.}
and builds from those the sort of the product using \coqe{sort_of_product}.
Functions are similar.
%
The cases of \coqe{tLetIn} and \coqe{tApp} clearly show the bidirectional
structure. For instance in
\coqe{tApp t u}, one needs to infer the type \coqe{ty} of \coqe{t},
then reduce it to some \coqe{tProd na A B} using the
\coqe{reduce_to_prod} function, and finally check that \coqe{u} as
type \coqe{A}.
%
All underscores \coqe{_} in the terms denote proof obligations that are filled using
tactics. Although they are hidden, the monadic notations \coqe{;;}
and \coqe{raise} also contain underscores for the propagation of completeness
information.%
\sidenote{For instance, \coqe{raise e} is syntactic sugar for \coqe{TypeError_comp e _}.}

Interestingly, the proofs of \kl(bidir){completeness} use uniqueness of inferred types a lot.
To see why, consider \eg the case of an application $t\ u$ where the recursive call succeeds on $t$
– say it infers a product type $\P x : A.\ B$ – but it fails on $u$ – giving us a proof $p$
that $u \cty A$ is absurd. We want to raise an error, and thus need to prove
that $t\ u \ity T$ for any $T$ is absurd.
An inversion on that last hypothesis gives some $A'$ and $B'$ such that
$t \pity{\P} \P x : A'.\ B'$ and $u \cty A'$.
But this second property cannot be directly fed $p$, because the type against which $u$ checks is
different! We thus need to use the two inference judgments and uniqueness to conclude that in
fact $A \conv A'$, and thus that $u \cty A$,
which this time we can the use to derive a contradiction from $p$.

\section{Beyond Typing: Environment Checking and Re-Typing}
\label{sec:kernel-beyond-typing}

There are two more functionalities defined in \kl{MetaCoq}
that are very related to the type-checker.

\subsection{Re-Typing}

The first, which is defined in \safefile{SafeRetyping}, aims at computing a type for a term
which is known to be well-typed. While this seems tautological, it is not: the aim is
to extract relevant content out of propositional one. This is useful in practice in \eg
the extraction procedure, which maintains the invariant that it operates on well-typed terms,
but needs to actually compute types to decide whether terms should be erased.

This is also different from standard inference,
because knowing \textit{a priori} that the term under consideration is well-typed allows to
skip a lot of checks. For instance, to re-type an application $t\ u$, it suffices to infer a
product type $\P x : A.\ B$ for $t$, and to return $\subs{B}{x}{u}$, since we know that $u$
has type $A$.

In order to be useful, this re-typing procedure needs to compute a \kl{principal type}, and
thus its definition was quite complex prior to the formalization of bidirectional typing,
effectively inlining a proof of uniqueness of types. Instead, bidirectional typing
simplifies greatly both the definition of re-typing and its proof of correctness, by
simplifying the specification of re-typing.
Instead of computing a principal type out of any type,
the function needs to compute an unsquashed inferred type out of a squashed one.

\subsection{Environment Checking}

The second thing we need to handle is the verification that a whole \kl{global environment}
is well-formed. While the main thing to check is that all definitions are well-typed,
there is quite a few more things to be done: checking universes constraints, that inductive
definitions are strictly positive, that the variance information used by universe polymorphism is
valid… All these are covered in \safefile{SafeChecker}.

