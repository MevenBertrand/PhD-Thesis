%----------------------------------------------------------------------------------------
%	BOOK INFORMATION
%----------------------------------------------------------------------------------------

% \titlehead{Document Template}
\title[Bidirectional Typing in the Calculus of Inductive Constructions]{Bidirectional Typing in the Calculus of Inductive Constructions}
\subtitle{PhD Thesis}
\author[M. Lennon-Bertrand]{Meven Lennon-Bertrand}
\date{\today}
% \publishers{An Awesome Publisher}

%----------------------------------------------------------------------------------------

\frontmatter % Denotes the start of the pre-document content, uses roman numerals

%----------------------------------------------------------------------------------------
%	COPYRIGHT PAGE
%----------------------------------------------------------------------------------------

\makeatletter
\uppertitleback{\@titlehead} % Header

\lowertitleback{
	% \textbf{Disclaimer} \\
	% You can edit this page to suit your needs. For instance, here we have a no copyright statement, a colophon and some other information. This page is based on the corresponding page of Ken Arroyo Ohori's thesis, with minimal changes.
	
	% \medskip
	
	% \textbf{No copyright} \\
	% \cczero\ This book is released into the public domain using the CC0 code. To the extent possible under law, I waive all copyright and related or neighbouring rights to this work.
	
	% To view a copy of the CC0 code, visit: \\\url{http://creativecommons.org/publicdomain/zero/1.0/}
	
	% \medskip
	
	% \textbf{Colophon} \\
	% This document was typeset with the help of \href{https://sourceforge.net/projects/koma-script/}{\KOMAScript} and \href{https://www.latex-project.org/}{\LaTeX} using the \href{https://github.com/fmarotta/kaobook/}{kaobook} class.
	
	% \medskip
	
	% \textbf{Publisher} \\
	% First printed in May 2019 by \@publishers
}
\makeatother

%----------------------------------------------------------------------------------------
%	DEDICATION
%----------------------------------------------------------------------------------------

\dedication{
	% The harmony of the world is made manifest in Form and Number, and the heart and soul and all the poetry of Natural Philosophy are embodied in the concept of mathematical beauty.\\
	% \flushright -- D'Arcy Wentworth Thompson
}

%----------------------------------------------------------------------------------------
%	OUTPUT TITLE PAGE AND PREVIOUS
%----------------------------------------------------------------------------------------

% Note that \maketitle outputs the pages before here
\maketitle

%----------------------------------------------------------------------------------------
%	PREFACE
%----------------------------------------------------------------------------------------

\addchap{Abstract}

“\kl{Coq} is an old man now, and it has a lot of scars”
\cite[citing Assia Mahboubi]{QuantaPA}.
Proof assistants have been around for 50 years now, and they have become a more and
more established technology over time. This history is both a blessing and a curse: as
the field matured, the tools have become more and more complex, making them more and more
powerful, but also more and more prone to critical bugs hiding in dark corners. At a time
when they are gaining traction in an increasing number of communities,
especially those concerned with high safety and security guarantees, this simply cannot be.
The historical solution of keeping a small, trusted \kl{kernel}
– the so-called de Bruijn criterion –
is  simply not enough if we wish to keep moving on and integrate more new complex features
to keep up with the needs of users.

There is a straightforward solution to this:
proof assistants have been used for decades to prove
programs correct. Why could they not prove \emph{themselves} correct? After all, if this is
the golden standard we demand for software, it should apply first and foremost to the ones
used to justify that trust. For the proof assistant \kl{Coq},
this is the ambition of the \kl{MetaCoq} project,
which aims at providing a drop-in replacement for \kl{Coq}’s \kl{kernel}, but one that has been
fully proven correct, even though it handles all the subtleties and quirks of said \kl{kernel}.
No more trusting a complex and ever-evolving implementation, trust the formally validated
\emph{proofs} instead!

But before we can hope to achieve that goal, we need more study of the structures at work
in the \kl{kernel}. In particular, its typing algorithm is \emph{bidirectional}, meaning that
it constantly alternates between the two problems of type \emph{inference} – finding a type for
a term – and type \emph{checking} – verifying that a type is adequate for a term. While this
structure is crucial in relating the specification of the type system to its implementation,
it has been rather little studied in the context of the \kl{Calculus of Inductive Constructions}
(\kl{CIC}), the theoretical foundation of \kl{Coq} – but also of the closely related
\kl{Agda}, \kl{Lean}…

This thesis aims at filling that gap, by providing a thorough study of this bidirectional
structure, formalized in the setting of the \kl{MetaCoq} project. This is a key
ingredient in the first formal proof of correctness and completeness of a type-checking
algorithm for (a significant subset of) \kl{Coq}, which was able to catch bugs that had gone
unnoticed until then.
But it is also an interesting theoretical tool in its own right, giving a form of control
over computation that turns out to be crucial in quite a few situations.

In particular, bidirectional typing is a necessary piece in the design of a gradual extension of
\kl{CIC}, \kl{GCIC}.
\kl{Gradual typing} aims at giving programmers in one and the same system the flexibility of
development offered by dynamic typing, and the strong guarantees offered
by static typing. \kl{GCIC} intends
to bring that flexibility to dependently-typed programming,
and, by using the power of the \kl{Curry-Howard correspondence}, to proof writing.
But this flexibility comes with subtle difficulties, that can only be solved in a bidirectional
setting.

The first part of the thesis gives a theoretical account of bidirectional typing; the second
describes the formalization of the ideas of the first section in the setting of \kl{MetaCoq};
the third and last introduces \kl{GCIC} and its properties.


\addchap{How to read this thesis}

This thesis has been written primarily for screen reading, as I think this will be the medium
used by most of my viewers. Therefore, I use hyperlinking as much as possible
inside the document.

While it is not visible – in order to keep the text readable by not being too distracting –,
most technical keywords are actually linked to the
place of their definitions. For instance \kl{bidirectional typing} links to
the place in \cref{chap:intro-en} where the notion is introduced.
This definition itself is put into emphasis like the following \intro{example}.
I might cheat a bit and introduce a notion twice, once on a high level in an introductory
section, and a second time precisely later on, in which case the link point to the precise
definition. Most notations are also linked: no more wondering what the symbol $\obsRef$
means again, just click on it!

The main text has large margins, which I use and abuse 
for notes, small figures and references. Hopefully this reduces the need to go
back and forth between the main text and information much too far away.
Regarding figures, rather than having large, bulky ones that take a full page,
I tried to keep them as close as possible to their explanation. This often means that
they are split in multiple small fragments, so that each part of the figure goes with its
explanation. In such cases, the fragments should really be understood
as different parts of one and the same figure. To indicate this, the fragments share the same
figure number, such as
\cref{fig:cic-var,fig:cic-nondep-fun,fig:cic-dep-fun,fig:cic-univ,fig:cic-prod,fig:cic-con,fig:cic-conv} which all define the same system, one rule at a time.



%----------------------------------------------------------------------------------------
%	TABLE OF CONTENTS & LIST OF FIGURES/TABLES
%----------------------------------------------------------------------------------------
\knowledgeconfigure{protect links}

\begingroup % Local scope for the following commands

\hypersetup{allcolors=.}

% Define the style for the TOC, LOF, and LOT
%\setstretch{1} % Uncomment to modify line spacing in the ToC
%\hypersetup{linkcolor=blue} % Uncomment to set the colour of links in the ToC
\setlength{\textheight}{230\vscale} % Manually adjust the height of the ToC pages

% Turn on compatibility mode for the etoc package
\etocstandarddisplaystyle % "toc display" as if etoc was not loaded
\etocstandardlines % "toc lines as if etoc was not loaded
\setcounter{tocdepth}{\sectiontocdepth} % Locally for the global toc

\tableofcontents % Output the table of contents

\setcounter{tocdepth}{\subsectiontocdepth} % To have correct tocs in the sections

% \listoffigures % Output the list of figures

% Comment both of the following lines to have the LOF and the LOT on different pages
% \let\cleardoublepage\bigskip
% \let\clearpage\bigskip

% \listoftables % Output the list of tables

\endgroup
\knowledgeconfigure{unprotect links}
