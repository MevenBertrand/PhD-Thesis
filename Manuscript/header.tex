%----------------------------------------------------------------------------------------
%	BOOK INFORMATION
%----------------------------------------------------------------------------------------

% \titlehead{Document Template}
\title[Bidirectional Typing in the Calculus of Inductive Constructions]{Bidirectional Typing in the Calculus of Inductive Constructions}
\subtitle{Doctoral Thesis}
\author[M. Lennon-Bertrand]{Meven Lennon-Bertrand}
\date{\today}
% \publishers{An Awesome Publisher}

%----------------------------------------------------------------------------------------

\frontmatter % Denotes the start of the pre-document content, uses roman numerals

%----------------------------------------------------------------------------------------
%	COPYRIGHT PAGE
%----------------------------------------------------------------------------------------

\makeatletter
\uppertitleback{\@titlehead} % Header

\lowertitleback{
	% \textbf{Disclaimer} \\
	% You can edit this page to suit your needs. For instance, here we have a no copyright statement, a colophon and some other information. This page is based on the corresponding page of Ken Arroyo Ohori's thesis, with minimal changes.
	
	% \medskip
	
	% \textbf{No copyright} \\
	% \cczero\ This book is released into the public domain using the CC0 code. To the extent possible under law, I waive all copyright and related or neighbouring rights to this work.
	
	% To view a copy of the CC0 code, visit: \\\url{http://creativecommons.org/publicdomain/zero/1.0/}
	
	% \medskip
	
	% \textbf{Colophon} \\
	% This document was typeset with the help of \href{https://sourceforge.net/projects/koma-script/}{\KOMAScript} and \href{https://www.latex-project.org/}{\LaTeX} using the \href{https://github.com/fmarotta/kaobook/}{kaobook} class.
	
	% \medskip
	
	% \textbf{Publisher} \\
	% First printed in May 2019 by \@publishers
}
\makeatother

%----------------------------------------------------------------------------------------
%	DEDICATION
%----------------------------------------------------------------------------------------

\dedication{\raggedleft\emph{À Tangi,\\
Parce que je sais que tu aurais été le premier à te réjouir avec moi de là où j’en suis arrivé…
et le premier à me taquiner pour me garder les pieds sur terre.}}

% \dedication{
% 	% The harmony of the world is made manifest in Form and Number, and the heart and soul and all the poetry of Natural Philosophy are embodied in the concept of mathematical beauty.\\
% 	% \flushright -- D'Arcy Wentworth Thompson
% }

%----------------------------------------------------------------------------------------
%	OUTPUT TITLE PAGE AND PREVIOUS
%----------------------------------------------------------------------------------------

\includepdf[pages=1]{../MathSTICTemplate/main.pdf}

% Note that \maketitle outputs the pages before here
\maketitle

%----------------------------------------------------------------------------------------
%	PREFACE
%----------------------------------------------------------------------------------------

\begingroup

  %in the header, chapters start on any page to avoid too many blank pages
  \let\cleardoublepage\clearpage
  \pagelayout{margin}

\addchap{Abstract}

This thesis broadly considers the question of giving a bidirectional treatment of the
Calculus of Constructions (CIC), which underpins the proof assistant \kl{Coq}, under
three different angles corresponding to its three parts.

It first considers the question of giving a bidirectional account of CIC
from a theoretical point
of view. It contains the exposition of such a bidirectional presentation of CIC, with the
general discipline that led to it. Follow a proof of equivalence between this presentation
and the standard one. This equivalence is then used to establish properties of CIC that
are hard to obtain in the standard setting: existence of principal types, and strengthening.

The second part sets on to formalize the idea of the first one,
in the setting of the \kl{MetaCoq} project,
which aims at formalizing the meta-theory CIC in \kl{Coq}, and to implement
a kernel that is proven sound and complete. The formalized bidirectional structure supplies
an intermediate between the high-level specification and the algorithm, which is key in
order to prove that the kernel is complete.

Finally, the last part considers the question of designing an extension of CIC incorporating
ideas from gradual typing, with the aim of bringing more flexibility to development in \kl{Coq}.
The bidirectional structure is once again valuable, as the characteristics of gradual typing
– in particular the way it relaxes conversion – 
make it impossible to base the extension on the standard presentation of CIC.

\addchap{Acknowledgments}

I wouldn’t have got halfway to the end of this thesis without the many great people around me,
so let me try and thank them here for all they have brought me.

Obviously, the one person this thesis owes the most to is my advisor Nicolas. I feel extremely
lucky to have had the possibility to run free and do my things, while still knowing that
he was in the next office with an open door whenever I got an issue.
I am proud that I can call myself his student.

Beyond Nicolas, I am very honoured that my jury members accepted to be part of it.
Neel and Conor,
because of their deep knowledge about bidirectional typing and programming language
theory in general.
Herman, because my year in Nijmegen has been an important one for where I am today.
Christine and Hugo, because they shaped the great tool that \kl{Coq} is today; and, for Hugo,
because he was the one who introduced me both to \kl{Coq} and to research.
Jesper, because I have a great respect his work on \kl{Agda}.
I hope there are still many pages to
write in the collaboration between our two close but yet subtly different worlds.
Matthieu, because his work on \kl{MetaCoq} is impressive, and he still is kind enough to
let me leave him discharge on him my ugliest lemmas.
Finally, Assia, because I admire both her achievements and her genuine humanity.

During these three and a half years, I have learned so much from the researchers around me,
and I am very grateful for that.
From Pierre-Marie, that you should not be afraid to stand by your ideas.
From Assia, that this does not mean you have to crush others,
but that you can instead learn a great deal from them, exactly because they took another path.
From Guillaume, that there are some non-negotiable values in academia that you cannot just 
ignore.
From Guilhem, I hopefully learned a bit about teaching. Thanks for helping me navigate the
meanders of the university and for trusting me with your exercise sessions.
From Matthieu, I learned most of what I know about another kind of meanders,
those of \kl{Coq} and formalization.
Éric taught me that having a solid technique
is only useful if readers get that far in your paper, and showed me how to achieve that.

But Gallinette is not only permanent researchers, and I learned an equally great deal from
my \textit{senpais}. Marie and Étienne came up with the brilliant idea of the Quésaco seminars.
Yannick still patiently answer my random questions, even after so many of them.
Théo guided me through the intricacies of the PhD, which is no small feat.
Loïc does not really fit in the \textit{senpai} category,
after all of the two thesis siblings I am
the one coming out first. Still, I learned a lot with him, from precious insight on
normalization to painting beautiful banners.
Good luck with your own writing, and see you on the other side!
Last but not least, Kenji taught me just too many things to be all recorded here.
I’ll miss our random but always worthwhile exchanges between computer screens and your
valuable ideas on dependent types and vegetables alike.

Learning is nice, but it’s not everything. Happily, the Gallinette members are not just
cool researchers, they are also wonderful people, and it’s been an immense pleasure to work
in that team. Thanks to all those I have already mentioned and to
Pierre (Vial), Maxime, Simon, Matthieu (Piquerez), Ambroise, Xavier, Martin, Enzo,
Pierre (Benjamin? Giraud? How do I cite you?), Hamza, Chris, Gaëtan
for the seminars, coffee discussion, corridor talks, online chats, beer disputes,
and all the other conversations. A special acknowledgment to Ambroise for thanking me in
advance in his thesis, the prophecy has been accomplished.
Following this now established tradition, I hereby thank in advance
Martin, Enzo, Pierre, Hamza and Chris for thanking me in their own theses when the time comes.
Best of luck to get there.
Another special thanks to those who proofread this thesis even with the rushy calendar,
and especially to Martin who found the courage to go through all of it. You made it much
better than it was.
And let me not forget Anne-Claire, who kindly watches over our unruly lot.

\selectlanguage{french}

Au-delà de Gallinette, je me dois aussi de mentionner les doctorant·e·s du LS2N.
Entre le Covid et la rédaction, je n’ai que trop peu profité d’eux, mais quelle
chouette équipe !
Plus loin de Nantes, je n’oublie pas mon camarade Curry-Howardien Rémy.
Un jour ce livre de théorie des catégories verra le jour, j’en suis sûr.
Merci aussi à toute l’équipe des CHATS, Sylvain, Bertrand, Rémi,
Brigitte, Audrey, Marianne, Baptiste, Mélanie, et tou·te·s les profs et élèves du lycée
Michelet. Ces trois années d’art et de sciences n’ont pas exactement été de tout repos,
mais j’ai beaucoup de plaisir à partager un peu de mon enthousiasme pour
les mathématiques et l’informatique, et de tant recevoir en retour.
Enfin, j’ai été très heureux de faire partie de l’éphémère collectif des précaires à l’hiver
2020, hélas presque aussi vite dispersé qu’il s’est formé.
% Où que le vent de la lutte vous
% porte, j’espère que vous continuerez à semer des petites graines de révolte.
Merci à eux et à tous les autres qui luttent pour faire de l’université et
du monde des endroits un peu meilleurs.

\selectlanguage{english}

These pandemics years have not exactly been the best for exchanges. Yet, I feel very
fortunate to be part of the thriving proof assistants and types communities. Thanks for
all the online questions, answers and discussions, for giving nice talks and listening to
mine. Hope to finally meet you all in person soon. And thanks to Andrej for giving me the
occasion to have the first (!) scientific trip of my PhD in Ljubljana.

\selectlanguage{french}

Je n’aurais jamais pu arriver là où j’en suis sans les nombreux·ses enseignant·e·s
qui ont croisé ma longue route de la maternelle au master. Ils et elles ont su nourrir
ma curiosité, et me montrer que ce n’est pas parce qu’un sujet est difficile que son
apprentissage doit être dur. Je ne saurais toutes et tous les nommer, mais je veux en
particulier remercier Jean-Michel Rey et François Sauvageot pour m’avoir fait entrer dans
le monde des grandes personnes en posant les fondations sur lesquelles tout le reste s’est
construit, et Daniel Hirschkoff pour avoir veillé avec bienveillance sur mes années à l’ENS
et m’avoir un jour parlé de cette petite équipe sympathique qui fait du Coq à Nantes.
\selectlanguage{english}
I have already mentioned Herman, but let me extend the praise to Freek, Jurriaan, and
my other Dutch professors, for having been great teachers and having introduced me to proof
assistants, type theory, and many of the ideas I still use today.

\selectlanguage{french}

Si on en croit von Neumann, si les gens ne croient pas que les mathématiques sont simples,
c’est qu’ils ne réalisent pas à quel point la vie est compliquée. Fort heureusement,
je suis au moins aussi bien entouré dans la vie qu’en mathématiques.

Bien sûr, par ma famille. Merci évidemment à mes parents
d’avoir été là pour moi depuis toujours, de m’avoir soutenu de toutes les manières possibles,
dans tous mes choix. Et d’avoir été
d’attentifs et exigeants lecteurs et relecteurs de cette introduction. Merci à Tangi pour
les leçons si profondes et les moments si légers. Et pour être toujours avec moi.
Merci à Malo et Maïan d’avoir repris avec brio le flambeau de la taquinerie de ses mains,
et à Marc et Karine de nous avoir fait grandir, eux et moi. Enfin, merci à Papi de toujours
croire en ma réussite avec une confiance sans faille, même sans y comprendre grand-chose.

Je serais devenu fou si je n’avais pas pu sortir de mon bureau pour prendre un peu de grand
air, merci donc à Karlijn (\foreignlanguage{english}{not sure our trips
qualify as simply “a little fresh air” though}),
Olivier (pour le trad et pour les olives), Daniel (à quand la grande voie en 7?),
Manon (les force 5, quelle aventure), Jodie, Émile, Alice,
et Sacha. J’accepterai un jour qu’avec les Zouzous on passe plus de temps
de part et d’autre d’une table bien garnie que d’une corde à double.
Merci à Léo pour avoir été non seulement un excellent compagnon de cordée, mais aussi un
compagnon de mathématiques, de musique, bref de vie ! Et à Paul de compléter avec brio
le trio et pour sa curiosité scientifique toujours passionnante.
Comment oublier les lyonnais qui ont toujours eu une place pour moi sur un canapé
quand je les rejoignais pour une soirée de fête ou à jouer (Dune c’est le feu).
Merci donc à Hugo, Angèle, Solène, Audrey, Clément, Loïs, Colin
et tous les autres. 
Enfin, comment parler de soirées sans évoquer celles à la Milleterie, organisées et
surtout illustrées avec un talent rare par Valentin.

L’évasion n’est pas tout, j’ai aussi été entouré par une équipe de choc dans ma vie nantaise.
Allan et Maxou sont aussi généreux comme hôtes que féroces aux aventuriers du rail, et
c’est parfait ainsi.
Marine, qui même si elle réussit l’exploit d’être encore plus busy que moi reste une ancre
à laquelle je sais que je peux me raccrocher.
Mes colocataires, Julien, Laurine et Martin, pour avoir partagé ma
vie et mon quotidien, les rires et les moments moins funs, et avoir survécu aux confinements
ensemble sans qu’on s’entretue. Lucie, qui à ce stade rentre
presque dans la catégorie précédente (la team Olivettes restera).
Enfin merci évidemment à Jo, qui est à la fois une colocataire, une ancre, une féroce joueuse,
une compagne d’aventure, et bien plus encore, pour m’avoir nourri pendant ma rédaction,
traîné à la mer, ou pallié mon inculture cinématographique quand je n’en pouvais plus…
Hâte de lire tes remerciements à toi, j’espère que j’y serai en aussi bonne place.

\selectlanguage{english}

\addchap{How to Read This Thesis}

As I think screen reading will be the medium
used by most of my readers, I use hyperlinking as much as possible
inside the document.
While it is not visible – in order to keep the text readable –,
most technical keywords are actually linked to the
place of their definitions. For instance, \kl{bidirectional typing} links to
the place in \cref{chap:intro-en} where the notion is introduced.
This definition itself is put into emphasis like the following \intro{example}.
I might cheat a bit and introduce a notion twice, once on a high level in an introductory
section, and a second time precisely later on, in which case the link point to the precise
definition. Most notations are also linked: if you wonder what the symbol $\obsRef$
means again, just click on it!

The main text has large margins, which I use and abuse 
for notes, small figures and references. Hopefully this reduces the need to go
back and forth between the main text and information too far away.
Regarding figures, rather than having large, bulky ones,
I tried to keep them as close as possible to their explanation. This means that
they are often split in multiple small fragments,
so that each part of the figure goes with its
explanation. In such cases, the fragments should really be understood
as different parts of one and the same figure. To indicate this, the fragments share the same
figure number, such as
\cref{fig:cic-var,fig:cic-nondep-fun,fig:cic-dep-fun,fig:cic-univ,fig:cic-prod,fig:cic-con,fig:cic-conv} which define a single system, one rule at a time.

Finally, although I primarily intend this document to be read on screen, I tried to
keep it adapted for printing. In particular,
no information should be conveyed using only colour, though I use it to ease readability.

\endgroup

%----------------------------------------------------------------------------------------
%	TABLE OF CONTENTS & LIST OF FIGURES/TABLES
%----------------------------------------------------------------------------------------

\pagelayout{wide}
\cleardoubleevenemptypage

\knowledgeconfigure{protect links}

\begingroup % Local scope for the following commands

\hypersetup{allcolors=.}

% Define the style for the TOC, LOF, and LOT
%\setstretch{1} % Uncomment to modify line spacing in the ToC
%\hypersetup{linkcolor=blue} % Uncomment to set the colour of links in the ToC
\setlength{\textheight}{230\vscale} % Manually adjust the height of the ToC pages

% Turn on compatibility mode for the etoc package
\etocstandarddisplaystyle % "toc display" as if etoc was not loaded
\etocstandardlines % "toc lines as if etoc was not loaded
\setcounter{tocdepth}{\sectiontocdepth} % Locally for the global toc

\tableofcontents % Output the table of contents

\setcounter{tocdepth}{\subsectiontocdepth} % To have correct tocs in the sections

% \listoffigures % Output the list of figures

% Comment both of the following lines to have the LOF and the LOT on different pages
% \let\cleardoublepage\bigskip
% \let\clearpage\bigskip

% \listoftables % Output the list of tables

\endgroup
\knowledgeconfigure{unprotect links}
