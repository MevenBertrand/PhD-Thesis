%----------------------------------------------------------------------------------------
%	BOOK INFORMATION
%----------------------------------------------------------------------------------------

% \titlehead{Document Template}
\title[Bidirectional Typing in the Calculus of Inductive Constructions]{Bidirectional Typing in the Calculus of Inductive Constructions}
\subtitle{Doctoral Thesis}
\author[M. Lennon-Bertrand]{Meven Lennon-Bertrand}
\date{\today}
% \publishers{An Awesome Publisher}

%----------------------------------------------------------------------------------------

\frontmatter % Denotes the start of the pre-document content, uses roman numerals

%----------------------------------------------------------------------------------------
%	COPYRIGHT PAGE
%----------------------------------------------------------------------------------------

\makeatletter
\uppertitleback{\@titlehead} % Header

\lowertitleback{
	% \textbf{Disclaimer} \\
	% You can edit this page to suit your needs. For instance, here we have a no copyright statement, a colophon and some other information. This page is based on the corresponding page of Ken Arroyo Ohori's thesis, with minimal changes.
	
	% \medskip
	
	% \textbf{No copyright} \\
	% \cczero\ This book is released into the public domain using the CC0 code. To the extent possible under law, I waive all copyright and related or neighbouring rights to this work.
	
	% To view a copy of the CC0 code, visit: \\\url{http://creativecommons.org/publicdomain/zero/1.0/}
	
	% \medskip
	
	% \textbf{Colophon} \\
	% This document was typeset with the help of \href{https://sourceforge.net/projects/koma-script/}{\KOMAScript} and \href{https://www.latex-project.org/}{\LaTeX} using the \href{https://github.com/fmarotta/kaobook/}{kaobook} class.
	
	% \medskip
	
	% \textbf{Publisher} \\
	% First printed in May 2019 by \@publishers
}
\makeatother

%----------------------------------------------------------------------------------------
%	DEDICATION
%----------------------------------------------------------------------------------------

% \dedication{
% 	% The harmony of the world is made manifest in Form and Number, and the heart and soul and all the poetry of Natural Philosophy are embodied in the concept of mathematical beauty.\\
% 	% \flushright -- D'Arcy Wentworth Thompson
% }

%----------------------------------------------------------------------------------------
%	OUTPUT TITLE PAGE AND PREVIOUS
%----------------------------------------------------------------------------------------

\includepdf[pages=1]{../MathSTICTemplate/main.pdf}

% Note that \maketitle outputs the pages before here
\maketitle

%----------------------------------------------------------------------------------------
%	PREFACE
%----------------------------------------------------------------------------------------

\addchap{Abstract}

This thesis broadly considers the question of giving a bidirectional treatment of the
Calculus of Constructions (CIC), which underpins the proof assistant \kl{Coq}. It is broadly
divided in three parts.

The first considers the question of giving a bidirectional account of CIC from a theoretical point
of view. It contains the exposition of such a bidirectional presentation of CIC, with the
general discipline that led to it. Follow a proof of equivalence between this presentation
and the standard one. This equivalence is then used to establish properties of CIC that
are hard to obtain in the standard setting.

The second part sets on to formalize the idea of the first one, in the setting of the \kl{MetaCoq}
project. This project aims at formalizing the meta-theory CIC in \kl{Coq}, and to implement
a kernel that is proven correct and complete. The formalized bidirectional structure supplies
an intermediate between the high-level specification and the algorithm, which is key in
order to prove that the kernel is complete.

Finally, the last part considers the question of designing an extension of CIC incorporating
ideas from gradual typing, with the aim of bringing more flexibility to development in \kl{Coq}.
The gradual structure is once again valuable, as the characteristic of gradual typing 
make it impossible to base the extension on the standard presentation of CIC.

\addchap{How to read this thesis}

As I think screen reading will be the medium
used by most of my readers, I use hyperlinking as much as possible
inside the document.
While it is not visible – in order to keep the text readable by not being too distracting –,
most technical keywords are actually linked to the
place of their definitions. For instance \kl{bidirectional typing} links to
the place in \cref{chap:intro-en} where the notion is introduced.
This definition itself is put into emphasis like the following \intro{example}.
I might cheat a bit and introduce a notion twice, once on a high level in an introductory
section, and a second time precisely later on, in which case the link point to the precise
definition. Most notations are also linked: if you wonder what the symbol $\obsRef$
means again, just click on it!

The main text has large margins, which I use and abuse 
for notes, small figures and references. Hopefully this reduces the need to go
back and forth between the main text and information much too far away.
Regarding figures, rather than having large, bulky ones that take a full page,
I tried to keep them as close as possible to their explanation. This often means that
they are split in multiple small fragments, so that each part of the figure goes with its
explanation. In such cases, the fragments should really be understood
as different parts of one and the same figure. To indicate this, the fragments share the same
figure number, such as
\cref{fig:cic-var,fig:cic-nondep-fun,fig:cic-dep-fun,fig:cic-univ,fig:cic-prod,fig:cic-con,fig:cic-conv} which all define the same system, one rule at a time.

Finally, although I primarily intend this document to be read on screen, I tried to
keep it adapted for printing. In particular,
no information should be conveyed using only colour, though I use it to ease readability.

%----------------------------------------------------------------------------------------
%	TABLE OF CONTENTS & LIST OF FIGURES/TABLES
%----------------------------------------------------------------------------------------
\cleardoubleevenemptypage

\knowledgeconfigure{protect links}

\begingroup % Local scope for the following commands

\hypersetup{allcolors=.}

% Define the style for the TOC, LOF, and LOT
%\setstretch{1} % Uncomment to modify line spacing in the ToC
%\hypersetup{linkcolor=blue} % Uncomment to set the colour of links in the ToC
\setlength{\textheight}{230\vscale} % Manually adjust the height of the ToC pages

% Turn on compatibility mode for the etoc package
\etocstandarddisplaystyle % "toc display" as if etoc was not loaded
\etocstandardlines % "toc lines as if etoc was not loaded
\setcounter{tocdepth}{\sectiontocdepth} % Locally for the global toc

\tableofcontents % Output the table of contents

\setcounter{tocdepth}{\subsectiontocdepth} % To have correct tocs in the sections

% \listoffigures % Output the list of figures

% Comment both of the following lines to have the LOF and the LOT on different pages
% \let\cleardoublepage\bigskip
% \let\clearpage\bigskip

% \listoftables % Output the list of tables

\endgroup
\knowledgeconfigure{unprotect links}
