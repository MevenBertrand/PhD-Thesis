% \pagelayout{fullwidth}
% \includegraphics{./figures/test.pdf}
% \pagelayout{margin}

\chapter{Formalized Meta-Theory of \kl{PCUIC}}
\label{chap:metacoq-general}

\margintoc

Before we can attempt to build a certified kernel, we need a thorough meta-theoretical study
of the type system. This is necessary in order to show that the
kernel preserves invariants – typically, well-typedness of the objects it manipulates.
The use of these invariants goes beyond correctness:
the conversion check performs evaluation of terms, and, since all functions in \kl{Coq}
must be terminating,
this evaluation is defined by well-founded induction on the normalization of well-typed terms.
Since evaluation is not normalizing on ill-typed terms, the mere \emph{definition} of the
conversion check relies on \kl{preservation} to be able to iterate reduction steps.

The properties under scrutinee here are not new,
and neither are the basic strategy of most proofs.
Indeed, the development roughly follows the architecture we already exposed in
\cref{sec:tech-properties}. The main difficulty is the scale of the proofs: due to the
complexity of \kl{PCUIC}, even well-understood techniques are challenging to apply.
Moreover, subtleties that do not appear in a simpler setting become apparent –
typically pertaining to universe levels or general inductive types –,
demanding original ideas. Thus, rather than getting lost in
the gory details of the formalisation which are best understood by looking at
– and maybe replaying – it, we try and focus on these interesting subtleties.

In more details, we start with the main definitions : the syntax, and conversion and typing
judgments (\cref{sec:pcuic-defs}). We follow (\cref{sec:pcuic-stabilities}) by the basic
stability properties: renaming, substitution, environment extension, etc.
Next comes the first important proof, that of confluence, and its multiple consequences
(\cref{sec:pcuic-confluence}). This leads to the properties pertaining to typing, culminating
with subject reduction (\cref{sec:pcuic-typing-prop}).
Finally, we discuss the place of normalization.

\section[Terms, Conversion and Types]{Setting up the Definitions: Terms, Conversion and Types}
\label{sec:pcuic-defs}


\section[Stabilities]{The Easy Properties: Stabilities}
\label{sec:pcuic-stabilities}

\section[Confluence]{Things Get Serious: Confluence}
\label{sec:pcuic-confluence}

\section[Properties of Typing]{Reaping the Fruits: Properties of Typing}
\label{sec:pcuic-typing-prop}

\section[Normalization]{Gödel’s Thorn in the Side: Normalization}
\label{sec:pcuic-normalization}