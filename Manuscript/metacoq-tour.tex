% \pagelayout{fullwidth}
% \includegraphics{./figures/test.pdf}
% \pagelayout{margin}

\chapter{Formalized Meta-Theory of \kl(tit){PCUIC}}
\label{chap:metacoq-general}

\margintoc

Before we can attempt to build a certified kernel, we need a thorough meta-theoretical study
of the type system. This is necessary in order to show that the invariants used by the kernel – typically, well-typedness of the objects it manipulates – 
re preserved during the type-checking algorithm.
The use of these invariants goes beyond correctness:
the conversion test used as a sub-routine by the \kl{kernel} needs to reduce terms, and,
since all functions in \kl{Coq} must be terminating,
this reduction is defined by well-founded induction on the normalization of well-typed terms.
Since evaluation is not normalizing on ill-typed terms, the mere \emph{definition} of the
conversion check relies on \kl{preservation} to be able to iterate reduction steps.

The properties under scrutinee in this chapter are not new,
and neither are the basic strategy of most proofs.
Indeed, the development roughly follows the architecture we already exposed in
\cref{sec:tech-properties}. The main difficulty is the scale: due to the
complexity of \kl{PCUIC}, even well-understood techniques are challenging to apply.
Moreover, subtleties that do not appear in a simpler setting become apparent –
typically pertaining to universe levels or general inductive types –,
demanding original ideas. Thus, rather than getting lost in
the gory details of the formalization which are best understood by looking at it
– and maybe replaying it –, we try and focus on describing these interesting subtleties.

In more details, we start with the main definitions : the syntax, and conversion and typing
judgments (\cref{sec:meta-defs}). We follow (\cref{sec:meta-stabilities}) by the basic
stability properties: renaming, substitution, environment extension, etc.
Next comes the first important proof, that of confluence, and its multiple consequences
(\cref{sec:meta-confluence}). This leads to the properties pertaining to typing, culminating
with subject reduction (\cref{sec:meta-typing-prop}).
Finally, we discuss the place of normalization.

\section[Terms, Cumulativity and Types]{Setting up the Definitions: Terms, Cumulativity and Types}
\label{sec:meta-defs}

\subsection{Terms}

First thing first: the syntax of terms, defined in \pcuicfile{Ast} and reproduced in \cref{fig:metacoq-ast}.

\begin{figure*}
  \coqfile{./code/PCUICAst.v}
  \caption{The Abstract Syntax Tree of terms in \kl{MetaCoq} (\pcuicline{Ast}{term}{194})}
  \label{fig:metacoq-ast}
\end{figure*}

It of course contains  the term formers introduced in \cref{chap:tech-intro}:
\coqe{tRel} for variables, \coqe{tAbs} for abstractions,
\coqe{tApp} for application, and \coqe{tProd} for dependent function types.
The syntax uses de Bruijn indices for binders – the integer argument of the \coqe{tVar}
term former –, but names are still recorded, mainly for printing annotations, directly in the
binders – this is the \coqe{aname} argument of \coqe{tProp} and \coqe{tAbs}.
There are also local definitions, in the form of the \coqe{tLetIn} constructor,
binding the term \coqe{b} of type \coqe{B} in \coqe{t}.

The \coqe{tSort} constructor is for sorts – what we have called universe earlier.
Its argument \coqe{Universe.t} represents its \kl{universe level}, which can be either $\Prop$
or an algebraic expression based on universe variables, in order to handle \kl{typical ambiguity}.

Next come \coqe{tVar} and \coqe{tEvar}, which
correspond respectively to named variables and existential variables. These are never well-typed in
the current formalization, and ignored in most of the development. Still, they are kept to be as
faithful as possible to the representation of the \kl{Coq} kernel.
Indeed, the inductive \coqe{term} corresponds directly to the
\mintinline{OCaml}{constr} datatype used there.%
\sidenote{The only differences are that the latter uses
an n-ary application rather than a binary one, and casts that inform the \kl{kernel} as to
which conversion algorithm to use, but which is left out since we implement only one.}

Follow the three term formers \coqe{tConst}, \coqe{tInd} and
\coqe{tConstruct} all refer to previous definitions, stored in a \kl{global environment}.
The first corresponds to constants, that either have a body – definitions –
or do not – axioms. The next two are respectively for
inductive types and inductive constructors, respectively. Co-inductive and record types and
constructors are also represented by these term formers, the information contained in the
\coqe{inductive} argument is used to separate between them.
All of these can be \kl{universe polymorphic}, in which case they must be instantiated
with a list of universes – their \coqe{Instance.t} argument.

The two subsequent \coqe{tCase} and \coqe{tProj} are destructors for (co-)inductive types. The
latter is a \kl{projection}, used to destruct \kl{record types}. The former represents the
pattern-matching construction. Its main components are the \kl{predicate} \coqe{p}, the
\kl{scrutinee} \coqe{c} and the \kl{branches} \coqe{brs}.

Finally, the two very similar \coqe{tFix} and \coqe{tCoFix} are for (co-)fixed-points. 
These can be mutual: the \coqe{mfix} argument is a list of definitions,
that can each refer each other.

\subsection{Cumulativity}

The next important definition is that of \kl{cumulativity}, given in \pcuicfile{CumulativitySpec}.%
\sidenote{The “Spec” part comes from the fact that this is the \emph{specification} of conversion,
by contrast to the algorithmic version encountered later on.}
It is stated in the \kl[declarative conversion]{declarative} \kl(conv){untyped} fashion,
akin to how we defined \kl{declarative conversion} in
\cref{fig:cic-uconv-beta,fig:cic-uconv-equiv,fig:cic-uconv-cong}.
This time, however, it is done relatively to
both a \kl{global environment} \coqe{Σ} and a context \coqe{Γ},
as these contain definitions that \kl{cumulativity} can unfold.

\kl{Cumulativity} is defined mutually with \kl{conversion}, because for instance when two Π-types
are compared for \kl{cumulativity}, their codomains are recursively compared for \kl{cumulativity},
but their domains are compared for \kl{conversion} instead.%
\sidenote{As explained in \cref{sec:tech-pcuic},
this is a consequence of the models justifying cumulativity.}
Since the two relations are extremely similar, they are actually fused in a single
inductive relation, \coqe{Σ ;;; Γ ⊢ t ≤s[pb] u}.
This relation is indexed by a \intro{conversion problem} \coqe{pb : conv_pb}, which can take
the two values \coqe{Conv} and \coqe{Cumul}, so that \kl{cumulativity}
is actually \coqe{≤s[Cumul]} – and \kl{conversion} is
\coqe{≤s[Conv]}. This has the advantage that a lot of definitions and proofs
can be factored using \coqe{conv_pb}. Moreover, using a simple boolean allows for case reasoning
when needed, which would not be possible if the index was \eg a relation.%
\sidenote{This used to be the case prior to the uniform introduction of \coqe{conv_pb}: the
relation was the one to be used at leaves to compare universes, which differed between 
\kl{conversion} and \kl{cumulativity}.}

\begin{figure}[h]
  \ContinuedFloat*
  \coqfile[firstline=4,lastline=15]{./code/PCUICCumulativitySpec.v}
  \caption{Pre-order rules (\pcuicline{CumulativitySpec}{cumulSpec0}{29})}
  \label{fig:meta-cumul-struct}
\end{figure}

The first set of rules are the pre-order rules of \cref{fig:meta-cumul-struct}:
transitivity, symmetry and reflexivity. Note that symmetry restricts the \kl{conversion problem},
since only \kl{conversion} should be symmetric. Using this rule twice shows that 
\kl{conversion} is included inside transitivity.
Another important thing to note is that symmetry is restricted: the middle term is required
to have all its variables in the context \coqe{Γ}.%
\sidenote{This is the \coqe{is_open_term} predicate.}
This is key in proving the equivalence between this notion of \kl{cumulativity} and the
algorithmic version that appears later on. Indeed, this equivalence relies on \kl{confluence},
which is only true on well-scoped terms in \kl{PCUIC}. Thus, we need to know that this
\kl{declarative conversion} only goes through well-scoped terms, which is exactly what this
condition ensures.

\begin{figure}[h]
  \ContinuedFloat
  \coqfile[firstline=19,lastline=32]{./code/PCUICCumulativitySpec.v}
  \caption{Cumulativity rules (\pcuicline{CumulativitySpec}{cumulSpec0}{29})}
  \label{fig:meta-cumul-cumul}
\end{figure}

\begin{figure}[ht]
  \ContinuedFloat
  \coqfile[firstline=55,lastline=59]{./code/PCUICCumulativitySpec.v}
  \caption{Example of congruence rule (\pcuicline{CumulativitySpec}{cumulSpec0}{29})}
  \label{fig:meta-cumul-cong}
\end{figure}

Next come the rules of congruence. There are actually two kinds of them. The first are the
“standard” ones – similar to those of \cref{fig:cic-uconv-cong}. An example
is given in \cref{fig:meta-cumul-cong}.
More interesting are the rules of
\cref{fig:meta-cumul-cumul}, which implement “real” cumulativity. Rule \coqe{cumul_Sort} directly
implements subtyping between universes, while rules \coqe{cumul_Ind} and \coqe{cumul_Construct}
implement cumulativity of inductive types \sidecite{Timany2017}. The latter two apply respectively
to \emph{fully applied} inductive types and inductive constructors, that can be considered equal
if their arguments are one-to-one convertible, and their universe levels are correctly related.
This means that \eg the \coqe{nil} constructor of polymorphic lists \emph{always} satisfies that
\coqe{nil@{u} A} is convertible \coqe{nil@{u'} A}, irrespective of the universe levels \coqe{u}
and \coqe{u'}.


Last, but not least, are the rules for computation.

\begin{figure*}[h]
  \ContinuedFloat
  \coqfile[firstline=103,lastline=114]{./code/PCUICCumulativitySpec.v}
  \caption{Computation rules for destructors (\pcuicline{CumulativitySpec}{cumulSpec0}{29})}
  \label{fig:meta-cumul-dest}
\end{figure*}

The three rules of \cref{fig:meta-cumul-dest} are for destructors, \ie applied functions,
pattern-matching on a constructor, and projections. The β rule for functions directly uses
substitution: \coqe|b{0 := a}| denotes the substitution of \coqe{a} for the variable of de Bruijn
index \coqe{0} in \coqe{b}. Similarly, the
\coqe{iota_red} function computes the substitution of the branch \coqe{br} by the arguments
of the constructor \coqe{args}.
Finally, the rule for projections simply selects the right field of the record.

\begin{figure*}[h]
  \ContinuedFloat
  \coqfile[firstline=117,lastline=126]{./code/PCUICCumulativitySpec.v}
  \caption{Computation rules for definitions (\pcuicline{CumulativitySpec}{cumulSpec0}{29})}
  \label{fig:meta-cumul-defs}
\end{figure*}

The next three rules deal with definitions. Rule \coqe{cumul_zeta} directly reduces a
let-binder into a substitution, while \coqe{cumul_rel} and \coqe{cumul_delta} unfold a definition,
respectively one that is present in the local context or in the global environment. In the latter
case, the definition must be instantiated with a universe instance, which is denoted \coqe{@[u]}.

\begin{figure}[h]
  \ContinuedFloat
  \coqfile[firstline=129,lastline=142]{./code/PCUICCumulativitySpec.v}
  \caption{Computation rules for fixed-points (\pcuicline{CumulativitySpec}{cumulSpec0}{29})}
  \label{fig:meta-cumul-fix}
\end{figure}

The last rules – \cref{fig:meta-cumul-fix} – pertain to the reduction of (co-)fixed-points.
In all cases, they are unfolded in a guarded fashion, in order to avoid a non-terminating behaviour.
On fixed-points, this guard is that they have to be applied to a constructor,
and dually co-fixed-points are unfolded when
they are forced, either by a pattern-matching or a projection. In both cases, this ensures that
the unfolded (co-)fixed-point can reduce further, either by consuming the constructor of
its recursive argument, or by producing a constructor to be consumed by the destructor
that forced the unfolding.

\subsection{Typing}

With the \kl{cumulativity} relation defined, we can turn to typing, defined in \pcuicfile{Typing}.
As for \kl{cumulativity}, typing is an inductively defined relation \coqe{Σ ;;; Γ |- t : T},
relative to a global environment \coqe{Σ} and a local context \coqe{Γ}.
The rules correspond roughly to ones we already went over in \cref{chap:tech-intro}.

\begin{figure}
  \ContinuedFloat*
  \coqfile[firstline=2,lastline=22]{./code/PCUICTyping.v}
  \caption{\kl{CCω} fragment (\pcuicline{Typing}{typing}{188})}
  \label{fig:meta-typing-ccw}
\end{figure}

The first set of typing rules pertain to the \kl{CCω} fragment. There are not many differences
there with respect to \cref{fig:ccw-typing}. Rule \coqe{type_Rel} looks up the type of the
variable in the context, and ensures that said context is well-formed:
the \coqe{wf_local} predicate corresponds to $\vdash \Gamma$, but here as everything else it is
relative to a global environment. Rule \coqe{type_Sort} uses the \coqe{super} function to compute
the successor of an algebraic universe, and similarly Rule \coqe{type_Prod} uses the
\coqe{sort_of product} to compute the universe level of a Π-type%
\sidenote{This not only includes computing the maximum of two algebraic universes expressions,
but also handling the \kl{impredicativity} of the $\Prop$ sort.}
Context extension with an assumption, \ie a variable without a body,
is written \coqe{Γ ,, vass na A}, and used as expected in Rules \coqe{type_Prod}
and \coqe{type_Lambda}. Finally, \coqe{type_App} is for application. It contains a “paranoid”
assumption that the product is well-formed, which is not strictly speaking needed once we
prove \kl{validity}, but is useful in some cases to provide a needed induction hypothesis.

\begin{figure}
  \ContinuedFloat
  \coqfile[firstline=23,lastline=27]{./code/PCUICTyping.v}
  \caption{Local definitions (\pcuicline{Typing}{typing}{188})}
  \label{fig:meta-typing-letin}
\end{figure}

The rule for local definitions, given in \cref{fig:meta-typing-letin}, is similar to the one
for λ-abstractions, with the only difference that the body too is typed, and that the context
is extended with a definition, \ie a variable with a body.

\begin{figure*}[h!]
  \ContinuedFloat
  \coqfile[firstline=28,lastline=42]{./code/PCUICTyping.v}
  \caption{Globally defined terms (\pcuicline{Typing}{typing}{188})}
  \label{fig:meta-typing-genv}
\end{figure*}

Next (\cref{fig:meta-typing-genv}) are the three rules performing look-ups in
the \kl{global environment}, respectively constants – \coqe{type_Const} –, inductive types –
\coqe{type_Ind} – and inductive constructors – \coqe{type_Construct}. In all cases the
term should be declared in the \kl{global environment}, the context well-formed
– since these are leaves of a term –,
and the universe instance given should respect the constraints coming from the
entry in the \kl{environment} – this is the \coqe{consistent_instance_ext} predicate.

\begin{figure*}[h]
  \ContinuedFloat
  \coqfile[firstline=43,lastline=58]{./code/PCUICTyping.v}
  \caption{(Co-)inductive destructors (\pcuicline{Typing}{typing}{188})}
  \label{fig:meta-typing-idest}
\end{figure*}

The rules of \cref{fig:meta-typing-idest} are the ones for the destructors of (co-)inductive
types. Rule \coqe{type_Case} is somewhat similar to \eg \ruleref{rule:bool-ind}: the \kl{scrutinee}
should be of an inductive type, and the \kl{predicate} and \kl{branches} should all be well-typed
in the appropriate context. The main complexities are to build these contexts, and
\coqe{case_side_conditions}, which handles universe instances, checks
that the elimination is allowed%
\sidenote{This is where \kl{PCUIC} enforces \kl{proof irrelevance}, by imposing the so-called
“singleton elimination” criterion, which ensures that only inductive
types of a certain specific shape – sub-singletons – can be matched on to build proof relevant
content, so that that content cannot actually depend on the value of a proof.}…
The second rule, \coqe{type_Proj} is somewhat similar, albeit a bit simpler: the \kl{scrutinee}
should still be some applied co-inductive/record type, and the type is constructed by
substitution from the projection information.

\begin{figure}[h]
  \ContinuedFloat
  \coqfile[firstline=59,lastline=74]{./code/PCUICTyping.v}
  \caption{(Co-)fixed-points (\pcuicline{Typing}{typing}{188})}
  \label{fig:meta-typing-fix}
\end{figure}

The last typing rules for terms are those for (co-)fixed-points,
given in \cref{fig:meta-typing-fix}. They are almost identical,
the main part amounts to checking that the types and bodies of all
the mutually definitions are well-typed.
The \coqe{wf_fixpoint} and \coqe{wf_cofixpoint} both check that the mutual definitions are on
the same block of mutually-defined (co-)inductive types, and that these are of the right kind –
inductives for \coqe{tFix} and co-inductives for \coqe{tCoFix}.
Finally, the \coqe{fix_guard} and \coqe{cofix_guard} predicates correspond to the guard condition,
ensuring that the definitions do not endanger \kl{normalization}. We come back to those in
\cref{sec:meta-normalization}.

\begin{figure}[h]
  \ContinuedFloat
  \coqfile[firstline=75,lastline=79]{./code/PCUICTyping.v}
  \caption{Cumulativity (\pcuicline{Typing}{typing}{188})}
  \label{fig:meta-typing-cumul}
\end{figure}

The final rule is that which uses \kl{cumulativity} as just defined
to change the type of the term, \eg \ruleref{rule:cic-cum}.

From the definition of \pcuicline{Typing}{typing}{188}, two more pervasively used definitions
follow. We have already encountered \pcuicline{PCUICTyping}{wf\_local}{284},
asserting that a local context is well-formed.
Its sibling predicate \pcuicline{PCUICTyping}{wf}{439}%
\sidenote{Sometimes replaced by \pcuicline{PCUICTyping}{wf\_ext}{444},
an extension that takes into account the universes of the current definitions.}
asserts that the \kl{global environment} is well-formed. It ensures not only that all definitions
are properly typed, but also of the validity of various information
related to inductive types – in particular the positivity criterion which ensures that inductive
definitions are well-founded – and universe polymorphism.

\section[Stabilities]{The Easy Properties: Stabilities}
\label{sec:meta-stabilities}

With the main definitions set up, we can turn to the properties that we collectively called
stabilities in \cref{sec:tech-properties}. These assert that \kl{cumulativity} and \kl{typing}
as just defined are stable by various ubuquitous operations: extension of the local context
(\pcuicfile[Conversion]{WeakeningConv}, \pcuicfile[Typing]{WeakeningTyp}) and global
environment (\pcuicfile[Conversion]{WeakeningEnvConv}, \pcuicfile[Typing]{WeakeningEnvTyp}),
and substitution, not only for terms (\pcuicfile[Conversion]{InstConv}, \pcuicfile{Substitution}),
but also for universe variables
(\pcuicfile[Conversion]{UnivSubstitutionConv}, \pcuicfile[Typing]{UnivSubstitutionTyp}).

A last property falling in the section of low-hanging fruits as well
is the fact that well-typedness implies
\intro{well-scopedness} (\pcuicfile{ClosedTyp}), that is that if \coqe{Σ ;;; Γ |- t : T}
then all free variables of \coqe{t} appear in \coqe{Γ}. This \kl{well-scopedness} appears
in the transitivity rule for \kl{cumulativity} (\cref{fig:meta-cumul-struct}) and is a
condition for many lemmas.

All of these are proven by induction on the typing derivation.
However, while the proof by themselves are not very surprising, the formalization of the definitions
is not sot easy, and deserves a few comments.

The first point to note is that while the weakening and substitution operations are defined
directly by induction on the syntax of terms, the proofs are not done directly
on those definitions. Rather, \kl{MetaCoq} uses definitions of renaming and instantiation coming
from the σ-calculus \sidecite{Abadi1991,Schaefer2015}, as functions of integers – to integers for
renamings, and to terms for instantiations. \kl{Weakening} and \kl{substitution} are then connected
respectively to a specific form of renamings and of instantiations, and the stability follows from
more general forms, for instance weakening is but a consequence of general stability by
(unconditional) renaming. This approach makes it easier to handle the complex binding structures
present in the syntax of \kl{PCUIC}, which require parallel substitution, lifting by a whole context
at once…

More interestingly, the same approach is taken also for well-scopedness: a general way of lifting
a boolean function on integers to a function boolean function
on terms\sidenote{This is \pcuicline[Syntax]{OnFreeVars}{on\_free\_vars}{71}.}
is used as a generalization of the well-scopedness predicate.
This makes it easy to relate the σ-calculus
to well-scopedness assumptions, \eg to show that if two substitutions \coqe{σ} and \coqe{σ'}
agree on the free variables of a term \coqe{t}, then the application of \coqe{σ} and \coqe{σ'}
to \coqe{t} are equal.

\begin{figure}[h]
  \coqfile{./code/PCUICSubstitution.v}
  \caption{Well-formed substitution (\pcuicline{Substitution}{subslet}{44})}
  \label{fig:meta-subslet}
\end{figure}

A last point of interest is the definition of a well-formed substitution, a predicate called
\pcuicline{Substitution}{subslet}{44}. Indeed the usual typing judgment for substitutions
is of the form $\Delta \vdash \sigma \ty \Gamma$, meaning that
$\sigma$ maps each assumption $(x : A) \in \Gamma$ to a term $t$ such that
$\Delta \vdash t \ty \multisubs{T}{\sigma}$. But in our setting we must account for variables
that can be \emph{defined} in $\Delta$. This leads to the definition of a well-formed substitution as in \cref{fig:meta-subslet}.
A similar definition, called \pcuicline[Syntax]{InstDef}{well\_subst}{65}, is also available for
instantiations.

\section[Confluence]{Things Get Serious: Confluence}
\label{sec:pcuic-confluence}


\section[Properties of Typing]{Reaping the Fruits: Properties of Typing}
\label{sec:meta-typing-prop}

\section[Normalization]{Gödel’s Thorn in the Side: Normalization}
\label{sec:meta-normalization}