\bookmarksetup{startatroot}
\chapter{Perspectives}
\label{chap:future-work}

I hope that this thesis gives compelling arguments for the adoption of bidirectional typing,
but there is more.

\arefpart{bidir} shows that one can use the valuable bidirectional structure,
without having to  leave their favourite declarative
system behind. Indeed, it can most likely  be shown equivalent to a bidirectional one
– given it is one’s favourite system, it does surely satisfy the good properties
needed for that.

In \arefpart{metacoq}, we enter the real world, and see how this plays out
on a complex type system: bidirectionalism gives good guidelines to analyse
typing rules, and provides a precise specification to prove the implementation correct,
while allowing separation of concerns. If this is enough
to catch bugs in \kl{Coq}, it should prove useful in finding those in other kernels, too.
Yet, \kl{MetaCoq} is much more than bidirectional typing: its certified kernel opens
up a new era for proof assistants, with a previously unreached trust level.

Finally,
\arefpart{gradual} exemplifies how the bidirectional structure can be useful when
simply designing a type system, even without a single implementation in sight.
But gradual typing can hopefully be more than a mere
example. As it enables the transition of programmers from the soft realms of
dynamic typing to the discipline of static typing, so it could open the door of
dependently typed programming to more than a fraction of fanatic enthusiasts.

Still, as most thesis, this one opens up at least as many questions as it answers,
in all its three broad directions.

\section{Bidirectional Typing for Dependent Types}

The formal study of bidirectional typing in the setting of dependent types still begs for
more investigations. While I hope the present work gives a robust answer in the setting
of Curry-style syntax, where every term infers a type,
the case of Church-style syntax is quite different. In the case of normal forms,
the proof ideas presented in this thesis should be easily adapted.
But if we wish to go beyond normal forms, we must consider the use of annotations in terms,
as is done in \eg \sidetextcite{McBride2022} \sidetextcite{Gratzer2019} or
\sidetextcite{Dunfield2021}.
However, due to the dependently-typed
setting, we have to investigate how these annotations play out with conversion and/or reduction.
To the best of my knowledge, only \citeauthor{McBride2022} has taken that question up,
but does not arrive – yet – at a definitive solution, so there is
matter left for further research.

Another thread to pull is the relation with Generalized Type Systems, as presented in
\sidetextcite{Bauer2020,Bauer2022}. Here as in McBride’s discipline we find
well-formation invariants to be preserved,
and carefully structured rules that should respect them. Re-casting
the bidirectional ideas in such a setting could allow for a better understanding both of the
ideas at work in bidirectional typing,
yielding a proper formal account of McBride’s discipline
together with a general proof that it ensures
good properties of the system, and of the well-formation conditions already explored in
\textcite{Bauer2020} on judgment boundaries.

Since Generalized Type Systems put conversion and typing on the same footing,
it also seems natural
to question how we can marry conversion and bidirectionalism. Here again there are ingredients in
the air: \sidetextcite{Abel2017} show a notion of conversion geared towards proving decidability
of typing, but which is clearly bidirectional, and could serve as a basis to give a general
notion of bidirectional conversion.

\section{\kl{MetaCoq}’s Future}

\kl{MetaCoq} is a mature project, and has reached the stage
where the formalization can really
serve as a tool to move \kl{Coq} forward.

We have already evoked in \cref{chap:bidir-pcuic} the question of the representation of pattern-matching.
This is a relatively minor question, but more complex ones – \eg 
the integration of a sort $\SProp$ of strict propositions, or subject reduction for
co-inductive types –
can now be investigated in \kl{MetaCoq}, providing a valuable guidance 
to their implementation in \kl{Coq}.

However, \kl{MetaCoq} is still quite some distance away from type-checking realistic developments
in \kl{Coq}, as it lacks some important features present in the latter’s kernel. Barring
template polymorphism,%
\sidenote{A restricted form of universe polymorphism, which the latter should hopefully
be able to replace.}
there are two main lacking elements that are to be integrated if we wish to really reach the
project’s goal.

The first are extensionality conversion rules: definitional proof irrelevance,
and η laws for functions and records.
The η conversion laws are basic features, present in virtually
any modern proof assistant. However,
in the precise context of \kl{MetaCoq}, they pose subtle questions.%
\sidenote{For an overview of these, see \textcite{LennonBertrand2022a}.}%
\margincite{LennonBertrand2022a}
Broadly, giving a specification of such η laws is easy in the setting of typed conversion, but
much trickier in that of untyped conversion. However, the whole structure of \kl{MetaCoq} is built
around that untyped notion of conversion, and could not be so easily adapted to a typed
conversion. This makes the integration of η laws challenging. The case of strict propositions is
less well-known, being much more recent, but poses similar challenges.

The second lacking feature are modules and functors.
While these are less pervasive than η laws, they are still important in a number of developments.
Here again the difficulty is not simply to show that an implementation is faithful to a given
semantic, but to precisely pin down said semantic. This is tricky in the case of modules, which
have interactions with global environments, contrarily to records – their first-class counterpart.
This unclear semantic is probably one of the reason modules are not used more, and so putting
them on a stabler ground might also give users more confidence to use them.

A last important investigation to make \kl{MetaCoq} closer to the real kernel is that of
guard conditions. The impossibility to prove full \kl{normalization} of \kl{PCUIC}
does not mean that we should not completely abandon this question. We
can at least implement a guard condition,
and show that it fulfils the conditions we abstractly ask for
in the current development. More ambitious, the complex guard condition implemented in \kl{Coq}
was designed \sidecite{Gimenez1995} in order to allow a translation back to eliminators.
This gives a much stronger validity criterion for the guard, and would not be a simple project.
But as for modules, reaching
that goal would make the guard much more trustworthy than it currently is. Moreover,
it could open the door to extending it, with the formalization
as a safeguard as to the validity of those extensions.

Beyond these missing but rather necessary pieces,
\kl{MetaCoq} should hopefully offer tools for
broader investigations around \kl{Coq}’s core: formalization of tactics,
of syntax transformation and generation… Some have already started to appear
\sidecite{Forster2019,Liesnikov2020}, but hopefully more are to come!

\section{Gradual CIC}

As for the last part of this thesis, if the aim goal of gradual typing
is to answer the needs of developers, we should get closer to those.
I believe that \kl{GRIP} gives at least a good starting point to experiment with,
so the main missing piece now is an implementation.
Such an implementation of course is no small feat: integrating a new feature to \kl{CIC} is
never easy, even more so one of this scale.
Moreover, it raises subtle questions. For instance, while almost all reduction rules of
dependent type theories are parametric over the universe levels,
reduction in \kl{CastCIC} crucially depends on those.
In a setting where these universe levels are not mere integers,
what becomes of those? How do we handle a non-total order between universe levels?