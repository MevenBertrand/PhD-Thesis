% Load the kaobook class
\documentclass[
  french,english,
	fontsize=10pt, % Base font size
	twoside=true, % Use different layouts for even and odd pages (in particular, if twoside=true, the margin column will be always on the outside)
	open=any, % If twoside=true, uncomment this to force new chapters to start on any page, not only on right (odd) pages
	secnumdepth=2, % How deep to number headings. Defaults to 1 (sections)
  numbers=enddot,
]{kaobook/kaobook}

% Choose the language
\usepackage{babel} % Load characters and hyphenation
\usepackage[english=british]{csquotes}	% English quotes

% Load packages for testing
\usepackage{blindtext}
%\usepackage{showframe} % Uncomment to show boxes around the text area, margin, header and footer
%\usepackage{showlabels} % Uncomment to output the content of \label commands to the document where they are used

% Load the bibliography package
\usepackage[style=alphabetic,maxbibnames=99]{kaobook/kaobiblio}
\DefineBibliographyExtras{french}{\restorecommand\mkbibnamefamily} %To have consistent citations between French and English text
\addbibresource{biblio.bib} % Bibliography file

%Load my own style
\usepackage{styles/layout}

% Load mathematical packages for theorems and related environments
\usepackage[boxed]{kaobook/kaotheorems}

% Load the package for hyperreferences
\usepackage{kaobook/kaorefs}

% Macros are after the knowledge package
% \knowledgestyle*{notion}{style={color=black!80!},intro style={color=black!80!,emphasize}}

\knowledgestyle{notion}{color=black!70!}
\knowledgestyle{notion intro}{color=black!70!,emphasize}
\knowledgedirective*{notion}{autoref,style=notion,intro style=notion intro}

\knowledgestyle{ignore intro}{emphasize}
\knowledgedirective*{ignore}{intro style=ignore intro}


%Chapter 1

\knowledge{notion}
  | type dépendant
  | types dépendants
  | théorie des types dépendants

\knowledge{notion}
  | assistant à la preuve
  | assistants à la preuve

\knowledge{notion}
  | Coq

\knowledge{notion}
  | correspondance de Curry-Howard

\knowledge{notion}
  | Gallina

\knowledge{notion}
  | noyau

\knowledge{notion}
  | critère de de Bruijn

\knowledge{notion}
  | MetaCoq

\knowledge{notion}
  | graduel
  | graduels
  | typage graduel
  | type graduel
  | types graduels

% Chapter 2

\knowledge{notion}
  | dependent type

\knowledge{notion}
  | Curry-Howard correspondence

% Chapter 3

\knowledge{notion}
  | Calculus of Constructions
  | CCω

\knowledge{notion}
  | Calculus of Inductive Constructions
  | CIC

\knowledge{notion}
  | Polymorphic, Cumulative Calculus of Inductive Constructions
  | PCUIC

\knowledge{notion}
  | α-equality
  | α-equal 

\knowledge{notion}
  | Church-style

\knowledge{notion}
  | Curry-style

\knowledge{ignore}
  | universe level
  | universe levels
  | level
  | levels

\knowledge{notion}
  | typical ambiguity

\knowledge{notion}
  | conversion

\knowledge{notion}
  | typed conversion
  | Typed conversion
  | typed@conv

\knowledge{notion}
  | untyped conversion
  | Untyped conversion
  | untyped@conv

\knowledge{notion}
  | Pure Type Systems
  | PTS

\knowledge{notion}
  | declarative conversion
  | declarative@conv
  | conversion@decl

\knowledge{notion}
  | algorithmic conversion
  | algorithmic@conv

\knowledge{notion}
  | reduction

\knowledge{notion}
  | top-level reduction
  | top-level@red

\knowledge{notion}
  | one-step reduction
  | one-step@red

\knowledge{notion}
  | weak-head reduction
  | weak-head@red

\knowledge{notion}
  | full reduction
  | full@red
\usepackage{styles/macros}
\usepackage{styles/alectryon-minted}

% \includeonly{header,intro-fr,toc,intro-en,technical-intro}

\graphicspath{{./figures/}} % Paths where images are looked for

%\makeindex[columns=3, title=Alphabetical Index, intoc] % Make LaTeX produce the files required to compile the index

\begin{document}

%----------------------------------------------------------------------------------------
%	BOOK INFORMATION
%----------------------------------------------------------------------------------------

% \titlehead{Document Template}
\title[Bidirectional Typing in the Calculus of Inductive Constructions]{Bidirectional Typing in the Calculus of Inductive Constructions}
\subtitle{PhD Thesis}
\author[M. Lennon-Bertrand]{Meven Lennon-Bertrand}
\date{\today}
% \publishers{An Awesome Publisher}

%----------------------------------------------------------------------------------------

\frontmatter % Denotes the start of the pre-document content, uses roman numerals

%----------------------------------------------------------------------------------------
%	COPYRIGHT PAGE
%----------------------------------------------------------------------------------------

\makeatletter
\uppertitleback{\@titlehead} % Header

\lowertitleback{
	% \textbf{Disclaimer} \\
	% You can edit this page to suit your needs. For instance, here we have a no copyright statement, a colophon and some other information. This page is based on the corresponding page of Ken Arroyo Ohori's thesis, with minimal changes.
	
	% \medskip
	
	% \textbf{No copyright} \\
	% \cczero\ This book is released into the public domain using the CC0 code. To the extent possible under law, I waive all copyright and related or neighbouring rights to this work.
	
	% To view a copy of the CC0 code, visit: \\\url{http://creativecommons.org/publicdomain/zero/1.0/}
	
	% \medskip
	
	% \textbf{Colophon} \\
	% This document was typeset with the help of \href{https://sourceforge.net/projects/koma-script/}{\KOMAScript} and \href{https://www.latex-project.org/}{\LaTeX} using the \href{https://github.com/fmarotta/kaobook/}{kaobook} class.
	
	% \medskip
	
	% \textbf{Publisher} \\
	% First printed in May 2019 by \@publishers
}
\makeatother

%----------------------------------------------------------------------------------------
%	DEDICATION
%----------------------------------------------------------------------------------------

\dedication{
	% The harmony of the world is made manifest in Form and Number, and the heart and soul and all the poetry of Natural Philosophy are embodied in the concept of mathematical beauty.\\
	% \flushright -- D'Arcy Wentworth Thompson
}

%----------------------------------------------------------------------------------------
%	OUTPUT TITLE PAGE AND PREVIOUS
%----------------------------------------------------------------------------------------

% Note that \maketitle outputs the pages before here
\maketitle

%----------------------------------------------------------------------------------------
%	PREFACE
%----------------------------------------------------------------------------------------

\addchap{Abstract}

“\kl{Coq} is an old man now, and it has a lot of scars”
\cite[citing Assia Mahboubi]{QuantaPA}.
Proof assistants have been around for 50 years now, and they have become a more and
more established technology over time. This history is both a blessing and a curse: as
the field matured, the tools have become more and more complex, making them more and more
powerful, but also more and more prone to critical bugs hiding in dark corners. At a time
when they are gaining traction in an increasing number of communities,
especially those concerned with high safety and security guarantees, this simply cannot be.
The historical solution of keeping a small, trusted \kl{kernel}
– the so-called de Bruijn criterion –
is  simply not enough if we wish to keep moving on and integrate more new complex features
to keep up with the needs of users.

There is a straightforward solution to this:
proof assistants have been used for decades to prove
programs correct. Why could they not prove \emph{themselves} correct? After all, if this is
the golden standard we demand for software, it should apply first and foremost to the ones
used to justify that trust. For the proof assistant \kl{Coq},
this is the ambition of the \kl{MetaCoq} project,
which aims at providing a drop-in replacement for \kl{Coq}’s \kl{kernel}, but one that has been
fully proven correct, even though it handles all the subtleties and quirks of said \kl{kernel}.
No more trusting a complex and ever-evolving implementation, trust the formally validated
\emph{proofs} instead!

But before we can hope to achieve that goal, we need more study of the structures at work
in the \kl{kernel}. In particular, its typing algorithm is \emph{bidirectional}, meaning that
it constantly alternates between the two problems of type \emph{inference} – finding a type for
a term – and type \emph{checking} – verifying that a type is adequate for a term. While this
structure is crucial in relating the specification of the type system to its implementation,
it has been rather little studied in the context of the \kl{Calculus of Inductive Constructions}
(\kl{CIC}), the theoretical foundation of \kl{Coq} – but also of the closely related
\kl{Agda}, \kl{Lean}…

This thesis aims at filling that gap, by providing a thorough study of this bidirectional
structure, formalized in the setting of the \kl{MetaCoq} project. This is a key
ingredient in the first formal proof of correctness and completeness of a type-checking
algorithm for (a significant subset of) \kl{Coq}, which was able to catch bugs that had gone
unnoticed until then.
But it is also an interesting theoretical tool in its own right, giving a form of control
over computation that turns out to be crucial in quite a few situations.

In particular, bidirectional typing is a necessary piece in the design of a gradual extension of
\kl{CIC}, \kl{GCIC}.
\kl{Gradual typing} aims at giving programmers in one and the same system the flexibility of
development offered by dynamic typing, and the strong guarantees offered
by static typing. \kl{GCIC} intends
to bring that flexibility to dependently-typed programming,
and, by using the power of the \kl{Curry-Howard correspondence}, to proof writing.
But this flexibility comes with subtle difficulties, that can only be solved in a bidirectional
setting.

The first part of the thesis gives a theoretical account of bidirectional typing; the second
describes the formalization of the ideas of the first section in the setting of \kl{MetaCoq};
the third and last introduces \kl{GCIC} and its properties.


\addchap{How to read this thesis}

This thesis has been written primarily for screen reading, as I think this will be the medium
used by most of my viewers. Therefore, I use hyperlinking as much as possible
inside the document.

While it is not visible – in order to keep the text readable by not being too distracting –,
most technical keywords are actually linked to the
place of their definitions. For instance \kl{bidirectional typing} links to
the place in \cref{chap:intro-en} where the notion is introduced.
This definition itself is put into emphasis like the following \intro{example}.
I might cheat a bit and introduce a notion twice, once on a high level in an introductory
section, and a second time precisely later on, in which case the link point to the precise
definition. Most notations are also linked: no more wondering what the symbol $\obsRef$
means again, just click on it!

The main text has large margins, which I use and abuse 
for notes, small figures and references. Hopefully this reduces
back and forth between the text and some stuff much too far away.
Regarding figures, rather than having large, bulky ones that take a full page,
I tried to keep them as close as possible to their explanation. This often means that
they are split in multiple small fragments, so that each part of the figure goes with its
explanation. In such cases, the fragments should really be understood
as different parts of one and the same figure. To indicate this, the fragments share the same
figure number, such as
\cref{fig:cic-var,fig:cic-nondep-fun,fig:cic-dep-fun,fig:cic-univ,fig:cic-prod,fig:cic-con,fig:cic-conv} which all define the same system, one rule at a time.



%----------------------------------------------------------------------------------------
%	TABLE OF CONTENTS & LIST OF FIGURES/TABLES
%----------------------------------------------------------------------------------------
\knowledgeconfigure{protect links}

\begingroup % Local scope for the following commands

\hypersetup{allcolors=.}

% Define the style for the TOC, LOF, and LOT
%\setstretch{1} % Uncomment to modify line spacing in the ToC
%\hypersetup{linkcolor=blue} % Uncomment to set the colour of links in the ToC
\setlength{\textheight}{230\vscale} % Manually adjust the height of the ToC pages

% Turn on compatibility mode for the etoc package
\etocstandarddisplaystyle % "toc display" as if etoc was not loaded
\etocstandardlines % "toc lines as if etoc was not loaded
\setcounter{tocdepth}{\sectiontocdepth} % Locally for the global toc

\tableofcontents % Output the table of contents

\setcounter{tocdepth}{\subsectiontocdepth} % To have correct tocs in the sections

% \listoffigures % Output the list of figures

% Comment both of the following lines to have the LOF and the LOT on different pages
% \let\cleardoublepage\bigskip
% \let\clearpage\bigskip

% \listoftables % Output the list of tables

\endgroup
\knowledgeconfigure{unprotect links}


%----------------------------------------------------------------------------------------
%	MAIN BODY
%----------------------------------------------------------------------------------------

\mainmatter % Denotes the start of the main document content, resets page numbering and uses arabic numbers

\selectlanguage{french}

\chapter{Introduction (Français)}
\label{ch:intro-fr}

\margintoc

\selectlanguage{english}
\begin{kaobox}[backgroundcolor=Black!10!White,frametitlebackgroundcolor=Black!10!White]
  This section is an introduction intended for French-speaking readers.
  If your English is better than your French,
  you should instead read \cref{ch:intro-en}, its translation in English.
\end{kaobox}
\selectlanguage{french}

\todo{Références manquantes}

Cette thèse se situe dans le domaine de la \kl{théorie des
types (dépendants)}, lui-même au croisement entre informatique et logique mathématique.
L’objectif principal est de participer aux fondements théoriques et pratiques des outils
que l’on appelle \kl{assistants à la preuve}, des logiciels qui, comme leur nom
l’indique, ont pour but d’assister des êtres humains dans la construction
et la vérification de preuves – au sens mathématique du terme. Il sera dans cette thèse
en particulier beaucoup question de l’assistant à la preuve \kl{Coq}, qui est celui
sur lequel mon travail s’est principalement concentré.

Pour replacer ce travail dans son contexte large, je propose dans cette introduction une histoire très parcellaire et orientée de la logique mathématique
(\cref{sec:logique-histoire}), puis une courte présentation des assistants à la preuve,
notamment ceux qui, comme \kl{Coq}, se basent sur la théorie des types (\cref{sec:assistants-preuve}). Enfin je finis par une présentation de mes contributions
personnelles pendant la durée de cette thèse (\cref{sec:cette-these}).
\todo{Ajouter disclaimer ?}

\section{Une (très courte) histoire de la logique}
\label{sec:logique-histoire}

\subsection{Les syllogismes}
Dans la tradition occidentale, on peut faire remonter l’étude de
la logique à Aristote, dans son \cite{Organon}.
L’un des apports de ce travail est d’introduire les syllogismes.\sidenote{
  Le syllogisme le plus connu est probablement Barbara, dont un exemple est :
  \emph{tous les humains sont mortels ; Socrate est humain ; donc Socrate est mortel.}
}
Il s’agit de raisonnements dont la validité tient seulement au fait qu’ils
suivent une structure générale, et non à son contenu particulier.
Si un raisonnement est construit en assemblant ces syllogismes,
le raisonnement dans son entier doit nécessairement l’être également, puisque
chaque pas de raisonnement est valide.
L’idée importante ici est celle de décomposition en composantes élémentaires. À
partir d’un système de règles de raisonnement qu’on a identifiées comme valides 
\textit{a priori},\sidenote{
  Il peut s’agir de syllogismes, mais de bien d’autres systèmes… On en rencontra
  un certain nombre dans cette thèse !
}
on a un moyen de s’assurer de la validité de raisonnements potentiellement
très complexes.
Il suffit de vérifier qu’ils peuvent être décomposés à partir
des règles de base – et, bien entendu, que celles-ci soient correctes.

\subsection{Les débuts de la logique mathématique}
À la suite d’Aristote, les mathématicien·ne·s se sont emparé·e·s de la question
de la logique, en cherchant comment il était possible de fonder les mathématiques
rigoureusement. Bien qu’il s’agisse d’une question ancienne, de véritables
progrès concrets sur sa résolution ont commencé à voir le jour dans la deuxième
moitié du 19\textsuperscript{e} siècle, sur deux fronts principaux.

Le premier a consisté à se dégager du langage dit
naturel\sidenote{
  Par opposition aux langages formels qui apparaissent
  en mathématiques, informatique, etc.
}, inadapté pour la description formellement précise de la déduction, et à
concevoir à la place une nouvelle forme de langage spécifique qui puisse servir de
base à un système de raisonnement. Une étape importante
de cette ligne de recherche est
probablement \sidecite{Begriffsschrift}, qui introduit un certain nombre de
caractéristiques des langages dont il sera question dans la suite de cette thèse,
en particulier la notion de quantificateur.

Le second a pour but de montrer que les mathématiques dans leur entier peuvent
effectivement être reconstruites à partir de briques élémentaires. Une étape
importante ici a été la réduction de l’analyse à un petit nombre de propriétés
des nombres réels, puis les constructions de ces nombres réels à partir
de l’arithmétique, données simultanément par plusieurs auteurs\sidenote{
  Cantor, Méray, Dedekind, Bertrand, Weierstraß \cite{??}
} autour de 1870. De son côté Peano \cite{PeanoAxioms} propose
l’axiomatisation des nombres entiers qui est nommée en son honneur.
Enfin Cantor \cite{??}
propose la théorie des ensembles comme un formalisme permettant
de décrire la totalité des objets mathématiques sous la forme d’ensemble
d’éléments.

\subsection{La crise des fondements}
Hélas, le système proposé dans \cite{Begriffsschrift}, fortement inspiré par
les travaux de Cantor, est incohérent\sidenote{
  C'est-à-dire qu’il permet de prouver le faux, et donc qu’il ne peut pas servir
  de base valide pour la logique.
} !
Ce constat, dû à Russell \cite{Begriffsschrift}\todo{Citer le bon appendice},
ouvre une période de crise, où la problématique de décrire un système qui permette
de fonder l’entièreté des mathématiques,
tout en évitant les inconsistances desquelles
le système de Frege et probablement ceux de Cantor étaient victimes.

Une première proposition de solution est avancée par Whitehead et Russell
avec \cite{Principia}, un énorme travail qui non seulement propose un système
logique qui évite les paradoxes conduisant à l’incohérence de
\cite{Begriffsschrift}, mais de plus réalise dans ce système une quantité importante
de mathématiques, en particulier une construction des entiers, de l’arithmétique et
finalement des nombres réels.

En parallèle, dans la continuité des travaux de Cantor, Zermelo \cite{Zermelo1905} et
d’autres travaillent à fournir une description formelle de la théorie des ensemble
qui soit elle aussi cohérente. Ceci aboutit à ce qui est appelé de nos jours la
théorie des ensembles de Zermelo-Fraenkel (ZF, ou ZFC quand on y ajoute l’axiome
du choix), qui semble également à même de fournir une base solide pour les
mathématiques.

\paragraph{L’incomplétude.}
Un deuxième coup vient cependant frapper la recherche d’un système adéquat pour
servir de fondation aux mathématiques : le théorème d’incomplétude de Gödel
\cite{incomplétude}. Celui-ci affirme que tout système formel dans lequel on peut
construire des nombres entiers vérifiant les axiomes de Peano – donc
\textit{a fortiori} tout système suffisamment riche pour faire des mathématiques –
ne peut pas démontrer sa propre cohérence. Ainsi, il n’existe pas de
système sur lequel on puisse fonder les mathématiques en ayant la certitude
formelle que ce système est adéquat : puisqu’on ne peut prouver la cohérence du
système dans lui-même, il pourrait finalement s’avérer incohérent, ruinant les
efforts fournis.

Une conséquence de ce théorème est qu’un système suffisamment riche
pour faire des mathématiques est nécessairement incomplet.\sidenote{
  Cela signifie qu’il
  existe des énoncés indépendants, à savoir des assertions qu’on ne peut
  ni démontrer, ni réfuter – c’est-à-dire montrer la négation. La cohérence du
  système considéré en est un exemple.
}
Ainsi, dans la suite je ne parlerai jamais de vérité absolue –
ce qui n’aurait de sens que dans un système complet
où tout énoncé est vrai ou faux –, mais
uniquement de prouvabilité \emph{dans un système donné}.

\subsection{Une première conclusion}
Malgré les difficultés mises au jour au début du 20\textsuperscript{e}
siècle, la communauté mathématique est globalement
satisfaite de la situation : ZFC fournit un système raisonnable sur
lequel baser en principe toutes les mathématiques, et même si peu de
personnes se risquent à tenter, dans la veine de \cite{Principia},
d’effectivement écrire leurs mathématiques à ce niveau de précision,
elle est globalement convaincue que ce serait en théorie
possible, et cela suffit amplement à la plupart.

De plus, le développement et la vérification humaine de mathématiques véritablement
formalisées semble essentiellement impossible et inutile.
D’un côté, cela demanderait un effort considérable,
tant de la part de l’autrice que de celle de la lectrice, tout en étant
extrêmement laborieux et désagréable.
Dans le même temps, cela ne permettrait pas de réduire de manière importante
les risques d’erreurs, puisqu’il est
humainement très difficile de détecter une petite erreur au milieu de centaines de pages de raisonnement.
Enfin, décrire les mathématiques à ce niveau obscurcirait
considérablement les intuitions mathématiques importantes,
rendant la communication stérile.

\section{Les ordinateurs entrent en scène}
\label{sec:assistants-preuve}

Un nouvel élément vient cependant changer fondamentalement cette situation :
l’avènement des ordinateurs.

\subsection{Pourquoi les ordinateurs ?}

Les ordinateurs excellent là où les humains pêchent : leur spécialité est de traiter
d’immenses volumes d’information très précise, exactement le type
de besoins que soulève la manipulation de preuves formelles. C’est pourquoi, dès
la fin des années 60\sidenote{
  Avec des systèmes comme Automath \cite{??}, Mizar \cite{??}…}
ont commencé à apparaître ce que l’on appelle \intro{assistants à la
preuve}, des outils informatiques servant à écrire, vérifier et communiquer des
preuves.
Via la formalisation de ces preuves et la vérification par l’ordinateur qu’elles
suivent bien les règles du système logique sous-jacent, les assistants à la preuve
donnent accès à une fiabilité bien plus important que celle des preuves
traditionnelles. Des mathématiciens de premier plan comme
Hales~\sidecite[0em][Preface, p. xi]{Hales2012},
Voevodsky~\cite{??} ou
Scholze~\cite{LiquidTensorExperiment}
se sont déjà emparés des assistants à la preuve dans le but de lever les incertitudes
sur la solidité de leur propre travail.

Mais le terme d’\emph{assistant} à la preuve n’a pas été choisi par hasard : au-delà
de la simple vérification, les assistants à la preuve ouvrent la porte à un large
éventail d’outils mis à la disposition de la programmeuse, souvent de façon
interactive. La preuve est ainsi construite comme le produit d’un échange entre 
la programmeuse et l’outil, plutôt que fournie d’un seul bloc.
Il peut s’agir de simples
facilités, comme la possibilité de pouvoir visualiser la structure des
preuves, de suivre l’utilisation des hypothèses…
Mais l’informatique rend surtout possible l’automatisation de pans entiers
de l’écriture de preuves,
par exemple via l’utilisation d’un langage de tactique \cite{??} qui permettent
de programmer la manière dont les preuves sont
générées, voire en utilisant directement la recherche en preuve automatique
\cite{Sledgehammer,SMTCoq}.
\textit{In fine}, cette automatisation permet généralement d’écrire
les preuves à haut niveau, en laissant à l’assistant à la preuve le soin
de construire une preuve formelle.
Un autre axe important, bien qu’encore relativement peu développé, concerne
les interactions entres les outils informatiques dédiés au calcul mathématique
(systèmes de calcul formel, analyse numérique) et les assistants à la
preuve sont une piste très prometteuse.

Enfin, si l’utilisation de programmes de toutes sortes permet de grandement augmenter
les possibilités offertes par les assistants à la preuve,
la présence au même endroit – l’ordinateur –
de preuves et de programmes est également parfaite pour… la preuve de programmes.
Ils offrent en effet un cadre naturel dans lequel décrire au même endroit
le code source d’un programme, sa spécification et la preuve formelle que le
programme s’exécute correctement,
remplissant sa spécification sans rencontrer de bug.
\todo{Citation ?}

\subsection{Logique, programmation et théorie des types}

Pour fonctionner, les assistants à la preuve ont besoin d’une
description formelle des "règles du jeu" mathématiques qu’ils sont censés imposer.
En clair, ils demandent une étude renouvelée de la logique, mais dans le but
pratique de construire des outils à la fois fonctionnels,
puissants et faciles à utiliser.
Il existe plusieurs familles d’assistants à la preuve, basées sur des fondements
logique relativement différents. La famille qui m’intéresse dans cette thèse, est
celle à laquelle appartient Coq, celle basée sur la
\kl{correspondance de Curry-Howard} et la \kl{théorie des types dépendants}.

Si on compare un programme informatique à un texte dans une langue naturelle,
les types sont une sorte d’équivalent des catégories grammaticales.
Contrairement aux langues naturelles, cependant, ces règles de typage sont conçues
en même temps que le langage de programmation, en général
de telle sorte à refléter des propriétés
des objets manipulés par le programme. Cela permet en premier lieu de
détecter des erreurs manifestes.
Par exemple, si une procédure attendant un objet de type "image" est
appliquée à un objet de type "chaîne de caractères", une erreur pourra être rapportée
à la programmeuse.\sidenote{
  Un slogan dû à Milner~\sidecite[5em]{Milner1978} est que
  « Les programmes bien typés ne peuvent pas  mal s’exécuter. »
}
Mais les utilisations des types sont très versatiles, et leurs capacités à encoder
des propriétés des programmes sous-jacents peuvent servir à la compilation, la
documentation, etc.

\begin{marginfigure}[2em]

  % \begin{mathpar}
  %   \inferrule{ \Gamma, A \vdash B}{\Gamma \vdash A \Rightarrow B} \and
  %   \inferrule{\Gamma \vdash A \Rightarrow B \\ \Gamma \vdash A}{\Gamma \vdash B} \and
  %   \inferrule{\Gamma, x : A \vdash t : B}{\Gamma \vdash \lambda x : A . t : A \to B} \and
  %   \inferrule{\Gamma \vdash f : A \to B \\ \Gamma \vdash u : A}{\Gamma \vdash f~u : B}

  % \end{mathpar}

  % \caption{Règles d’inférence pour l’implication et de typage des fonctions}

  \begin{mathpar}
    \inferrule{A \\ B}{A \wedge B} \and
    \inferrule{A \wedge B}{A} \and
    \inferrule{A \wedge B}{B} \\
    \inferrule{a : A \\ b : B}{(a,b) : A \times B} \\
    \inferrule{p : A \times B}{p.1 : A} \and
    \inferrule{p : A \times B}{p.2 : B}
  \end{mathpar}
  
  \caption{Règles d’inférence pour la conjonction et de typage pour les paires}
  \label{fig:curry-howard-exemple}
\end{marginfigure}

Plutôt qu’un théorème précis, la \intro{correspondance de Curry-Howard} est une
idée très générale,
selon lequel il existe une ressemblance forte entre d’un côté le monde de la
logique et des preuves, et de l’autre celui des programmes et de leurs types.
Un exemple valant mieux qu’un discours abstrait, on peut voir la correspondance à l’œuvre dans la \cref{fig:curry-howard-exemple}, sous la forme de règles d’inférence
ou de typage : chaque bloc présente une règle, avec au-dessus de la barre les
hypothèses, et en dessous la conclusion.
Les trois premières règles gouvernent la conjonction logique $\wedge$.
La première signifie que pour déduire la proposition $A \wedge B$,
il suffit de déduire $A$ et $B$ individuellement.
À l’inverse si a comme hypothèse $A \wedge B$, alors on peut déduire $A$ et $B$
individuellement.
Les trois dernières règles gouvernent le typage du type des paires $A \times B$.
Une paire $(a,b)$ a le type $A \times B$ si $a$ est de type $A$ et $b$
est de type $B$.
À l’inverse si $p$ est de type $A \times B$, alors sa première projection $p.1$
est de type $A$ et sa seconde projection $p.2$ est de type $B$.
Si on efface les termes\sidenote{
  Dans ce contexte, on parle souvent de \emph{termes} plutôt que de programmes,
  mais les deux sont synonymes.
} des règles du bas, on obtient \emph{exactement} les règles du haut !

Ceci s’étend bien au-delà du cas de la conjonction,
en une correspondance générale entre d’une part les énoncés de la logique et les types des langages de programmation – \emph{propositions as types} –, et d’autre part les preuves d’un énoncé et les programmes ayant le type correspondant – \emph{proofs as programs}.
Au-delà de la simple analogie entre formalismes d’origines différentes, cette correspondance est un outil puissant pour faire dialoguer deux mondes.
En particulier, elle permet de relier deux problèmes \textit{a priori} éloignés :
la vérification de la correction d’une preuve et l’analyse du type d’un terme.

La correspondance de Curry-Howard est donc idéale pour servir de fondements aux
assistants à la preuve, puisqu’elle permet de voir un système
comme une logique, tout en autorisant l’utilisation d’idées venant de
la large littérature sur les langages de programmation, notamment
la théorie et l’implémentation des systèmes de types.
Dans ce cadre, les \intro{systèmes de types dépendants} sont une famille dont
la caractéristique principale, comme leur nom l’indique, est d’autoriser les
types à dépendre de termes. L’exemple archétypique du point de vue de la 
programmation est le type
$Vect~A~n$\todo{setup macros} des listes contenant exactement $n$ éléments
de type $A$, où $n$ est un entier.
Du point de vue de la logique, cette
dépendance correspond aux quantificateurs, nécessaires pour exprimer des
propriétés universelles – pour tout entier $n$, la propriété $P(n)$ est
vérifiée — et existentielles – il existe un entier $n$ tel que $P(n)$ tient,
sans lesquelles il est tout bonnement impossible d’exprimer la plupart des
mathématiques. En revanche les systèmes à types dépendants, eux, sont suffisamment
riches et puissant pour espérer y développer la plupart des mathématiques.

\section{Grandeur et décadence d’un assistant à la preuve}
\todo{Trouver un meilleur titre}
\label{sec-coq}

  Intéressons-nous un peu plus en détail à l’assistant à la
  preuve dont il sera le plus question dans cette thèse : Coq.

  \subsection{La clé de voûte du système}
  
  \begin{figure}[h]

    \centering
    \includegraphics{./figures/coq-kernel-fr.pdf}
  
    \caption{Le fonctionnement de Coq}
    \label{fig-coq}
  \end{figure}

  Coq est basé sur la \kl{correspondance de Curry-Howard} : les preuves sont vues comme des programmes dans un langage appelé \intro{Gallina},
  et leur vérification est effectuée par un algorithme proche
  de ceux utilisés pour les types des langages conventionnels.
  Cependant, si dans les premières versions des années 80 Coq ressemblait à un langage de programmation un peu étrange, ce n’est actuellement plus du tout le cas.
  La raison, comme on l’a expliqué plus haut, est que Coq est un \emph{assistant} à la preuve.
  C’est pourquoi la majeure partie de Coq dans ses versions actuelles a pour but d’aider l’utilisatrice à générer une preuve correcte sans avoir à l’écrire directement.
  Ce fonctionnement est illustré en \cref{fig-coq} : l’utilisatrice échange interactivement avec Coq, qui utilise cette interaction pour générer un terme de preuve. Celui-ci est ensuite envoyé à une partie bien spécifique du système, appelée \intro{noyau}.
  C’est lui qui implémente l’algorithme de vérification de type, et s’assure ainsi de la correction des termes de preuve construits interactivement.
  Le noyau est donc l’élément crucial du système, car c’est lui – et lui seul – qui est responsable en dernier recours de la validation des preuves.
  
  
  Cette architecture, qui isole clairement la partie critique du système
  est appelée \intro{critère de de Bruijn} \sidecite{Barendregt2001}, en 
  hommage à l’un des pionniers des assistants à la preuve.
  Elle a permis de mener à bien des projets de grande ampleur, parmi lesquels CompCert \sidecite{Kaestner2017} – un compilateur optimisant pour le langage C entièrement prouvé correct –, ou les preuves du théorème des quatre couleurs \sidecite{Gonthier2007} et du théorème de l’ordre impair \sidecite{Gonthier2013}, deux théorèmes importants et difficiles respectivement de la théorie des graphes et des groupes.
  Cependant, si le reste de l’écosystème s’est beaucoup plus développé que le noyau depuis les débuts, celui-ci a également évolué, en se complexifiant graduellement.
  Et comme tout développement logiciel, le noyau n’est pas à l’abri de bugs\sidenote{De l’ordre d'un bug détecté par an, une liste est maintenue à l’adresse suivante : \url{https://github.com/coq/coq/blob/master/dev/doc/critical-bugs}.}.
  Ceux-ci sont en général difficilement exploitables par une utilisatrice de Coq, encore plus sans s’en rendre compte.
  Néanmoins, ils existent, et la tendance à la complexification du noyau ne risque pas de s’arrêter de si tôt.

\subsection{MetaCoq, une formalisation par Coq, pour Coq}
\label{sec-metacoq}

Si on veut garantir un niveau de fiabilité le plus élevé possible, il faut donc de nouvelles idées.
Le projet MetaCoq, a pour but de répondre à cette problématique.
L’idée est simple : il s’agit d’utiliser Coq lui-même pour certifier la correction de son noyau.

Plus précisément, la première étape est de décrire à haut niveau le système de type sur lequel est basé le noyau, puis de démontrer ses propriétés théoriques.
% , comme la confluence de la réduction, la préservation du typage par cette même réduction, etc.
Une fois ces propriétés établies, la deuxième étape consiste à programmer un algorithme de vérification de type ressemblant au maximum à celui du noyau, directement en Gallina,\sidenote{
  En effet, grâce à la correspondance de Curry-Howard, Gallina est certes un langage pour décrire des preuves, mais aussi un véritable langage de programmation fonctionnel !}
tout en démontrant qu’il est bien correct\sidenote{
  Si l’algorithme prétend qu’un terme est bien typé, alors c’est bien le cas.}
et complet\sidenote{
  L’algorithme répond bien affirmativement sur tous les termes bien typés.}.
Enfin, une troisième étape extrait de ce programme Gallina certifié
un programme efficace qui puisse être utilisé en lieu et place du noyau actuel.
Cette extraction est elle-même complexe, car pour obtenir ce programme efficace il
faut effacer le contenu lié à la preuve de correction
pour ne garder que le contenu algorithmiquement intéressant.
C’est pourquoi là encore on prouve que l’extraction est correcte\sidenote{
  C'est-à-dire qu’elle préserve la sémantique des programmes.},
à nouveau en la programmant en Gallina.

Pour prouver la complétude de l’algorithme de typage, il est très utile de
passer par une spécification intermédiaire plus structurée que la description
théorique du système de type utilisée dans la première étape.
En particulier, il est important de séparer deux problèmes proches, mais
différents :
d'une part, la vérification, où on cherche à \emph{vérifier}
qu’un terme a bien un type
donné ; de l’autre, l’inférence, où on cherche à \emph{trouver}
un type pour un terme, s'il existe.
L’algorithme de typage du noyau de Coq est bidirectionnel,
c'est-à-dire qu’il alterne en permanence entre ces deux problèmes
lorsqu’il vérifie qu’un terme est bien typé.
% Par exemple, dans le cas d’une application $f~u$, il commence par inférer un type pour $f$, vérifie qu’il s’agit d’un type produit $\P x : A . B$ (une généralisation du type fonctionnel $A \to B$), puis vérifie que $u$ a le type $A$.
Décrire formellement cette structure bidirectionnelle plus proche de l’algorithme
permet de bien diviser les difficultés entre d’un côté
son équivalence avec la présentation
originale, et de l’autre la partie purement liée aux questions d’implémentation.

\subsection{Un peu de flexibilité dans un monde désespérément statique}


\section{Et cette thèse, alors ?}
\label{sec:cette-these}

Bien que très puissante, l’utilisation des types dépendants a un coût,
car du fait de leur puissance logique, leur  tous leurs aspects
est très complexe.

\subsection{Le typage bidirectionnel}

Ce typage bidirectionnel présente par ailleurs un certain nombre d’avantages théoriques au-delà du cadre de MetaCoq, qui sont cruciaux pour le travail évoqué en \cref{sec-graduel}.

J’ai donc entrepris dans \cite{LennonBertrand2021} à la fois l’étude théorique de ce typage bidirectionnel dans le cadre complexe de Gallina, et la preuve en Coq de l’équivalence avec le système non-dirigé.
Au passage, cette preuve a permis de détecter et corriger un bug dans le noyau qui était passé inaperçu jusque là.
Il reste maintenant à prouver la complétude de l’algorithme vis-à-vis de cette nouvelle spécification, travail qui est en cours.

\subsection{MetaCoq}

\subsection{Élaboration bidirectionnelle pour le typage graduel}


\selectlanguage{english}

\chapter{Introduction}
\label{chap:intro-en}

\emph{“\kl{Coq} is an old man now, and it has a lot of scars.”}
\vspace{-1.5em}
\begin{flushright}
  \sidecite[][citing Assia Mahboubi]{QuantaPA}
\end{flushright}

\margintoc[4em]

This thesis belongs to the domain of \kl[dependent type]{type theory},%
\sidenote{If you do not know what this or any other word in this introduction
means, read on! They will be explained in due time.}
itself at the crossroads between computer science and mathematical logic.
One of the field’s goals is to give theoretical and practical foundations
for software tools helping humans in constructing and verifying proofs –
in the mathematical sense.
Such tools are called \kl{proof assistants}, and \kl{Coq}, the one
on which my work was mainly focused, is central in this thesis.

Over their more than 50 years of existence, proof assistants have
turned into an established technology. This history is both a blessing and a curse: as
the field matured, the tools have become more and more complex, making them more and more
powerful, but also more and more prone to critical bugs hiding in dark corners. At a time
when they are gaining traction in an increasing number of communities
concerned with high trust levels, this simply cannot be.
The historical solution of keeping a small, trusted \kl{kernel}
– the so-called De Bruijn criterion –
is not enough if we wish to keep moving on and integrate new, powerful features
to keep up with the needs of users.

There is a straightforward solution to this:
proof assistants have been used for decades to certify programs correctness.
Why could they not prove \emph{themselves} correct? After all, if this is
the gold standard we demand for software, it should apply first and foremost to the ones
used to justify that trust. For the proof assistant \kl{Coq},
this is the ambition of the \kl{MetaCoq} project,
which aims at providing a drop-in replacement for \kl{Coq}’s \kl{kernel} that has been
proven correct,
even though it handles all the subtleties and quirks of said \kl{kernel}.
No more trusting a complex and ever-evolving implementation, trust the formally validated
\emph{proofs} instead!

But before we can hope to achieve that goal, we need a deeper study of the structures at work
in the \kl{kernel}. In particular, its typing algorithm is \emph{bidirectional}, meaning that
it constantly alternates between the two problems of type \emph{inference} –
finding a type for a term – and type \emph{checking} –
verifying that a type is adequate for a term. While this
structure is crucial in relating the specification of the type system to its implementation,
it has been rather little studied in the context of the
\kl{Calculus of Inductive Constructions} (\kl{CIC}),
the theoretical foundation of \kl{Coq} – but also of the closely related
\kl{Lean}, \kl{Agda}…

This thesis aims at filling that gap, by providing a thorough study of bidirectional \kl{CIC},
formalized in the framework offered by \kl{MetaCoq} project. This is a key
ingredient in the first formal proof of soundness and completeness of a type-checking
algorithm for a realistic proof assistant kernel.
It was also able to uncover bugs in \kl{Coq}’s kernel that had gone unnoticed until then.

But bidirectional typing is also an interesting theoretical tool in its own right,
giving a valuable form of control over computation.
In particular, it is a necessary piece in the design of a gradual extension of
\kl{CIC}, \kl{GCIC}.
\kl{Gradual typing} aims at bringing to programmers both the flexibility of
development offered by dynamic typing, and the strong guarantees given
by static typing, in one and the same system. \kl{GCIC} intends
to bring that flexibility to dependently-typed programming,
and, by using the power of the \kl{Curry-Howard correspondence}, to proof writing.
But this endeavour comes with subtle difficulties,
that can only be solved in a bidirectional setting.

To replace this work in its larger context, this introduction begins with a very
short history of mathematical logic (\cref{sec:logic-history}), which exposes the
main questions of that field. Follows a presentation of the links between logic and
computer science, through \kl{proof assistants} (\cref{sec:proof-assistants}).
Next, \cref{sec:intro-coq-en} focuses more closely on presenting
the research questions I worked on: bidirectional typing, \kl{MetaCoq} and gradual typing.
Finally, \cref{sec:this-thesis} summarizes my contributions to these questions.

\section{A Very Short History of Logic}
\label{sec:logic-history}

\subsection{Syllogisms}

The main question that logic seeks to answer is that of finding criteria in order to determine
if a reasoning is valid. In Western tradition, this challenge can be traced back to the
Antiquity, and particularly to Aristotle's \textit{Organon}.
The main contribution of this work is to introduce the notion of syllogism.
These are simple fragments of reasoning, whose validity stems from the
fixed structure they follow, rather than a specific content.%
\sidenote{The most well-known is probably the \textit{Barbara} syllogism, and example
of which is: \emph{all humans are mortals; Socrates is human; so Socrates is mortal.}}
If complex reasoning is built from assembling such syllogisms, it must necessarily be valid as
a whole, since every assembled fragment is. There are two important ideas at work here.

The first is that reasoning can be valid or not, depending only on its structure,
independently of its content.
It can be syllogisms, but many other systems. We will come across a certain number of them
in this thesis!

The second idea is that of a construction from elementary components.
Starting from a set of rules
we have identified as valid \textit{a priori}, we have a means to ensure the validity
of potentially very complex reasoning: it suffices to check that these
can be decomposed into the base components.

For the Greek philosophers, logic was also conceived as a means towards communication.
The aim was to check one’s own reasoning, but also to be able to convey
it, by fixing a logical formal system.%
\sidenote{Structural rules reasoning should obey, as those of syllogisms.}
A person wanting their conclusion to be accepted by others would only have to express their
reasoning in a perfectly precise way in the framework of such a formal system.

From that point on, the main focus of logic as a discipline
concentrates on this structure which underlies reasoning.
The main challenge is to construct a formal system, adapted to a specific
field of reasoning. In the case we are interested in, mathematical logic, this
allows us to give a precise meaning to what constitutes a valid mathematical proof.


\subsection{The beginning of mathematical logic: towards a formal foundation}[Towards a formal foundation]

Following Aristotle, mathematicians seized logic in order to build a formal system
able to serve as a rigorous foundation for mathematics.
The links between logic and mathematics go back to Greek Antiquity, but
mathematical logic as a standalone discipline really established itself
during the 19\textsuperscript{th} century, thanks to important progress on two main aspects.

The first consisted in freeing mathematical logic from natural languages%
\sidenote{By opposition with the formal languages which appear in mathematics,
  computer science, etc.},
unsuited to a formal description of reasoning, and to instead design a new specific
form of language that could serve as a basis for mathematical reasoning.
An important step here was \citeauthor{Begriffsschrift}'s
\citetitle{Begriffsschrift}~\sidecite{Begriffsschrift}, which, for the first time,
gave a formal language rich enough to express mathematics satisfyingly. Its
major addition was the notion of quantifier, essential to the mathematical vernacular,
as they give a faithful way to account for universal%
\sidenote{For instance: “Every even natural number is the sum of two prime numbers”.}
and existential%
\sidenote{For instance: “There exists a real whose square is 2”.}
properties.

The second aimed at showing that mathematics as a whole could be reconstructed from a
few simple properties. An important step was the reduction of analysis to the properties
of real numbers, followed by constructions of those from arithmetic given almost
simultaneously by – among others – \sidetextcite{Dedekind1872} and
\sidetextcite{Cantor1872} in 1872.
Meanwhile, \sidetextcite{Peano1889} proposed an axiomatization of natural numbers close to the
one still used today. Finally, Cantor again proposed set theory \sidecite{Cantor1883}
as a formalism expressive enough to describe all mathematical object as sets of elements.

\subsection{The foundational crisis of mathematics}[The foundational crisis]

Unfortunately, the system proposed in the \citetitle{Begriffsschrift} is inconsistent !
That is, it is possible to use it to prove falsity, 
making the logical system collapse.%
\sidenote{In a system where falsity is provable, all propositions are,
  which is known as the principle of explosion.
  Such a system, where everything – and its negation – is provable can obviously not
  serve as an adequate foundation for mathematics.
}
This result, due to Russell%
\sidenote{
  In a letter to Frege in 1902 the latter made made public
  in \textcite[Nachwort p.~253]{Frege1903}.}%
\margincite{Frege1903}
marked the opening of a crisis period.
Indeed, it cast doubt upon the systems that had started to establish
themselves as good candidates to serve as foundations – that of Frege, but
mainly those of Cantor, which were affected by the same difficulties.

A possible solution has been suggested ten years later  by \citeauthor{Whitehead1913} in their
\citetitle{Whitehead1913} \sidecite{Whitehead1913}. This colossal piece of work
not only proposed a formal system avoiding the inconsistency
of \citetitle{Begriffsschrift}. It also built a significant amount
of mathematics in this system, including a construction of integers,
some arithmetic, and finally real numbers.

In parallel, in the continuity of Cantor’s work, \sidetextcite{Zermelo1908} and others
worked towards giving a version of Cantor’s set theory that is consistent. This lead to what
is colloquially referred to as Zermelo-Fraenkel set theory – ZF, or ZFC when the
axiom of choice%
\sidenote{An axiom very useful in numerous branches of mathematics, but which is often treated
separately, as it is both less crucial than the other axioms of ZF and at the root of
counter-intuitive results.}
\sidecite{Zermelo1904} is added –, which also seemed able to serve as a
solid foundation for mathematics.

\subsection{Incompleteness}

The search for a formal system adequate as a foundation for mathematics however hit a
second major difficulty: Gödel’s incompleteness theorem \sidecite{Goedel1931}. It asserts
that a formal system in which one can construct integers such as those of Peano – and so
\textit{a fortiori} any system rich enough to serve mathematician’s needs – cannot
prove its own consistency.%
\sidenote{Unless the system is inconsistent, in which case it can prove \emph{everything},
by virtue on the explosion principle, including its own consistency… and inconsistency!}
Thus, no formal system can serve as a basis for mathematics
with a formal certitude as to its adequacy.
Indeed, as we cannot prove the consistency of the system in itself, it could very well
turn out to be inconsistent, ruining all the efforts put into its use – just like what
happened with Frege’s \citetitle{Begriffsschrift}. And if we were to use a second system
to prove the first consistent, we would only shift the prolem: now we rely on the
consistency of the second system.

A consequence of this theorem is that a system rich enough to found mathematics is
necessarily incomplete.%
\sidenote{
  This means that there exist independent statements, that is assertions which
  cannot be proven, and whose negation cannot be proven either.
  The consistency of the system under consideration is one example of such a statement.
}
Thus, in what follows, I will never refer to truth in an absolute sense – which could
only be meaningful in a complete system where every statement is true or false –, but
only about provability \emph{relatively to a given system}.

\subsection{A satisfactory situation?}

Despite the difficulties put into light in the beginning of the 20\textsuperscript{th}
century, the research in mathematical logic reached a somewhat satisfactory situation
a few decades later.
First, ZFC is a reasonable formal system on which mathematics can be founded. Moreover,
the mathematical community is overall convinced it would be \emph{theoretically} possible
to write down all mathematics using ZFC. This is enough for most of its members,
even if those who attempt to actually give it a try, in the vein of the
\citetitle{Whitehead1913}, are quite few.

In \emph{practice}, however, things are very different. The human development and
verification of formalized mathematics%
\sidenote{
  That is, effectively expressed in a fixed formal system.}
seems both impossible, and unnecessary.
On the one hand, it would demand a considerable effort, because such mathematics would
require an extremely high level of precision, both from the author of the formal proof
and from the reader. At the same time, this would not significantly reduce the risk of
errors. It would indeed be very hard for humans to check that some reasoning doubtlessly
follows the rules of the system: a tiny error can easily creep inside thousands of pages
of formal reasoning. Finally, describing mathematics in this way would drown the vital
mathematical intuitions, making communication sterile.

If we wish to make formal mathematics practicable, and benefit from the guarantees
they bring while eliminating these crippling defaults, we thus need new tools.

\section{Computers Enter the Scene}
\label{sec:proof-assistants}

A new element however radically modifies the previous situation: the advent of computers.
Indeed, computer science provides new tools, making formalized mathematics both possible
and attracting.

\subsection{Proof assistants}

Computers excel where humans are weak: their speciality is to treat large volumes of
information in a very precise way, exactly the kind of needs brought up when manipulating
formalized mathematics. Therefore, already at the beginning of the 70s,%
\sidenote{With systems like Automath \cite{DeBruijn1970}, or Mizar
  \cite{Rudnicki1992}.}%
\margincite{DeBruijn1970}%
\margincite{Rudnicki1992}
software tools, collectively called \intro{proof assistants}, start to
appear, that are dedicated to writing and verifying formal proofs.
Through the formalization of proofs and the verification by computers that they
actually follow the rules of the underlying logical system, proof assistants open the
door to a level of trust much higher than that allowed by “informal” proofs.
Renowned mathematicians, such as \sidetextcite{Voevodsky2010},
\sidetextcite[][Preface, p. xi]{Hales2012}, or \sidetextcite{Scholze2021} have indeed
turned to proof assistants, particularly in order to lift uncertainties regarding the
solidity of their own work.

Moreover, proof \emph{assistants} are not simply proof checkers: beyond verification,
they supply users with a large range of tools to ease the conception of
formal proofs. These tools allow users to write proofs at a
high level, and in an interactive manner,%
\sidenote{In most modern proof assistants, the final proof is built as the result of
  an exchange between the programmer and the tool, rather than written as a single block.}
leaving it to the proof assistant to construct the formal proofs.
They range from simple facilities, such as the possibility to visualize the structure
of proofs, or the tracking of hypotheses, to much more ambitious techniques.

Indeed, computer science lets us automatize entire parts of
proof writing, for instance through the use of tactic languages \sidecite{Delahaye2000},
with which one can program proof generation.
In addition, the automatic construction of proofs is a research field by itself,
and the question of its integration intro proof assistants is an active topic
\sidecite{Blanchette2016,Ekici2017}. Computer science has also proven its worth in the
setting of mathematical computations (computer algebra systems, numerical analysis),
and here again promising interactions with proof assistants are starting to arise
\sidecite{Lewis2022,Mahboubi2019}.

Finally, if the use of software eases the writing of proofs, proof assistants conversely
open new possibilities for programming. They indeed offer a natural framework to describe in
the same place the source code of a program, its specification, and the formal proof that the
former fulfils the latter. This way, we can \emph{prove} that the program runs correctly,
without encountering any bugs.
This mathematical certainty is much more reliable than any test set!
In this field, numerous projects have already achieved large scale programs, entirely proven
correct: compiler for the C language \sidecite{Kaestner2017}, implementation of the
\textsc{Https} protocol \sidecite{Bhargavan2017}, differential equations solving
\sidecite{Immler2018}…

\subsection{Logic, Programming and Type Theory}

In order to work, proof assistants must be founded on a formal system, corresponding to
the “rules” of the mathematical “game” they are supposed to enforce.
Thus, they require a renewed study of mathematical logic, but with the practical aim of
building tools that are at the same time powerful and easy to use.
There are multiple families of proof assistants, based on very different formal systems.
The one I am interested in in this thesis relies on the \kl{Curry-Howard correspondence}
and \kl[dependent type]{dependent type theory}. The proof assistant \kl{Coq}
\sidecite{CoqDevelopmentTeam2022}, which is at the heart of my work, belongs to this family.

If one compares a computer program with a text in a natural language,
\intro(en){types}
are a kind of equivalent of grammatical categories. However, contrarily to natural
languages, these types are conceived at the same time as the programming language, in order
to mirror properties of the objects it manipulates.
Their first use is to detect manifest errors. For instance, if a procedure
intended for an object of type “image” is applied to an object of type “character string”,
an error can be reported to the programmer.%
\sidenote{A well-known slogan due to \textcite{Milner1978} claims that
“Well-typed programs cannot go wrong.”}%
\margincite{Milner1978}
But types are very versatile, and their capacity to encode properties of the underlying
programs can be used for compilation, documentation, and many other applications. In our
framework, for instance, types correspond to the validity of a logical reasoning.

This idea is that of the \intro{Curry-Howard correspondence}.%
\sidenote{Made explicit for the first time in informal notes by Howard dating back to 1969,
but published only much later \cite{Howard1980},
themselves based upon previous remarks by Curry \cite{Curry1958}.}%
\margincite{Howard1980}%
\margincite{Curry1958}
Rather than a precise theorem,
it is more of a very general concept, according to which two worlds closely resemble each
other: on the one hand, that of logic and proofs, on the other that of programs
and their types.

\begin{marginfigure}

  % \begin{mathpar}
  %   \inferrule{ \Gamma, A \vdash B}{\Gamma \vdash A \Rightarrow B} \and
  %   \inferrule{\Gamma \vdash A \Rightarrow B \\ \Gamma \vdash A}{\Gamma \vdash B} \and
  %   \inferrule{\Gamma, x : A \vdash t : B}{\Gamma \vdash \lambda x : A . t : A \to B} \and
  %   \inferrule{\Gamma \vdash f : A \to B \\ \Gamma \vdash u : A}{\Gamma \vdash f~u : B}

  % \end{mathpar}

  % \caption{Règles d’inférence pour l’implication et de typage des fonctions}

  \begin{mathpar}
    \inferrule{A \\ B}{A \wedge B} \and
    \inferrule{A \wedge B}{A} \and
    \inferrule{A \wedge B}{B} \\
    \inferrule{a \ty A \\ b \ty B}{(a,b) \ty A \times B} \\
    \inferrule{p \ty A \times B}{p.1 \ty A} \and
    \inferrule{p \ty A \times B}{p.2 \ty B}
  \end{mathpar}
  
  \caption{Inference rules for conjunction and typing rules for pairs}
  \label{fig:curry-howard-example-en}
\end{marginfigure}

A short example says more than a long abstract talk, so let’s look at the correspondence
at work in \cref{fig:curry-howard-example-en}, in the form of inference/typing rules:
each bloc presents a rule, with above the bar the hypotheses, and below the conclusion.
The first three rules govern the logical conjunction “and”, written $\wedge$.
The first means that to deduce the proposition $A \wedge B$ (“$A$ and $B$”), it is enough
to deduce $A$ and $B$ taken individually.
Conversely, if we have as hypothesis $A \wedge B$, then we can deduce both $A$ (second rule),
and $B$ (third rule).
The last three rules govern typing%
\sidenote{Written using a colon.}
for the pair type $A \times B$. A pair $(a,b)$ built
from a first object $a$ of type $A$ and a second object $b$ of type $B$ has type $A \times B$.
Conversely, if $p$ is a pair of type $A \times B$, then we can retrieve its first component
$p.1$, which is of type $A$, and its second $p.2$, of type $B$.
If we erase the terms%
\sidenote{ In the context of type theory, we often talk about \emph{terms} instead of programs,
  but the two are synonyms.
}
of the bottom rules, we obtain \emph{exactly} the rules above!
Thus, the programming construct of pairs corresponds to the logical concept of conjunction.

This extends well beyond the specific case of conjunction, in a general correspondence
between, on one side, logical propositions and their proofs, and, on the other, types and programs.
We can see properties as types, and a proof of a given property as a program of the
corresponding type – or the other way around!
Beyond a simple analogy between formalisms of different origins, this correspondence
is a powerful tool to establish a dialogue between two worlds. In particular, it
relates two \textit{a priori} quite distant problems: checking that a proof
is valid, and checking that a term is well-typed. In both cases, it amounts to checking that
a construction – program on one side, proof on the other – respects a set of formal
rules guaranteeing it is well-formed.

The \kl{Curry-Howard correspondence} is therefore ideal to serve as a foundation for
\kl{proof assistants}, since it gives access, when studying formal logical systems,
to the rich literature on programming languages, in particular on the theory and
implementation of types. In this framework, the
\intro[dependent types]{dependent type systems} are a specific family of type systems,
whose main characteristic is the ability for types to depend on terms. The archetypical
example from the point of view of programming is the type $\Vect(A,n)$
of vectors of length $n$. These are lists that contain exactly $n$ elements of type $A$ – with
$n$ a natural number.
This type depends on $n$, in the sense that the type’s inhabitants differ depending on the
integer’s value.
From the point of view of logic, this dependency corresponds to quantification: if we
wish to express a universal property “for all $x$, the property $P(x)$ holds”, then we need
the property $P$ to depend on $x$.
Thanks to this ability to express quantification, dependent types are rich enough
to serve as foundations for mathematics.

\section{\kl{Coq} and Its Kernel}
\label{sec:intro-coq-en}

Let us now focus a bit more on the proof assistant which we will consider mainly in this
thesis: \kl{Coq}.

\subsection[The kernel]{The kernel, cornerstone of the system}

\begin{figure}[h]

  \centering
  \includegraphics{./figures/coq-kernel-en.pdf}

  \caption{\kl{Coq}’s schematic architecture}
  \label{fig:coq-en}
\end{figure}

\kl{Coq} is based on the \kl{Curry-Howard correspondence}: proofs are seen as programs,
in a language called \intro{Gallina}, and their verification is done using an algorithm
close to those used for types in conventional languages. However, if, in the first versions
from the 80s, \kl{Coq} proof were mostly written directly in \kl{Gallina}, it is
no longer the case at all. The reason is that the major part of the tool in its
current versions aims at helping the user in generating a correct proof. It is a true
\kl[proof assistant]{proof \emph{assistant}}!
The way \kl{Coq} works is illustrated in \cref{fig:coq-en} : the user interactively exchanges
with \kl{Coq}, which uses this interaction to generate a proof term. This proof term is then
sent to a very specific part of the tool, called the \intro{kernel}.
This is the part implementing the type-checking algorithm, and thus responsible for ensuring
that the proof terms built interactively are correct.
The \kl{kernel} is thus the crucial part of \kl{Coq}, because it is the one – and only –
ultimately responsible for proof-checking.
This architecture, which clearly isolates the critical part of the system, is called
\intro{De Bruijn criterion} \sidecite{Barendregt2001}, in tribute to one of the pioneer
of proof assistants.

If the rest of the ecosystem has grown much more than the \kl{kernel} since the beginning,
the latter has also evolved, becoming gradually more complex.
And, as any other software development, it is not safe from bugs.%
\sidenote{ The magnitude is that of one critical bug found every year, a list is maintained
at the following address: \url{https://github.com/coq/coq/blob/master/dev/doc/critical-bugs}.}
These are in general hard to exploit for a user, even more so without noticing.
But still, they exist, and since the \kl{kernel} tends to get more and more complex, they
are likely to continue appearing.

\subsection{\kl{MetaCoq}, a formalization in \kl{Coq}, for \kl{Coq}}[\kl{MetaCoq}]
\label{sec:intro-metacoq-fr}

If we wish to guarantee a trust level as high as possible in the \kl{kernel}, we must
resort to new ideas. This is what the \kl{MetaCoq} project is all about. The idea
is simple: use \kl{Coq} itself to certify the correctness of its \kl{kernel}.

More precisely, the first step is to describe formally the type system on which the \kl{kernel}
is based, and to show its theoretical properties.
This is already a difficult endeavour: in order to ease its use, \kl{Coq}’s type theory
incorporates a lot of complex features.

Once this meta-theory is established, the second step
% \sidenote{This is the one on which I mostly work, and on which we will come back in more
% length later on.}
consists in implementing a type-checking algorithm as close as possible to the one of the
\kl{kernel}, directly in \kl{Gallina}%
\sidenote{Indeed, thanks to the \kl{Curry-Howard correspondence}, \kl{Gallina} is not
only a proof language, but also a true programming language!}.
We show, while defining the algorithm, that it is indeed \reintro(bidir){sound}%
\sidenote{If the algorithm claims that a term is well-typed, then it is the case.}
and \reintro(bidir){complete}%
\sidenote{The algorithm answers positively on all well-typed programs.}.
Together, these two properties correspond to the \intro(bidir){correctness} of
the program.

Finally, in a third step, we extract out of this certified \kl{Gallina} program another
more efficient program, by erasing the content related to the proof of correctness, in order
to keep only the algorithmically relevant one.
This extraction is a complex but crucial step if we wish to replace the current \kl{kernel}
while keeping a reasonable efficiency. Therefore, we also prove that said extraction
is correct,%
\sidenote{Meaning that it preserves the semantics of programs.}
once again by programming it in \kl{Gallina}.

\subsection{Checking, inference and bidirectional typing}[Bidirectional typing]

While proving the correctness of the type-checker is relatively easy once the
meta-theoretical properties of the type system have been established, completeness is harder.
In order to prove it, it is very useful to go through an intermediate specification,
which is more structured than the theoretical one.
In particular, it is important to separate two close but distinct questions:
on the one side, type-checking, where we \emph{check} that a term indeed has a
given type;
on the other side, inference, where we try and \emph{find} a type for a term, if such a
type exists.
The typing algorithm of \kl{Coq}'s \kl{kernel} is \intro{bidirectional}, meaning that it
alternates constantly between these two processes when it checks that a term is well-typed.
Describing this bidirectional structure independently of the algorithm allows for a
clear separation between, on the one side, its equivalence with the original specification,
and, on the other, the part purely dedicated to implementation questions.

In the specific case of dependent types, even if present in type-checking algorithms since
the origin – see \eg \sidecite{Huet1989} –, bidirectional typing has been relatively little
studied. However, beyond its strong relation to algorithms, this approach also presents
theoretical advantages: its more constrained structure makes it easier
to obtain properties that are difficult to obtain in the standard context.

\subsection{Gradual types: some flexibility in a desperately static world}
  [Gradual types]
\label{sec:intro-graduel-en}

There are two main approaches to program type-checking. In the static approach,%
\sidenote{On which \kl{Coq} is based.}
types are verified prior to the execution, whereas, in the dynamic approach, the well-typedness
of operations is verified on the fly during that same execution.
The dynamic discipline is more flexible, as it checks exactly what is necessary
for the good execution of a program.
The strictness of static typing, conversely, allows for error detection earlier in the
development, and imposes invariants useful to optimize compilation or execution.

Instead of opting exclusively for one of the two approaches,
\reintro{gradual typing} \sidecite{Siek2015} aims at integrating
the static and dynamic disciplines in one and the
same language.
The main idea is to have a first pass of verification before the execution, as in static typing,
while leaving the possibility to defer parts of the verification to the execution, as in
dynamic typing.
This gives access to a whole spectrum of options, from a rigid completely static
discipline to a flexible dynamic one. It particularly allows for a fine-grained, local choice
of how each part of a program is type-checked.
One can thus evolve the discipline during software development, benefiting from
the flexibility of dynamic typing in early phases, and from the guarantees of static typing
later on.

As the case of \kl{MetaCoq} illustrates, \kl{Coq} can be used as a true programming language.
Even better: its type system can express very complex properties of programs, and thus
verify even before their execution that the code indeed enforces them.
Sadly, these reinforced constraints can turn against the user, by making the
early development phase more difficult. Indeed, nobody writes correct code on the first try,
and it would often be nice to temporarily lift the strong guarantees of typing to
facilitate experimentation. The idea then is to take inspiration from gradual typing,
in order to pave the way for a more flexible logical or software development. Once again, the
\kl{Curry-Howard correspondence} is at work, since we adapt concepts from the world of
programming languages to the logical one.

\section{And this Thesis?}
\label{sec:this-thesis}

My doctoral work itself is centred around bidirectional typing, under three main aspects,
corresponding to the three parts of this thesis.
They are preceded by \cref{chap:tech-intro}, which introduces the main technical notions
used in what follows.

\subsection{Theory of bidirectional typing}

The first part (\nameref{part:bidir}) proposes to – partially – fill the theoretical gap around
bidirectional typing for dependent types. More precisely, it contains a proof of equivalence
between the standard presentation of CIC in the literature, and a bidirectional one.
\Cref{chap:bidir-ccw} presents the main ideas in a relatively
simple setting, in order to ease the exposition. \Cref{chap:bidir-pcuic} shows how to extend
them to a more realistic setting, close to the type theory implemented in \kl{Coq}.
Finally, \cref{chap:bidir-conv} focuses on the particular status of conversion%
\sidenote{This crucial notion allows the integration into dependent type theory of
the notion of computation of programs.},
and the links between recent work on this subject and bidirectional typing.

\subsection{Bidirectional typing in \kl{MetaCoq}}

The second part of the thesis (\nameref{part:metacoq}) focuses on the \kl{MetaCoq} project,
and especially the formalization, in \kl{Coq}, of the ideas presented in the first part.
\Cref{chap:metacoq-general} gives a general overview of the project, while
\cref{chap:kernel-correctness} concentrates more specifically on the proof that the
\kl{kernel} implemented in \kl{MetaCoq} fulfils its specification.

\subsection{Gradual dependent types}

Finally, the third and last part (\nameref{part:gradual}) presents my work in the area
of \kl{gradual types}. Since dependent types already form complex systems, their adaptation
to the gradual approach is particularly delicate. A summary of the possibilities and issues is
presented in \cref{chap:gradual-dependent}. An interesting point of emphasis is that the
usual presentation of dependent types turns out to be unsuited, as it is too flexible.
The additional structure provided by bidirectional typing is key to solve this issue. It is also
relevant to present the type-directed elaboration of terms from a source language
to a target one, an important characteristic shared by all \kl[gradual types]{gradual languages}.
The use of a bidirectional elaboration, and the properties it allows us to obtain, are described
in \cref{chap:bidir-gradual-elab}. Finally, \cref{chap:beyond-gcic} describes follow-up work
complementing that of \cref{chap:bidir-gradual-elab}, but which is not directly linked to
bidirectional typing.

\subsection{Technical contributions}

My doctoral work started with the study of \kl(typ){gradual}
\kl(typ){dependent} types.
I contributed, together with Kenji Maillard, Nicolas Tabareau and Éric Tanter, to
\sidetextcite{LennonBertrand2022}, where we study a gradual extension to the
Calculus of Inductive Constructions. My main technical contribution corresponds
to \cref{chap:bidir-gradual-elab}. The precise literature review and the impossibility
theorem of \cref{chap:gradual-dependent} it leads to also comes from this
publication.
The second technical part of \textcite{LennonBertrand2022}, in which I participated but
whose main author is Kenji Maillard, as well as a second article,%
\sidenote{\textcite{Maillard2022}, currently under review.}%
\margincite{Maillard2022}
together with the same authors and again Kenji Maillard as main investigator,
correspond to \cref{chap:beyond-gcic}.

This work having shown the relevance of a bidirectional dependent type system and the relative
scarceness of results on the subject, I focused more closely on it, both on
paper and by means of a formalization based on \kl{MetaCoq}. This led to a second publication
\sidecite{LennonBertrand2021}, and corresponds to \cref{chap:bidir-ccw,chap:bidir-pcuic}
for the theoretical part, and \cref{sec:kernel-bidir} for the formalized proof
of equivalence between bidirectional and undirected typing.
The completeness bug in the kernel of \kl{Coq} found during this formalisation, together with
the impact of this discovery on the implementation of \kl{Coq} is presented in
\sidetextcite{Sozeau2022}.

I then turned to the closer integration of this formalization into \kl{MetaCoq}, and its use
in order to prove completeness of the \kl{kernel} it implements.%
\sidenote{A definition of a type-checking algorithm proven sound but not complete by
Simon Boulier was already present, although I had to alter it during the completeness
proof.}
This is described in \cref{sec:kernel-typing}.
I also contributed more generally to the project on various more minor points.
This part of my thesis work has not been published yet, but the other contributors to
\kl{MetaCoq} and I are currently working on it.

Finally, \cref{chap:bidir-conv} corresponds to a project I initiated in order to extend
\kl{MetaCoq} to integrate extensionality η rules to conversion,
but which did not reach the stage of publication yet. Yet, I presented the difficulties
that led me to it in \sidetextcite{LennonBertrand2022a}.

\chapter{The Calculus of Inductive Constructions}
\label{chap:tech-intro}

Most of this thesis revolves around \kl[dependent type]{dependent type systems}.
Since these are
quite complex, there is a high number of points where one can introduce slight variations
when giving a precise definition of a system to study it.
Some of these variations are unimportant, but some introduce large differences in the
resulting systems. Thus, in this chapter we go over in details over
the definition of what we refer to as the
\kl{Calculus of Inductive Constructions} (or simply \kl{CIC}), in the rest of this
thesis, and which serves as the basis for variations, theoretical study and extensions.
While doing so, I try to give an idea of the trade-offs involved, and of the reasons
behind the choices made. Quite a few of those choices vary during the thesis,
and this is by design: there is no single better choice, 
something that I try to make as clear as possible.

For the impatient specialists, let me say now that with \kl{CIC}, I
mean an intentional type theory, with Curry-style abstractions,
a predicative hierarchy of universes \textit{à la Russel}\sidenote{
  And only those: by default I do \emph{not} include an impredicative sort of propositions, a feature that is sometimes associated to the name \kl{CIC}.},
and any amount of required inductive types, presented by recursors – 
the ones appearing most often in what comes next being the empty and
unit types, booleans, natural numbers, dependent sums, lists, vectors and the equality.
Conversion is by default the reflexive, symmetric, transitive and congruent
closure of β-reduction, and
so in particular it is untyped. For the others, let me now detail what I mean by this.

\section{Terms, typing and derivations}

Throughout this chapter, type systems are defined by means of a relation
$\Gamma \vdash t : T$, which reads "in the context $\Gamma$, term $t$ has type $T$".
From the logical point of view, this judgement means that $\Gamma$
contains the hypothesis available to deduce the
result $T$ by means of the proof $t$. On the programming side, it means that
$t$ is a well-formed program,
which uses the variables listed in $\Gamma$ together with their types,
and has type $T$.
Thus, $\Gamma$ is a list of declarations of the form $x : A$. We write $\emptycon$ for the
empty context, and $\Gamma, x : A$ for the context $\Gamma$ extended with the new variable $x : A$, and $(x : A) \in \Gamma$ to denote that the declaration $x : A$ appears at some
point in the context $\Gamma$.

\begin{marginfigure}
  \begin{mathpar}
  \inferdef{Var}{(x : A) \in \Gamma \\ \vdash \Gamma}{\Gamma \vdash x : A}
  \label{rule:cic-var}
  \end{mathpar}
\end{marginfigure}

This typing relation itself is defined by means of inference rules,
such as \ruleref{rule:cic-var} opposite. The way to read this rule is that the judgement
underneath the line follows from the one above,
\ie from $(x : A) \in \Gamma$
and $\vdash \Gamma$ – a judgement that will soon be defined asserting that the context
$\Gamma$ is well-formed – we can deduce $\Gamma \vdash x : A$.
When objects appear in the hypothesis but not the conclusion, they are implicitely
universally quantified.
Once a set of such inference rules is fixed,
typing is defined as the least relation closed by those
rules. Equivalently, a judgement such as $\Gamma \vdash t : T$
holds as soon as we can build a tree whose nodes are instances of the inference rules,
and whose root is the judgement in question. A general setting
to make this kind of definitions precise can be found in \sidetextcite{Bauer2020},
in our case we restrict to this level of detail until \arefpart{metacoq}.

As we have already introduced variables, a word on those as well. Variables are difficult
to account for precisely, because of issues like shadowing – a conflict between two variables
with the same name – or $\alpha$-equivalence – the identification between two terms
only differing on variable names. There are multiple techniques to solve these issues
– see the many solutions to the POPLMark Challenge~\sidecite{Aydemir2005} –, 
but again, we treat these issues in an informal way until \arefpart{metacoq}, assuming
there is no shadowing whatsoever and identifying $\alpha$-equivalent terms when needed.

A final important building block that we use in all our type theories is substitution,
that we write $\subs{t}{x}{u}$. This replaces every occurrence of $x$ in $t$ by the term
$u$. Once again, we treat this operation informally, assuming it never creates
shadowing – what is sometimes called "capture-avoiding" substitution.

\section{Functional core: \kl{CCω}}
\label{sec:tech-ccw}

\subsection{Functions and applications}

Let us now turn to the core of CIC, namely the
\intro{Calculus of Constructions} (\kl{CCω}). Through the \kl{Curry-Howard correspondence},
it is both a typed form of λ-calculus – \ie a kind of purely functional
programming language – and a minimal form
of logic – only containing universal quantification and implication.

\begin{marginfigure}
  \ContinuedFloat*
  \begin{mathpar}
    \inferrule{\Gamma, x : A \vdash t : T}{\Gamma \vdash \l x : A .\ t : A \to T}
    \and
    \inferrule{\Gamma \vdash f : A \to T \\ \Gamma \vdash u : A }{ \Gamma \vdash t\ u : T}
  \end{mathpar}
  \caption{Typing for non-dependent functions}
  \label{fig:cic-nondep-fun}
\end{marginfigure}

Functions, also called λ-abstraction are written $\l x : A .\ t$. This corresponds
to the mathematical notation $x \mapsto t$: the body $t$ of the function,
is a term that might contain the variable $x$,
and the constructor λ abstracts over that variable to build a function.
Conversely, function application is written simply using juxtaposition, as in $t\ u$.
The type of functions is written $\to$, as in ordinary mathematics.
You can see those at work in \cref{fig:cic-nondep-fun}: an abstraction build a term of arrow
type, and application needs its function to be of arrow type, whose domain must correspond to
that of the argument for it to be well-typed.
Logically, those rules correspond to that of implication: if from a hypothesis $A$ one can
deduce $T$, then $A \to T$ holds – reading $\to$ as implication; conversely if $A \to T$
and $A$ both hold, then $T$ does as well.

\begin{marginfigure}
  \ContinuedFloat
  \begin{mathpar}
    \inferdef{Abs}{\Gamma, x : A \vdash t : T}{\Gamma \vdash \l x : A .\ t : \P x : A.\ T}
    \label{rule:cic-abs}
    \and
    \inferdef{App}{\Gamma \vdash f : \P x : A.\ T \\ \Gamma \vdash u : A }{ \Gamma \vdash f\ u : \subs{T}{x}{u}}
    \label{rule:cic-app}
  \end{mathpar}
  \caption{Typing for dependent functions}
  \label{fig:cic-dep-fun}
\end{marginfigure}
These arrow types, however, are not as expressive as one would wish for.
Remember that we are in the realms of dependent types, so not only $t$ might mention $x$,
but also $T$. For instance, $T$ might be something like "$x$ is even". In such a case,
we need to record that dependency, which is the point of $\P$-types – or product types –,
shown in \cref{fig:cic-dep-fun}.
Seen as function types, they record the fact that the codomain
might vary depending on the argument. This is reflected in the typing rule for application:
since the codomain $T$ might depend on $x$, the type of the application $f\ u$ is $T$
\emph{specialized at the argument $u$}, using substitution.
Seen on the logical side, product types correspond to universal quantification
$\operatorname{\forall} x : A.\ T(x)$.
Indeed, if one can show that $T(x)$ holds for an unspecified $x$,
then it must hold for all $x : A$ – this is \ruleref{rule:cic-abs}.
Conversely, if $T$ holds for all $x : A$, then one can deduce $T(u)$ for any specific
$u : A$ – this is \ruleref{rule:cic-app}.
Now the rules for arrow types in \cref{fig:cic-nondep-fun} are just a special case
of those for product types, in the case where the codomain $T$ does not depend
on the variable $x$, and we use this convention throughout the thesis:
$A \to T$ is a shortcut for $\P x : A.\ T$ when $T$ does not mention $x$.

One last thing to note about our functions is that they record the type of their
domain – what is called \intro{Church-style}
abstraction~\sidecite[][Section~3]{Barendregt1992}. There is an alternative – 
the \intro{Curry-style} abstractions –, that
does not do so, simply using $\l x.\ t$ for functions. At this point in the
presentation, this does not make much difference,
but it is crucial as soon as one looks at the bidirectional structure. 
Indeed, that annotation is required if one wants to infer types for functions,
rather than barely checking them.
The \kl{Curry-style}
option is definitely sensible, see for instance \sidetextcite[][p.~19]{Norell2007}
– which describes the implementation of the very successful proof assistant
\kl{Agda}, which uses the \kl{Curry-style} approach –,
\sidetextcite[][Section~4.1]{Gratzer2019} or \sidetextcite{McBride2022}.
In the end, this is really a design choice between being able to infer a type for any term,
or requiring annotations that in a lot of cases are useless, and in this
thesis we stick with the approach used in \kl{Coq}, and annotate our abstractions.

\subsection{Universes}

To be able to express things like induction principles or polymorphic functions, it is
extremely useful to be able to use functions and products quantifying over types.
This is what the universe $\uni$ are for. It is the type… of a type.
This means that the frontier between types and terms is not syntactic any more.
Instead, types are simply terms of type $\uni$.
Despite this, we still use upper case letters for terms which we want to think of as types.
Such a universe is called \textit{à la} Russell~\sidecite{Palmgren1998}, by contrast with
universes \textit{à la} Tarski, which regain the distinction between types and terms at
the cost of a somewhat heavier treatment of types.
The presentation \textit{à la} Tarski
is mostly useful when building models. Since this is not the subject at hand in this thesis,
we keep the simpler of the two presentations.

\begin{marginfigure}
  \ContinuedFloat
  \begin{mathpar}
    \inferdef{Univ}
    {\vdash \Gamma}
    {\Gamma \vdash \uni[i] : \uni[\unext{i}]}
    \label{rule:cic-univ}
  \end{mathpar}
  \caption{Typing for universes}
  \label{fig:cic-univ}
\end{marginfigure}

There is however, an important caveat.
Since the disappointments of Frege and the paradox exhibited
by Russell in his system, logicians know that considering a set of all sets is a great
source of inconsistencies. Type theory is not devoid of this issue:
Girard~\sidecite[][Annex~A]{Girard1972}
shows how having a type with itself as type is inconsistent.
This inconsistency directly applies to the first dependent type system proposed by
Martin-Löf~\sidecite{MartinLoef1972}, which had a single universe $\uni$ and a rule $\uni : \uni$.
A common solution to this
is to stratify universes into an infinite hierarchy, which gives us \ruleref{rule:cic-univ}:
note the \intro{universe levels} $i$ and $\unext{i}$.

\begin{marginfigure}
  \ContinuedFloat
  \begin{mathpar}
    \inferdef{Prod}
    {\Gamma \vdash A : \uni[i] \\ \Gamma, x : A \vdash B : \uni[j]}
    {\Gamma \vdash \P x : A.\ B : \uni[\umax{i}{j}]}
    \label{rule:cic-prod}
  \end{mathpar}
  \caption{Typing for product types}
  \label{fig:cic-prod}
\end{marginfigure}

\begin{marginfigure}
  \ContinuedFloat
  \begin{mathpar}
    \inferdef{EmptyCon}
    { }{\vdash \cdot}
    \label{rule:cic-empty-con} \and
    \inferdef{ConsCon}
    {\vdash \Gamma \\ \Gamma \vdash A : \uni}{\vdash \Gamma, x : A}
    \label{rule:cic-cons-con}
  \end{mathpar}
  \caption{Context well-formedness}
  \label{fig:cic-con}
\end{marginfigure}

Using those universes, \ruleref{rule:cic-prod} gives the typing rule for the product constructor. We can also use these universes to give a definition of the $\vdash \Gamma$
judgement, asserting that a context is well-formed, in \cref{fig:cic-con}.
It simply means that all its types
are indeed types. Note that in \ruleref{rule:cic-cons-con}, we did not give an index for the
universe, we do so to mean the existence of some unconstrained level $i$.

One last important point regarding universes is the kind of levels used. The simplest solution
is to rely on natural number (of the meta-theory), with the $\unextsymb$
and $\umaxsymb$ operations interpreted by the usual ones.
This is however not strictly necessary: we need levels
to form an order so as to ensure we avoid inconsistency, and operations $\unextsymb$ and
$\umaxsymb$, but levels could very well be something different from natural numbers.
In particular, the natural number approach fixes at which level a particular construction
is done, which is usually much more rigid than what one would wish for.
A more flexible approach, introduced under the name \intro{typical ambiguity} by
\sidetextcite{Harper1991},
uses level expressions based on level variables, rather than integers.
This way, one can collect the constraints between levels required for a
term to type-check in a consistent system, without artificially enforcing a
rigid interpretation of levels by fixing them to a precise integer once and for all.
To simplify the presentation, our "standard" \kl{CCω}/\kl{CIC} nonetheless uses integers,
but we switch to level expressions when needed.

\section{50 Shades of Conversion}
\label{sec:tech-conversion}

\begin{marginfigure}
  \ContinuedFloat
  \begin{mathpar}
  \inferdef{Conv}
    {\Gamma \vdash t : T \\ \Gamma \vdash T \conv T' : \uni}
    {\Gamma \vdash t : T'}
  \label{rule:cic-conv}
  \end{mathpar}
  \caption{Conversion rule}
\end{marginfigure}

There is one big missing part in the picture so far. Remember we are working with
dependent types, and that those can contain terms, which in turn can be arbitrary programs.
If you recall for instance the vector type we used in the introduction (and that we are
about to introduce formally), what happens if a function expects an argument of type
$\Vect A\,3$, but it is given an argument of type $\Vect A\,(2+1)$, for instance the output
of a concatenation function? Surely we must have a way to relate both, since after all
the small program $2+1$ ought to compute to $3$! This is exactly what \intro{conversion} is
for: in dependent type theory, there is a way to change a type to one that
is related to it by this relation – this is \ruleref{rule:cic-conv}, which wraps up our typing
rules for \kl{CCω}, collected in \cref{fig:ccw-typing}.
As usual, there are two ways to look at this relation. From the point of view of programs,
it allows to incorporate the computational aspect of those, directly inside the type system.
From the point of view of logics, this corresponds to things being the same "by definition"
rather than due to some reasoning
– which is why conversion is also called definitional, or judgmental equality. In our vector
example, for instance, the two types are the same by virtue of the definition of addition.

\begin{figure*}[h]
  \LastFloat

  \begin{mathpar}
    %
    \jform{\vdash \Gamma}
    \inferdef{EmptyCon}
      { }{\vdash \cdot}
    \and
    \inferdef{ConsCon}
      {\vdash \Gamma \\ \Gamma \vdash A : \uni}{\vdash \Gamma, x : A}
    \\\\
    \jform{\Gamma \vdash t : T}
    \inferdef{Var}{(x : A) \in \Gamma \\ \vdash \Gamma}{\Gamma \vdash x : A}
    \and
    \inferdef{Univ}
      {\vdash \Gamma}
      {\Gamma \vdash \uni[i] : \uni[\unext{i}]}
    \and
    \inferdef{Prod}
      {\Gamma \vdash A : \uni[i] \\ \Gamma, x : A \vdash B : \uni[j]}
      {\Gamma \vdash \P x : A.\ B : \uni[\umax{i}{j}]}
    \and
    \inferdef{Abs}{\Gamma, x : A \vdash t : T}{\Gamma \vdash \l x : A .\ t : \P x : A.\ T}
    \and
    \inferdef{App}
      {\Gamma \vdash f : \P x : A.\ T \\ \Gamma \vdash u : A }
      {\Gamma \vdash f\ u : \subs{T}{x}{u}}
    \and
  \inferdef{Conv}
    {\Gamma \vdash t : T \\ \Gamma \vdash T \conv T' : \uni}
    {\Gamma \vdash t : T'}
  \end{mathpar}

  \caption{Collected typing rules for \kl{CCω}}
  \label{fig:ccw-typing}
\end{figure*}

Conversion is a very complex beast, arguably the most subtle part of dependent types.
Consequently, there are very different ways to present it, which in turn serve different
needs.
For this reason, we took care to setup the typing rules of
\cref{fig:ccw-typing} so that \emph{nothing} has to
be changed in those when one definition of conversion or another is taken. The only
difference is in how the relation $\Gamma \vdash T \conv T' : \uni$ is defined.
This way, we can treat it as a black box when talking about typing,
making the design modular.

A first divide is between \intro[typed conversion]{typed} and
\intro[untyped conversion]{untyped} conversion.
On one side, conversion is intrinsically typed, terms are only convertible
\emph{at a given type}. On the other, conversion is a relation between raw terms,
that does not presuppose any form of typing. \Cref{fig:typed-untyped-conv} gives an
example of the "same" rule – the computation rule for functions – in both systems.
As we can see, the content of the two rules is the same, it equates $(\l x : A.\ t)\ u$
and $\subs{t}{x}{u}$, only the side-conditions differ wildly.
Typed conversion goes back to
\sidetextcite{MartinLoef1972}, and is a recurring feature in its many descendants.
Untyped conversion relates strongly to (untyped) λ-calculus – Barendregt
for instance uses the name "conversion" for the equational theory of untyped λ-calculus
in his reference work on the subject~\sidecite{Barendregt1985} –, via
the Pure Type Systems (PTS)~\sidecite{Barendregt1991} literature.
In this thesis, we mainly consider untyped conversion, as \kl{Coq}’s meta-theory
has been mostly studied in that tradition.
But the relation between both in the context of
bidirectional typing is the main subject of \cref{chap:bidir-conv}.

\begin{figure}[h]
  \begin{mathpar}
    \inferrule
      {\Gamma, x : A \vdash t : B \\ \Gamma \vdash u : A}
      {\Gamma \vdash (\l x : A.\ t)\ u \tdconv \subs{t}{x}{u} : \subs{B}{x}{u}}
    \and
    \inferrule{ }{(\l x : A.\ t)\ u \udconv \subs{t}{x}{u}}
  \end{mathpar}
  \caption{Example: typed and untyped β rule for conversion}
  \label{fig:typed-untyped-conv}
\end{figure}

A second axis is about how close the conversion relation is to an implementation.
For instance, conversion should be an equivalence relation,
but there are two approaches to that. The first – and standard – one
is to simply \emph{define} conversion as an equivalence relation, by adding rules 
for \eg transitivity, as the one of \cref{fig:trans-conv}.
\begin{marginfigure}
  \begin{mathpar}
    \inferrule
      {t \udconv t' \\ t' \udconv t''}
      {t \udconv t''}
  \end{mathpar}
  \caption{Example: transitivity rule for conversion}
  \label{fig:trans-conv}
\end{marginfigure}
This ensures that conversion has the right properties, but means it does not directly correspond
to an algorithm: the transitivity rule cannot be directly implemented, since its middle
term is not recorded in any place.
The λ-calculus theorists have known this issue for a long time, and they
have a solution: characterizing conversion by means of a \kl{reduction} relation $\red$, which
corresponds to the idea of program evaluation – see \sidetextcite{Barendregt1985} for
instance. If this reduction has good
properties\sidenote{The main one being confluence.}, then
two terms are be convertible exactly when they reduce to the same third term.
This much more operational characterization is closer to what can be implemented.
Turning things around, one can define conversion through reduction,
and only show \emph{afterwards}
that it has the good properties of the first approach – typically, that it is transitive.
Conversion of the first kind we call \intro{declarative conversion}, while for the second
we talk about \intro{algorithmic conversion}.

In the rest of this section we give two presentations of \kl{untyped conversion}.
A \kl{declarative} one, which we use to define \kl{CCω}, as is the standard.
And an \kl{algorithmic} one, anticipating \arefpart{metacoq} where it is needed
to show decidability of type-checking, and
\arefpart{gradual} where we extend it into a relation that is by design not transitive, so
that basing it on declarative conversion would be nonsensical.

\subsection{Untyped declarative conversion}

\begin{marginfigure}
  \ContinuedFloat*
  \begin{mathpar}
    \inferdef{UConv}{\Gamma \vdash T' : \uni \\ T \udconv T'}{\Gamma \vdash T \conv T' : \uni}
    \label{rule:cic-conv-unty}
  \end{mathpar}
  \caption{Typing constraint on untyped conversion}
\end{marginfigure}

To start our presentation of \kl{untyped conversion},
let us first go back to \ruleref{rule:cic-conv}.
Even if we wish to describe conversion as
an untyped relation, we still enforce a typing constraint in \ruleref{rule:cic-conv},
in order to ensure that, whenever $\Gamma \vdash t : T$ is derivable,
$\Gamma \vdash T : \uni$ is as well.
This is exactly the content of \ruleref{rule:cic-conv-unty}, which combines conversion
with a check that the target type is indeed a well-formed type.

\begin{figure}[h]
  \ContinuedFloat
  \begin{mathpar}
    \inferdef{ConvRefl}{ }{t \udconv t}
    \label{rule:cic-uconv-refl} \and
    \inferdef{ConvSym}{t \udconv t'}{t' \udconv t}
    \label{rule:cic-uconv-sym} \and
    \inferdef{ConvTrans}
      {t \udconv t' \\ t' \udconv t''}
      {t \udconv t''}
    \label{rule:cic-uconv-trans}
  \end{mathpar}
  \caption{Equivalence rules}
  \label{fig:cic-uconv-equiv}
\end{figure}

Regarding conversion itself, the first set of rules asserts
that conversion forms an equivalence relation: it
is reflexive (\ruleref{rule:cic-uconv-refl}), symmetric (\ruleref{rule:cic-uconv-sym}),
and transitive (\ruleref{rule:cic-uconv-trans}).

A second set of rules, collected in \cref{fig:cic-uconv-cong},
asserts that conversion is a congruence, meaning that it is compatible
with all term formers. As for the previous three, these correspond to properties we expect
from the conversion relation, that we simply declare to be true. Note that we include only
congruence rules for term formers with sub-terms – we \eg omit $\uni$. This is because
those are special cases of \ruleref{rule:cic-uconv-refl}. Conversely, we could omit 
reflexivity altogether and only have congruence rules, which can be seen as a generalized
form of reflexivity.

\begin{figure}[h]
  \ContinuedFloat
  \begin{mathpar}
    \inferrule
    % \inferdef{ProdConv}
      {A \udconv A' \\ B \udconv B'}
      {\P x : A.\ B \udconv \P x : A'.\ B'}
    % \label{rule:cic-uconv-prod}
    \and
    \inferrule
    % \inferdef {AbsConv}
      {A \udconv A' \\ t \udconv t'}
      {\l x : A .\ t \udconv \l x : A'.\ t'}
    % \label{rule:cic-uconv-abs}
    \and
    % \inferdef{AppConv}
    \inferrule
      {f \udconv f' \\ u \udconv u' }
      {f\ u \udconv f'\ u'}
    % \label{rule:cic-uconv-app}
  \end{mathpar}
  \caption{Congruence rules}
  \label{fig:cic-uconv-cong}
\end{figure}

\begin{marginfigure}
  \ContinuedFloat
  \begin{mathpar}
    \inferdef{βConv}{ }{(\l x : A.\ t)\ u \udconv \subs{t}{x}{u}}
    \label{rule:cic-uconv-beta}
  \end{mathpar}
  \caption{Computation rule for functions}
\end{marginfigure}
Finally \ruleref{rule:cic-uconv-beta}, is the crucial one:
it corresponds to the computational behaviour
of functions, replacing the variable of an applied λ-abstraction by its argument by
means of substitution.

\subsection{Reduction}

Before we can describe \kl{algorithmic conversion}, we first need
to give a look at \intro{reduction}. Reduction is in some way an operational variant of
conversion. The main difference is that it is oriented, in the direction which would
correspond to program evaluation. It itself decomposes into three components.

\begin{marginfigure}
  \ContinuedFloat*
  \begin{mathpar}
    \inferdef{βRed}{ }{(\l x : A.\ t)\ u \tred \subs{t}{x}{u}}
    \label{rule:beta-red}
  \end{mathpar}
  \caption{Top-level reduction}
\end{marginfigure}

The first is \intro{top-level
reduction} $\tred$, which corresponds purely to computation, without any congruent closure.
In \kl{CCω} there is only the single \ruleref{rule:beta-red}.

The second component is the congruent closure of top-level reduction,
\intro{one-step reduction} $\ored$, which allows triggering top-level reduction exactly once,
but at any position in a term. Its definition is given in \cref{fig:ccw-ored}.
Note that while we talk about congruent closure both for
\kl{conversion} and \kl{one-step reduction}, we mean it differently: in the
case of conversion, we demand the relation to hold in all sub-terms,
while for one-step reduction it is allowed in exactly one position.

\begin{figure}[ht]
  \ContinuedFloat
  \begin{mathpar}
    \inferrule
    % \inferdef{TopRed}
      {t \tred t'}
      {t \ored t'}
    % \label{rule:top-red}
    \and
    \inferrule
    % \inferdef{ProdRedDom}
      {A \ored A'}
      {\P x : A.\ B \ored \P x : A'.\ B}
    % \label{rule:red-prod-dom}
    \and
    \inferrule
    % \inferdef{ProdRedCod}
      {B \ored B'}
      {\P x : A.\ B \ored \P x : A.\ B'}
    % \label{rule:red-prod-cod}
    \and
    \inferrule
    % \inferdef{AbsRedDom}
      {A \ored A'}
      {\l x : A .\ t \ored \l x : A'.\ t}
    % \label{rule:red-abs-dom}
    \and
    \inferrule
    % \inferdef{AbsRedBod}
      {t \ored t'}
      {\l x : A .\ t \ored \l x : A.\ t'}
    % \label{rule:red-abs-bod}
    \and
    \inferrule
    % \inferdef{AppRedFun}
      {f \ored f'}
      {f\ u \ored f'\ u}
    % \label{rule:red-app-fun}
    \and
    \inferrule
    % \inferdef{AppRedArg}
      {u \ored u'}
      {f\ u \ored f\ u'}
    % \label{rule:red-app-arg}
  \end{mathpar}
  \caption{One-step reduction}
  \label{fig:ccw-ored}
\end{figure}

Finally, we obtain \kl{reduction} as the reflexive
transitive closure of one-step reduction, see \cref{fig:red}.

\begin{figure}[h]
  \begin{mathpar}
    \inferrule
    % \inferdef{TopRed}
      {t \tred t'}
      {t \ored t'} \and
    % \label{rule:top-red}
    \inferrule{ }{t \fred t}
    \and
    \inferrule
      {t \ored t' \\ t' \fred t''}
      {t \fred t''}
  \end{mathpar}
  \caption{Reduction}
  \label{fig:red}
\end{figure}

Now, the reason we separate these three layers is that reduction as just defined is
somewhat too unrestricted, in particular it is non-deterministic.
In a lot of places, what we care about is exposing the head
constructor of a term, and there is a deterministic strategy we can impose to do so, called
\intro{weak-head reduction} $\hred$. It amounts to restricting the places in the term where
\kl{top-level reduction} can be used, by removing some congruence rules. More precisely,
λ-abstractions, product types and universes are not reduced, as they already are canonical
forms of their types. Variables are not reduced either, since they simply cannot be. Thus,
the only reduction that is allowed is in the function position of an application, with the
hope to get an abstraction there that can be further reduced at top-level. In the end,
we get \cref{fig:wh-red}. When we want to contrast this weak-head reduction with the
previously defined one $\red$, we call the latter \kl{full reduction}.
\begin{figure}[h]
  \begin{mathpar}
    \inferrule
    % \inferdef{TopRed}
      {t \tred t'}
      {t \hored t'}
    % \label{rule:top-red}
    \and
    \inferrule
      {f \hored f'}
      {f\ u \hored f'\ u}
    \and
    \inferrule{ }{t \hred t}
    \label{rule:red-refl} \and
    \inferrule
      {t \hored t' \\ t' \hred t''}
      {t \hred t''}
    \label{rule:red-trans}
  \end{mathpar}
  \caption{Weak-head redution}
  \label{fig:wh-red}
\end{figure}



\subsection{Algorithmic conversion}

\section{Adding Inductive Types: \kl{CIC}}

\subsection{Datatypes}

\subsection{Indexed inductive types}

\section{Beyond \kl{CIC}: \kl{PCUIC}}

\subsection{Cumulativity}

\subsection{The sort of propositions}

\subsection{Local and global definitions}

\subsection{Enhancing inductive types}

\subsection{Co-inductive types and records}

\pagelayout{wide} % No margins
\addpart{Bidirectional typing for dependent types}
\label{part:bidir}
\pagelayout{margin} % Restore margins


When presenting a typing derivation the way we did in \cref{chap:tech-intro}, there is
an important piece of information missing.
In logical programming, this is called the \kl{mode} of the inference rules,
\ie which objects are considered as inputs and which as outputs in the search
for a derivation.
This information, however, is crucial when one tries to build a type-checker:
some rules might seem fine when writing them down on paper, but trying to give them a
sensible mode fails, indicating they are not suited for an implementation.
In the case of the typing judgement $\Gamma \vdash t \ty T$,
usually both the term $t$ under inspection and the context $\Gamma$ are inputs –
although some depart from this \sidetextcite{Jim1996}.
The mode of the type $T$, however, is much less clear: should it be inferred based upon
$\Gamma$ and $t$, or do we merely want to check whether $t$ conforms to a given $T$?
Both are sensible questions, and in fact typing algorithms for complex type systems usually 
alternate between them during the inspection of a single term/program.
\AP The bidirectional approach makes this difference between modes explicit,
by decomposing \intro{undirected typing}%
\sidenote{We call anything related to the $\Gamma \vdash t \ty T$ judgement
\kl(typ){undirected}, by contrast with bidirectional typing.}
$\Gamma \vdash t \ty T$ into two separate judgements $\inferty{\Gamma}{t}{T}$
(\kl{inference}) and $\checkty{\Gamma}{t}{T}$ (\kl{checking})%
\sidenote{We use the $\ity$ and $\cty$ symbols rather than the more usual
$\Rightarrow$ and $\Leftarrow$ to avoid confusion with implication and with the
\kl{Coq} notation for functions.},
that differ only by their \kl{modes}. The type is an
\kl{input} in inference, but an \kl{output} in checking.
Following this decomposition%
\sidenote{Pioneered by \textcite{Pierce2000}, a general survey can be found
in \textcite{Dunfield2021}.}%
\margincite{Pierce2000}%
\margincite{Dunfield2021}
leads to type systems that are more structured and directly amenable to implementations,
and to good quality algorithms.%
\sidenote{\textcite{Pierce2000} for instance stress good error reporting as an important 
property of their approach.}

This is appealing, but in the dependently typed world, despite advocacy by \eg
\sidetextcite{McBride2018,McBride2019} to adopt this approach during
the design of type systems on paper rather than in their implementations only,
most of the theoretical work to this day remains undirected.
Bidirectionality appears mostly
in articles concentrating on the description of typing algorithms, for instance
\sidetextcite{Huet1989}, \sidetextcite{Coquand1996}, or \sidetextcite{Norell2007}.
However, since these primarily insist on the algorithmic aspect, they do not consider the
bidirectional structure for itself. Moreover, in the case of
\textcite{Coquand1996} and \textcite{Norell2007}, they concentrate on bidirectional typing
as a way to remedy for the lack of annotations on their \kl{Curry-style} λ-abstractions.
This is sensible when looking for lightness of the input syntax, but poses an inherent completeness problem: a term such as $(\l x . x)~\z$ is not typeable in those systems.%
\sidenote{They are actually only able to give types to normal forms.}
In the context of \kl{Church-style} abstraction, the closest there is to a description of
bidirectional typing for \kl{CIC} is probably the one given by the
\kl{Matita} team \sidecite{Asperti2012},
which however concentrates again on the challenges posed by the
elaboration and unification algorithms.
They also do not consider the problem of completeness with respect to a given undirected system, as it would fail in their setting due to the undecidability of higher order unification.

In this part (\nameref{part:bidir}), we wish to fill this gap in the literature,
by describing a bidirectional type system that is complete with respect to (undirected)
\kl{CIC}, as presented in \cref{chap:tech-intro}.
By completeness, we mean that any term that is typeable in the undirected system should also
infer a type in the bidirectional one.
This feature is very desirable when implementing kernels for proof assistants,
whose algorithms should correspond to their undirected specification –
even on terms not in normal form. Indeed, reduction is only normalizing
on well-typed terms, so it should not be called on a term that is not known to be
well-typed. Thus if a developer wishes to generate a term using tactics, they cannot use
reduction before knowing that it is well-typed, but might not be able to type-check it
because it is not a normal form… And ensuring that a tactic returns normal forms only
might be unfeasable, and should not be a concern.

The bidirectional system we present naturally forms an intermediate
step between actual algorithms and undirected type systems, something we exploit
in \arefpart{metacoq}.
But its interest is not limited to the relation with implementations.
Indeed, the structure of a bidirectional derivation is more constrained than that of
an undirected one, especially controlling the usage of computation – \eg the conversion rule
\refname{rule:cic-conv}.
This finer structure can make proofs easier,
while the equivalence ensures that they can be transported to the undirected world.
We show this by providing straightforward proofs of \kl{uniqueness of types up to}
\kl{cumulativity}, and of \kl{strengthening}.

% The bidirectional structure also provides a better base for new type systems. This was actually the starting point for this investigation: in \cite{LennonBertrand2020}, we quickly describe a bidirectional variant of CIC, as the usual undirected CIC is unfit for the gradual extension we envision due to the too high flexibility of a free-standing conversion rule. This is the system we wish to thoroughly describe and investigate here.
 
% This step has proven useful in an ongoing completeness proof of MetaCoq's \cite{Sozeau2020a} type-checking algorithm\footnote{A completeness bug in that algorithm – also present in the Coq kernel – has already been found, see \cref{sec:to-pcuic} for details.}: rather than proving the algorithm complete directly, the idea is to prove it equivalent to the bidirectional type system, separating the implementation problems from the ones regarding the bidirectional structure.

As we did in \cref{chap:tech-intro}, we start by exposing the main ideas in the
simpler setting of \kl{CCω}, in \cref{chap:bidir-ccw}.
With those set clear, we go on with their adaptation to \kl{PCUIC}, and the subtle
issues that arose in that context, in \cref{chap:bidir-pcuic}.
Finally, \cref{chap:bidir-conv} describes early investigations into giving a
bidirectional treatment not only of typing, but also of conversion.

% This equivalence has been formalised on top of MetaCoq \cite{Sozeau2020}\footnote{A version frozen as described in this article is available in the following git branch: \url{https://github.com/MevenBertrand/metacoq/tree/itp-artefact}.}
% We next turn back to less technical considerations, as we believe that the bidirectional structure is of general theoretical interest. \Cref{sec:beyond} thus describes the value of basing type systems on the bidirectional system directly rather than on the equivalent undirected one. Finally \cref{sec:related} investigates related work, and in particular tries and identify the implicit presence of constrained inference in various earlier articles, showing how making it explicit clarifies those.
\chapter{Warm-up: CCω}
\label{chap:bidir-ccw}

\margintoc

% In this chapter, we give an alternate presentation of \kl{CCω}, as defined in
% \cref{fig:ccw-typing}. Most of the ideas are abstract over the notion of \kl{conversion}
% that is considered,
% so either \kl(conv){declarative} or \kl(conv){algorithmic} conversion can be used without
% much impact.

\section{Turning \kl{CCω} Bidirectional}
\label{sec:bidir-ccw}

\subsection{McBride’s discipline}

To design our bidirectional type system, we follow a discipline exposed by McBride
\sidecite{McBride2018,McBride2019}.
The central point is to distinguish in a judgement between the \intro{subject},
whose well-formedness is under scrutiny, \intro{inputs},
whose well-formedness is a condition for it to be meaningful,
and \intro{outputs}, whose well-formedness is a consequence of it.
We use the word \intro{well-formed} in a generic way for contexts, terms and types,
it stands for:
\begin{itemize}
  \item $\vdash \Gamma$ in the case of a context $\Gamma$,
  \item $\Gamma \vdash T : \uni$ in the case of a type $T$,
  \item the existence of some $T$ such that $\Gamma \vdash t \ty T$ in the case of a term $t$.
\end{itemize}
We also use \intro{well-typed} for a term, with the same meaning as \intro{well-formed}.

For instance, in the case of inference $\inferty{\Gamma}{t}{T}$, the subject is $t$,
$\Gamma$ is an input and $T$ is an output.
This means that one should consider whether $\inferty{\Gamma}{t}{T}$
only in cases where $\vdash \Gamma$ is already known,
and if the judgement is derivable it should be possible to conclude that
not only $t$, but also $T$ are well-formed.

In order to enforce this property globally, all inference rules should locally
preserve it as an invariant.%
\sidenote{The motto – slightly adapted from \textcite{McBride2018} – is:
  \textit{A rule is a server for its conclusion and a client for its premises.
  Servers receive promises about inputs and make promises about outputs, clients make
  promises about inputs and receive promises about outputs.}}
More precisely, information flows in a clockwise manner. First, one can assume that inputs
to the conclusion are well-formed – they are inputs to the rule. Next, we move to the
premises. Here the constraint is reversed: we should ensure that inputs to the premises are
well-formed, but can assume that their outputs and subjects are. In particular,
the well-formedness of inputs to a premise can rely on that of subjects or outputs
of previous ones.
Finally, information goes to the conclusion again, and now not only the subject but also
the output should be well-formed if all those of the premises are.

This distinction also applies to computation-related judgements, although those have no subject – instead, what is under scrutiny is the computational context of the rule.
For conversion $\Gamma \vdash T \conv T' : \uni$, 
both $T$ and $T'$ are inputs and thus should be known to be well-formed beforehand.
For reduction $T \red T'$, on the contrary, $T$ is an input,
but $T'$ is an output. Hence, only $T$ needs to be well-formed \textit{a priori},
and we rely on the \kl{subject reduction} property to ensure
that the output $T'$ also is.

\subsection{Constrained inference}

Beyond the already mentioned inference and checking judgements,
we need to introduce a third one: \intro{constrained inference}, written
$\pinferty{h}{\Gamma}{t}{T}$, where $h$ is either $\Pi$ or $\uni$.%
\sidenote{This is because these are the only type formers in \kl{CCω}.
In \kl{PCUIC}, $h$ can also be \eg and inductive type.}
Constrained inference is a judgement – or, rather, a family of judgements indexed by $h$ –
with the exact same modes as inference, but where the type output is not completely free.
Rather, as the name suggests, a constraint is imposed on it, namely that its head constructor can only be the corresponding element of $h$.
This is needed to handle the behaviour absent in simple types that some terms might not have a desired type “on the nose”. Take for instance the first premise
$\Gamma \vdash t \ty \P x : A .\ B$ of \ruleref{rule:cic-app}.
What bidirectional judgement should replace it?
It would be too much to ask $t$ to directly infer a $\Pi$-type, as some reduction might be needed to uncover this $\Pi$. Checking also cannot be used, because the domain and codomain of the tentative $\Pi$-type are not known at that point: they are to be inferred from $t$.

\subsection{Structural rules}

To transform the rules of \kl{CCω} as given in \cref{fig:ccw-typing}, we start by recalling that we wish to obtain a complete bidirectional type system.
Therefore, any term should infer a type, and thus all structural rules –
\ie all rules where the subject of the conclusion starts with a term former –
should give rise to an inference rule.
It thus remains to choose the judgements for the premises, which amounts to determining their modes.
If a term in a premise appears as input in the conclusion or output of a previous premise, then it can be considered an input, otherwise it must be an output. Moreover, if a type output is unconstrained, then inference can be used, otherwise we must resort to constrained inference.

\begin{figure}[ht]
  \ContinuedFloat*
  \begin{mathpar}
    \inferdef{Var}
      {(x : T) \in \Gamma}
      {\inferty{\Gamma}{x}{T}}
    \label{rule:bd-var} \and
    \inferdef{Univ}
      { }
      {\inferty{\Gamma}{\uni[i]}{\uni[i+1]}}
    \label{rule:bd-univ} \and
    \inferdef{ΠTy}
      {\pinferty{\uni}{\Gamma}{A}{\uni[i]} \\
        \pinferty{\uni}{\Gamma, x : A}{B}{\uni[j]}}
      {\inferty{\Gamma}{\P x : A .\ B}{\uni[\umax{i}{j}]}}
    \label{rule:bd-prod} \and 
    \inferdef{Abs}
      {\pinferty{\uni}{\Gamma}{A}{\uni[i]} \\ \inferty{\Gamma, x : A}{t}{B}}
      {\inferty{\Gamma}{\l x : A .\ t}{\P x : A .\ B}}
    \label{rule:bd-abs} \and
    \inferdef{App}
      {\pinferty{\Pi}{\Gamma}{t}{\P x : A .\ B} \\ \checkty{\Gamma}{u}{A}}
      {\inferty{\Gamma}{t\ u}{\subs{B}{x}{u}}}
    \label{rule:bd-app} \\
  \end{mathpar}
  \caption{Rules for inference in bidirectional \kl{CCω}}
  \label{fig:ccw-bidir-infer}
\end{figure}

In anticipation, we set the typing rules for \kl{CCω} so that this transformation would be
directly applicable. This particularly applies to the undirected \ruleref{rule:cic-abs},
recalled opposite.
\marginnote{
  \normalsize
  \begin{mathpar}
  \inferrule*[vcenter,right=Abs]
  {\Gamma \vdash A \ty \uni \\ \Gamma, x : A \vdash t \ty B}
  {\Gamma \vdash \l x : A.\ t \ty \P x : A.\ B}
  \end{mathpar}
}
Indeed, there are at least two other ways to write this rule, which do not lead to a valid
bidirectional presentation.
The first, which is the usual one in \kl{PTS},
is to have $\Gamma \vdash \P x : A.\ B \ty \uni$ instead of simply $\Gamma \vdash A \ty \uni$.
In the setting of a general \kl{PTS}, this is needed, because not every Π-type is well-formed,
even if the domain and codomain are.%
\sidenote{\kl{PTS} where this is true are called \intro{full}.}
However, this premise is problematic in the bidirectional setting. Indeed, $B$ can only be
inferred as a type for the body of the abstraction $t$. But to infer a type for $t$, the
context $\Gamma, x : A$ needs to be well-formed, which is not know if this premise is
the first one.
This issue has been identified by \sidetextcite{Pollack1992}, who remarks that the
bidirectional structure we present here only is equivalent to the undirected one
in semi-full \kl{PTS} – a slight generalization of the full ones.
In a full \kl{PTS}, the opposite approach of simply removing the first premise altogether
can also be taken, relying on \kl{validity} to ensure that $\vdash \Gamma, x \ty A$ and thus
$\Gamma \vdash A : \uni$. But again, in a bidirectional setting,
this does not respect McBride’s discipline.

The main difference between the bidirectional and undirected rules is that we dropped
hypotheses of context well-formedness in Rules~\nameref{rule:bd-univ} and
\nameref{rule:bd-var}. Indeed, as the context is always supposed to be well-formed
as an input to the conclusion, it is not useful to re-check it. This is also in line with implementations, where the context is not re-checked at leaves of a derivation tree, with performance issues in mind. The well-formedness invariants then ensure that any derivation starting with the (well-formed) empty context will only ever encounter well-formed contexts.

\subsection{Computation rules}

\begin{marginfigure}
\ContinuedFloat
\begin{mathpar}
  \inferdef{Check}
    {\inferty{\Gamma}{t}{T'} \\ T' \conv T}
    {\checkty{\Gamma}{t}{T}}
  \label{rule:bd-check} \and
  \inferdef{Univ-Inf}
    {\inferty{\Gamma}{t}{T} \\ T \fred \uni[i]}
    {\pinferty{\uni}{\Gamma}{t}{\uni[i]}}
    \label{rule:bd-pinf-univ} \and
  \inferdef{Π-Inf}
    {\inferty{\Gamma}{t}{T} \\ T \fred \P x : A. B}
    {\pinferty{\P}{\Gamma}{t}{\P x : A . B}}
    \label{rule:bd-pinf-prod}
\end{mathpar}
\caption{Computation rules for bidirectional \kl{CCω}}
\end{marginfigure}

We are left with the only non-structural rule, \ruleref{rule:cic-conv}.
As we observed, there are two possible modes for premises of inference rules where some
constraint is present, leading to the use of either checking or constrained inference.
In turn, this leads to two different modes for computation.
If the target type is an input, it can be compared to the inferred one using \kl{conversion}.
But if it is unknown (constrained inference) we must resort to \kl{reduction},
and obtain it from the inferred one.
This eventually leads to the decomposition of \ruleref{rule:cic-conv} into
\ruleref{rule:bd-check} in the first case, while
\nameref{rule:bd-pinf-univ} and \nameref{rule:bd-pinf-prod} correspond to the second case.

Note that while the use of conversion and reduction have changed, those relations themselves
remain untouched. We only refined them by giving them an explicit mode.

\subsection{Constrained inference in disguise}

This need to split the conversion rule into a reduction and conversion sub-routine depending on the mode is of course known to the implementors of proof assistants \sidecite{Abel2011}.
It explains in part the ubiquity of \kl(red){weak-head} reduction
in the dependently typed setting.
Indeed, it is exactly the minimal reduction strategy that is needed to expose the
head constructor of a type, and thus to implement constrained inference.

Still, reduction is only a means to determine whether a certain term fits into a certain category of types. In the setting of \kl{CCω}, this is basically the only way to do.
However, as soon as conversion is extended,
% for instance with unification
% \sidecite{Asperti2012}, coercions \sidecite{Asperti2012,Sozeau2007}
% or graduality \sidecite{LennonBertrand2022},
reduction is often not enough any more.
Putting constrained inference forward explains some ideas required in those: 
they are not ad-hoc workaround, but are based on the need to account for constrained inference.

We already mentioned \sidetextcite{Pollack1992}, where $\Gamma \vdash t \ty T$ is used
for inference, and a judgement written $\Gamma \vdash t \mathrel{:\geq} T$ –
denoting type inference followed by reduction –
is used to effectively inline the two hypothesis of our constrained inference rules.
Checking is also inlined.
Similarly, \sidetextcite{Abel2008} use a judgement written $\Delta \vdash V \delta \Uparrow \operatorname{Set} \rightsquigarrow i$, where a type $V$ is checked to be well-formed, but with its exact level $i$ free. This corresponds very closely to our use of $\pity{\uni}$.

But the main area where constrained inference repeatedly becomes apparent is that of
elaboration. For instance,
\sidetextcite{Saibi1997} describes an elaboration mechanism inserting coercions between types.
This happens primarily during checking, when both types are known.
However, \citeauthor{Saibi1997} introduces two special classes to handle the need
to cast a term to a sort or a function type without more information,
exactly in the places where we resort to constrained inference instead of checking.
More recently, \sidetextcite{Sozeau2007} describes a system where conversion is augmented
to handle coercion between subset types.
As in \textcite{Pollack1992}, $\Gamma \vdash t \ty T$ is used for inference,
and the other judgements are inlined.
Once again, reduction is not enough to perform constrained inference, this time
because type constructors can be hidden in subsets:
a type such as $\{f : \Nat \to \Nat \mid f\ 0 = 0 \}$
should be usable as a function of type $\Nat \to \Nat$.
An erasure procedure is therefore required on top of reduction to remove subset types in the places where we use constrained inference.

Analogous ideas can also be found in \kl{Matita}'s elaboration algorithm, as described in 
\sidetextcite{Asperti2012}.
Indeed, the presence of unification meta-variables on top of coercions makes it
even clearer that specific treatment of what we identified as constrained inference is
required.
\citeauthor{Asperti2012} introduce a special judgement they call
type-level enforcing, which corresponds to our $\pity{\uni}$ judgement.
As for $\pity{\P}$, they have two rules to apply a function, one where its inferred type reduces to a Π-type, corresponding to \ruleref{rule:bd-pinf-prod}.
and another one to handle the case when the inferred type instead reduces to a meta-variable.
As \citeauthor{Saibi1997} and \citeauthor{Sozeau2007}, they also
need to handle coercions for terms in function position. However, their solution is different:
They introduce new meta-variables for the domain and codomain, and rely on unification, which is available in their setting, to find values for those.
For $\pity{\uni}$, though, this solution is not viable, as one would need a kind of universe
meta-variable. Instead, they rely on backtracking to test multiple possible universe choices.

Finally, in \arefpart{gradual}, somewhat akin to the use of meta-variables in
\textcite{Asperti2012}, there are two rules per constrained inference judgement.
One when the head constructor is the desired one – as for \kl{CCω} –,
and a second one to handle the wildcard $\?$, characteristic of gradual type systems.


\section{Properties of the Bidirectional System}
\label{sec:bidir-prop}

Let us now state and prove the main properties of the bidirectional system.
The first two relate the bidirectional system to the
undirected one: it is both \kl(bidir){correct} – terms typeable in the bidirectional system are typeable in the undirected system – and \kl(bidir){complete} – all terms typeable in the undirected system are also typeable in the bidirectional system.
Next, we investigate uniqueness of typing, and its relation to the choice of a strategy for reduction.
Finally, we show how strengthening can be shown for undirected \kl{CCω} by proving it on the
directed side.

\subsection{Correctness and completeness}

A bidirectional derivation can be seen as a refinement of an undirected derivation.
Indeed, the bidirectional structure can be erased
– replacing each bidirectional rule with the corresponding undirected rule – to obtain an undirected derivation. This is missing some sub-derivations of context well-formedness at
leaves of a derivation, or that of the target type for \ruleref{rule:cic-conv}.
These can nevertheless be retrieved by using the invariants
on well-formedness of inputs and outputs.
Thus, we get the following correctness theorem – note how McBride’s discipline manifests as well-formedness hypothesis on inputs.

\begin{theorem}[\intro{Correctness} of bidirectional typing for \kl{CCω}]
  \label{thm:corr-ccomega}
  If $\Gamma$ is well-formed and $\inferty{\Gamma}{t}{T}$ or $\pinferty{h}{\Gamma}{t}{T}$,
  then $\Gamma \vdash t \ty T$.
  If both $\Gamma$ and $T$ are well-formed and
  $\checkty{\Gamma}{t}{T}$, then $\Gamma \vdash t \ty T$. 
\end{theorem}
  
\begin{proof}
  The proof is by mutual induction on the bidirectional typing derivation.

  Each rule of the bidirectional system can be replaced by the corresponding rule of the
  undirected system, with all three Rules \nameref{rule:bd-check}, \nameref{rule:bd-pinf-univ} and \nameref{rule:bd-pinf-prod} replaced by
  \nameref{rule:cic-conv}. In all cases, the induction hypothesis can be used on sub-derivations of the bidirectional judgement, because context extensions and checking are
  done with types that are known to be well-formed,
  either by induction hypothesis on previous premises and \kl{validity}.%
  \sidenote{This is the point where following McBride’s discipline is crucial!}

  Some sub-derivations of the undirected rules that have no counterpart
  in the bidirectional ones are however missing.
  In Rules \nameref{rule:cic-univ} and \nameref{fig:cic-var},
  the hypothesis that $\Gamma$ is well-formed is enough to get the required premise.
  For \ruleref{rule:bd-check},
  the well-formedness hypothesis on the type is needed to get the typing premise of
  \nameref{rule:cic-conv-unty}.
  As for Rules \nameref{rule:bd-pinf-univ} and \nameref{rule:bd-pinf-prod},
  that typing premise is obtained by combining the induction hypothesis,
  validity and \kl{subject reduction}.

  Alternatively, usage of validity could be handled by
  strengthening the theorem to incorporate the well-formedness of outputs on top of that of
  the subject. But we follow the proofs in \kl{MetaCoq}, which establish
  meta-theoretical properties of the undirected system – including validity –,
  rather than those of the bidirectional one – such as stability by substitution.

\end{proof}

Contrarily to correctness, which keeps the structure of a derivation,
completeness is of a different nature.
Because in bidirectional derivations the computation rules are much less liberal than in
undirected derivations,
the crux of the proof is to ensure that all uses of \ruleref{rule:cic-conv}
can be permuted down through the other, structural, rules,
in order to concentrate them in the places where they are authorized in the bidirectional
derivation.
In a way, composing completeness with correctness gives a kind of normalization procedure
which produces a canonical undirected derivation by pushing conversion
down as much as possible.

The proof crucially relies on the following lemma,
which can be seen as a form of \kl{injectivity of type constructors} – which is a direct
consequence of it.

\begin{lemma}[Conversion implies reduction for type constructors]
  \label{lem:conv-red-tycons}
  If $T \conv \uni[i]$, then $T \red \uni[i]$.

  If $T \conv \P x : A.\ B$, then there exist $A'$ and $B'$ such that:
  \begin{itemize}
    \item $T \red \P x : A'.\ B'$
    \item $A' \conv A$
    \item $B' \conv B$
  \end{itemize}
\end{lemma}

\begin{proof}
  Let us spell out the proof on Π-types – the case of $\uni$ is similar, but easier.

  For \kl{algorithmic conversion}, by definition there must exist $T'$ and $T''$
  such that $T \red T'$, $\P x : A.\ B \red T''$, $T' \alpheq T''$.
  But there can be no \kl{top-level} reduction step in $\P x : A.\ B \red T''$, so actually
  $T''$ is some $\P x : A''.\ B''$ and $A \red A''$, $B \red B''$.
  Similarly, $T'$ must be some $\P x : A'.\ B'$
  such that $A' \alpheq A''$ and $B' \alpheq B''$.
  Combining these, we obtain that $A' \conv A$ and $B' \conv B$, as expected.

  For \kl{declarative conversion}, one can use the previous proof of
  equivalence with algorithmic conversion – and thus \kl{confluence}.
\end{proof}

\begin{theorem}[\intro{Completeness} of bidirectional typing for \kl{CCω}]
  \label{thm:compl-ccomega}
  If $\Gamma \vdash t \ty T$, then there exists $T'$ such that $\inferty{\Gamma}{t}{T'}$
  and $T' \conv T$.
\end{theorem}

\begin{proof}
  The proof is by induction on the undirected typing derivation.
  
  Rules \nameref{rule:cic-var} and \nameref{rule:cic-univ} are base cases,
  and can be simply replaced by the corresponding bidirectional rules.
  In the case of \ruleref{rule:cic-conv}, the property is a direct consequence of the induction hypothesis together with transitivity of conversion: we simply conflate two conversions
  together.
  
  As for \ruleref{rule:cic-prod}, the induction hypothesis on the domain $A$
  gives the existence of $T_A$
  such that $\inferty{\Gamma}{A}{T_A}$ and $T_A \conv \uni[i]$. Using
  \cref{lem:conv-red-tycons}, we can derive $\pinferty{\uni}{\Gamma}{A}{\uni[i]}$.
  Applying a similar reasoning on the codomain and combining both is enough to conclude.


  In \ruleref{rule:cic-abs}, we do the same reasoning again on the type annotation.
  Combined with the induction hypothesis on the body $t$,
  we get $\inferty{\Gamma}{\l x : A.\ t}{\P x : A .\ B'}$ for some $B'$ such that $B \conv B'$, and thus $\P x : A . B \conv \P x : A . B'$ as desired.

  We are finally left with \ruleref{rule:cic-app}.
  Again, the key is \cref{lem:conv-red-tycons}, which can be combined with the induction
  hypothesis on the function $f$ to get $\pinferty{\P}{\Gamma}{f}{\P x : A' .\ B'}$
  for some $A'$ and $B'$ such that $A \conv A'$ and $B \conv B'$ – where $\P x : A. B$ is
  the type for $f$ in the undirected derivation.
  The induction hypothesis on the argument $u$ gives
  $\inferty{\Gamma}{u}{A''}$ with $A'' \conv A$. Thus, by transitivity of conversion
  $\inferty{\Gamma}{u}{A'}$, and we can apply \ruleref{rule:bd-app} to conclude.

\end{proof}

Interestingly, the proof of correctness relies on \kl{subject reduction}, which itself
needs \kl{injectivity of type constructors} and \kl{transitivity} of conversion.
Similarly, completeness relies both on the injectivity as given by \cref{lem:conv-red-tycons},
and transitivity of conversion. Be it for algorithmic or declarative conversion, one at
least of those is not directly provable – we need \kl{confluence}.
We already hit this same tension with \kl{subject reduction}, and
must draw the same conclusion: there is no free lunch!

\subsection{Uniqueness and reduction strategies}


\subsection{Strengthening}
\chapter{Bidirectional \kl{PCUIC}}
\label{chap:bidir-pcuic}

\section{Cumulativity}
\label{sec:cumulativity}

Principality

\section{Inductive Types}
\label{sec:inductives}

The thorny issue of case representation…
\chapter{Bidirectional conversion, Abel style}
\label{chap:bidir-conv}

Does this belong in this part, or in a self-standing part at the end?
It would make little sense to come back to this after formalisation + gradual typing,
but at the same time having it so early is a bit odd.


\pagelayout{wide} % No margins
\addpart{A Certified Kernel for \kl{Coq}, in \kl{Coq}}
\label{part:metacoq}
\pagelayout{margin} % Restore margins

\kl{Coq} is a very complex tool. Even its \kl{kernel}, which is only but a very small
fraction of the tool, is already quite complex, relying on subtle implicit invariants which
might not be properly maintained, especially when the code evolves.
In practice, around one critical bug is found every year.%
\sidenote{A \href{https://github.com/coq/coq/blob/master/dev/doc/critical-bugs}{compilation}
  of those is maintained by \kl{Coq}’s development team.}
Although it is not easy to exploit these to actually derive an inconsistency, even less
so inadvertedly, simply relying on the \kl{De Bruijn criterion} is not enough if
one wants to have a maximal trust in the kernel – and thus \kl{Coq} as a whole.
Although \kl{CIC} is well-understood and has been widely studied,
this is much less true of the type theory which is actually implemented, \kl{PCUIC}.
This is why bugs usually creep in with the extra level of complexity added by \kl{PCUIC},
which is rarely handled in details on paper proofs.%
\sidenote{This is for instance the case of the completeness issue
  explained in \cref{sec:bidir-pcuic-inductives}.}

This all begs for a precise investigation of \kl{PCUIC}, from the subtleties of the
type system’s meta-theory, all the way down to the sophisticated details of the implementation.
Due to the level of complexity of the endeavour, it is not feasible on paper. Nor is it
desirable: if in the end we wish to write down a certified kernel, it is natural to do so
in a proof assistant, so that we can run that certified implementation.
Instead, the natural framework is the \kl{MetaCoq} project,
which aims at giving tools to reify and manipulate \kl{Coq} terms%
\sidenote{Or, maybe more accurately, \kl{Gallina}.}
inside of \kl{Coq} itself. This gives the possibility to write down and certify
all kinds of procedures operating on these terms – the first to come to mind being of course
a type-checker. This way, we can have both the help and guarantees offered by
formal proofs inside a \kl{proof assistant}, and the possibility to execute
our implemented kernel.

There are two important caveats to this, though.
The first pertains to Gödel’s second incompleteness
theorem. Because of it, it is impossible to prove \kl{Coq}’s consistency inside \kl{Coq} itself,
meaning that the meta-theoretical study can only be partial, since otherwise it would allow
a proof of consistency contradicting Gödel’s theorem. Indeed, \kl{MetaCoq} relies on a single
axiom, asserting the \kl{normalization} of \kl{PCUIC},
which is the blind spot we must allow in order to circumvent this limitation.
A second is that writing down a certified kernel is not enough. Indeed, executing directly
such a kernel in \kl{Coq} would be much too slow to actually be able to type-check
any reasonable term. Rather, we must rely on extraction, a procedure which erases the
proof-related content of a certified program to only keep the algorithmically relevant one.
As this erasure itself is a complex transformation, \kl{MetaCoq} also incorporates a certified
erasure procedure.

In this part, I shall describe the portion of \kl{MetaCoq} which is relevant to the thesis.
\Cref{chap:metacoq-general} gives a general overview of the meta-theoretical study of \kl{PCUIC},
with the main definitions and properties.
My technical contributions to this part of the development is relatively minor,
mainly consisting of small patches. However, since I reuse quite a lot of those properties
in my main contributions, it seems fitting to at least say a word on it.
\Cref{chap:kernel-correctness} concentrates on the formalization of bidirectional typing, as
presented in \arefpart{bidir}, and on the proof of correctness and completeness of
the kernel implementation based on it. It contains my main technical
contributions to the \kl{MetaCoq} project.

Throughout the part, source files of the \kl{MetaCoq} project
and specific definitions or theorems are referenced respectively as follows:
\pcuicfile{Typing}, and \pcuicline{Typing}{typing}{188}. They link directly to the source
code of the project on \kl{GitHub}.
\chapter{The MetaCoq Project}
\label{chap:metacoq-general}

\margintoc

\section[Terms, conversion and types]{Setting up the definitions: terms, conversion and types}

\section[Stabilities]{The easy properties: stabilities}

\section[Confluence]{Things get serious: confluence}

\section[Properties of typing]{Reaping the fruits: properties of typing}

\section[Normalization]{Gödel’s thorn in the side: normalization}
\chapter{Building a Certified Kernel}
\label{chap:kernel-correctness}

\margintoc

With the meta-theory set down, we can turn to building a kernel – and proving that it is
correct. The first step (\cref{sec:kernel-bidir})
is to move from the declarative specification of
\cref{chap:metacoq-general} to a bidirectional presentation,
closer to the kernel we wish to implement.
Once this specification is set down, we can get to the kernel itself.
\cref{sec:kernel-subroutines} goes over the implementation of the \kl{global environment} and the
\kl{cumulativity} check, and \cref{sec:kernel-typing} describes the type-checker.
Finally, \cref{sec:kernel-beyond-typing} describes two extra functions belonging to the safe
\kl{kernel}: re-typing, and checking of \kl{global environment}.

I personally contributed the formalizations of \cref{sec:kernel-bidir},
the \kl(bidir){completeness} part of \cref{sec:kernel-typing} –
by modifying the pre-existing proven-\kl(bidir){sound} type-checker –,
and heavily modified re-typing – part of \cref{sec:kernel-beyond-typing}.

\section{Formalized Bidirectional Typing}
  [Bidirectional Typing, Formalized]
\label{sec:kernel-bidir}

We already saw the main theoretical ideas around our approach to bidirectional typing in
\arefpart{bidir}, so let us get to their implementation in the formalization.

\subsection{Definitions}

Before we can get to the definition of typing, we must go through the small
\pcuicfile[Bidirectional][BD]{EnvironmentTyping}, which is dedicated to refining a few
definitions on contexts in the bidirectional setting. First,
in the case of a definition \coqe{Γ,, vdef na b T},
\coqe{wf_local} enforces that \coqe{T} is a well-formed type, and that
\coqe{b} has type \coqe{T}. In the bidirectional setting we want to use constrained inference
$\pity{\uni}$ for the first, and checking for the second, but the generic definition
on which \coqe{wf_local} is built%
\sidenote{Called \coqe{All_local_env}.}
only allows for a single parameter – instantiated with \coqe{typing} in the case
of \coqe{wf_local}.
Similarly, we need a definition expressing that a context $\Delta$ is well-formed
\emph{over another context $\Gamma$}, but which does not enforce $\Gamma$ to be well-formed
\textit{a priori} – \eg something more precise than simply $\vdash \Gamma , \Delta$. This
allows to stay faithful to McBride’s discipline%
\sidenote{Taken from \textcite{McBride2018}, see its exposition in \cref{sec:bidir-mcbride}.}%
\margincite{McBride2018}
when typing context extensions,
by only demanding that the extension is well-formed, but not the initial segment.%
\sidenote{Which is an input, and thus should not be re-checked.}

The bidirectional typing judgment is defined in
\pcuicfile[Bidirectional][BD]{Typing}, as a set mutual defined inductive predicates:
one for inference, one for checking, and one for each constrained inference, \eg respectively
sorts, Π-types and inductive types. The definition of the predicates themselves
is very close to that of \cref{fig:ccw-bidir-infer,fig:bidir-ccw-other} for the
functional fragment, the main innovation being that constrained inference – 
written \coqe{Σ ;;; Γ |- t |>Π (na,A,B)} for Π-types –
takes the variable name, domain and codomain of the inductive type as three separate arguments,
which our pen-and-paper notation did not make explicit.
The predicates defined in \pcuicfile[Bidirectional][BD]{EnvironmentTyping} are used in the
definition of inference for the \coqe{tCase} node, where we want to ensure that the
context extensions used to type the \kl{predicate} and \kl{branches} are well-formed.

Regarding the notions of computation, the definitions of \kl{constrained inference} use
\kl{full reduction} rather than \kl{weak-head reduction}, mainly because MetaCoq currently
lacks a treatment of the latter adequate for our needs.%
\sidenote{In particular, a standardization theorem is missing,
which would be needed to show the analogue
of \cref{lem:conv-red-tycons} for weak-head reduction.}
As for \kl{cumulativity}, the \kl(conv){algorithmic} variant is used in the \kl{checking} rule,
but this is relatively irrelevant,
since the equivalence between both presentations of \kl{cumulativity}
appears much earlier in the development than bidirectional typing.

Maybe more interesting from the formalization point of view is how we obtain a usable
induction principle. This is a common issue in \kl{MetaCoq}: while
\kl{Coq} is able to detect that our inductive definitions are well-founded,
the default generation is often unable to derive a sensible induction principle,
and neither are the \coqe{Scheme} specialized commands. This is due to their nested
character, \ie the presence of lists and records containing recursive instances
of the inductive types as arguments to the type constructors.
The bidirectional typing predicate is the paroxysmal example of this, as it
reaches the limit of expressiveness offered by
\kl{Coq}’s inductive types: it is not only nested, but also mutual.
We thus have to prove our desired induction principle by
hand. To do so, we introduce a notion of “generic” typing object
\pcuicline[Bidirectional][BD]{Typing}{typing\_sum}{256},
together with a notion of size for such a typing object,
and finally show the induction principle
\pcuicline[Bidirectional][BD]{Typing}{bidir\_ind\_env}{312}
by well-founded induction on that size.

This induction principle is not as strong as we might expect, as it does not provide the extra induction hypothesis on inputs that would go with McBride's discipline.
Ideally, we could use this discipline
in order to thread the well-formation invariants, giving stronger induction
hypotheses. I did not try to take this path and prove such a strong induction principle,
as it did not seem so easy: it would effectively correspond to an inline proof
of \kl{validity}.
Instead, the discipline is reflected in the choice of the predicates proven by induction.
For instance, in the case of soundness, the mutually proven predicate for inference
is \coqe{wf_local Σ Γ -> Σ ;;; Γ |- t : T}, and more generally
assumptions are added as pre-condition for all inputs.
Still, I conjecture that such a strong induction principle should be provable, if the need
would arise, and might be nice in order to factor proofs, by showing once and for all that
the rules follow the discipline correctly.

\subsection{Equivalence with undirected typing}

\kl(bidir){Soundness}%
\sidenote{Bidirectional typing implies undirected typing, akin to \cref{thm:corr-ccomega}.}
is shown in \pcuicfile[Bidirectional][BD]{ToPCUIC}.
The main proof is by induction on the derivation,
its key point being to show that well-formation invariants are preserved, and in particular
that all contexts that are constructed are valid.

There is one particular difficulty linked to the \coqe{tCase} constructor, and the question
of its representation evoked at the very end of \cref{sec:bidir-pcuic-inductives}.
More precisely, the issue is related to the fact that case nodes store the universe
instance and parameters of the inductive type being matched upon, in order to be able to
construct the context in which the \kl{predicate} and \kl{branches} are typed.
In undirected typing, the hypothesis on the \kl{scrutinee} is that it should be of some type
\coqline|mkApps (tInd ci.(ci_ind) p.(puinst)) (p.(pparams) ++ indices)|
where \coqe{ci.(ci_ind)}, \coqe{p.(puinst)} and \coqe{p.(pparams)}
are respectively the inductive type being matched upon, its universe instance, and its
parameters – all stored in the case node –, and \coqe{indices}
are free. From the point of view of bidirectional typing, this rule is invalid:%
\sidenote{It is not mode-correct \cite{Dunfield2021}.}%
\margincite{Dunfield2021}
because \coqe{indices} is free, this cannot be turned into a checking premise, but it also
cannot be directly turned into inference, or even constrained inference, because it is
not free enough due to the presence of \coqe{p.(pparams)} and \coqe{p.(puinst)}.
The solution is still to turn it into an inference premise
\coqe/Σ ;;; Γ |- c |>{ci} (u,args)/, and to compare the inferred universe instance \coqe{u}
and parameters – the first part of the list \coqe{args} – to those stored in the node, \eg
\coqe{p.(puinst)} and \coqe{p.(pparams)}. But it requires some work to
show that this relation between the two lists of parameters is enough to use the second part
of \coqe{args}, the inferred indices, in place of \coqe{indices} above.

In the opposite direction, \kl(bidir){completeness}%
\sidenote{Undirected typing implies bidirectional typing, akin to \cref{thm:compl-ccomega}.}
is also proven by induction, once we have used
the injectivity properties of \pcuicfile{Conversion} to show that
inference of a type related by \kl{cumulativity} to a sort, Π- or inductive type implies
constrained inference of the corresponding kind.
In order to simplify proofs in the case of projections,
\kl(bidir){soundness} is used in conjunction
with \kl{validity}, but this could probably be avoided, making the two proofs independent.

\subsection{Properties of bidirectional typing}

As we did in \cref{thm:unique-inf}, we show that two inferred types have a common reduct in
\pcuicfile[Bidirectional][BD]{Unique}. While the proof requires some playing with
well-scoping predicates,%
\sidenote{
  To relate the reduction used to defined \kl{constrained inference} and the one on which most lemmas
  around confluence are stated, which is defined directly on well-scoped terms.}
it is conceptually \emph{much} simpler than the direct proof of \pcuicfile{Principality},
which shows the existence of \kl{principal types} without going through bidirectional typing.
Indeed, due to the difficulty of the proof,
for quite some time only a weaker version was proven.
This version that if $T$ and $T'$ are both types
for the same term $t$ then there exists a third $T''$ which
is both a type for $t$ and smaller than 
$T$ and $T'$ for cumulativity.

Finally, \pcuicfile[Bidirectional][BD]{Strengthening} shows strengthening.
Its first important property is that if $\Gamma \vdash t \ity T$ then
$T$ can only use variables appearing in either $t$ or in the types of $\Gamma$
(\pcuicline[Bidirectional][BD]{Strengthening}{infering\_on\_free\_vars}{487}),
which is \emph{not} true in general in undirected typing.%
\sidenote{Consider for instance $n : \Nat \vdash 0 : (\l x : \Nat.\ \Nat)\ n$: $n$ appears in the
type, but neither in the body of the term nor any types in the context.}
It then goes on with the proof that bidirectional typing is stable under any renaming, while
\pcuicfile{RenameTyp} only shows stability of undirected typing under unconditional renaming.
Finally, we get to the proof of \pcuicline[Bidirectional][BD]{Strengthening}{strengthening}{989}
\textit{per se}, once we have shown that strengthening is indeed a well-formed renaming.

\section{Before Typing: Environment Querying and Cumulativity Checking}[Before Typing]
\label{sec:kernel-subroutines}

Before we can get to typing, we need to have a look at its two main sub-routines:
querying the \kl{global environment}, and \kl{cumulativity} check.

\subsection{Abstract environment}

The type and \kl{cumulativity} checking algorithms both need to query the \kl{global environment},
for two main purposes: retrieving information about previous definitions
of constants and inductive types, and checking that (in)equalities between
universe expression hold.

While this might seem anecdotal,
a surprisingly important amount of time is spent in the actual checker
on the second problem, which requires a form of shortest-path algorithm
on a graph obtained from the universe constraints, in order
to detect the presence of negative cycles.
These correspond to violations of the universe stratification.
\kl{MetaCoq} implements such an algorithm, with a proof that it is correct – \ie sound and complete –, meaning that
the algorithm answers “yes” exactly when there is a mapping from universe levels to integers satisfying
all constraints declared in the environment.
More details on this can be found in \sidetextcite[][Section 3.3]{Sozeau2020}.

Sadly, said algorithm is too naive to be actually run on reasonable examples: it is currently the main
performance bottleneck of the extracted type-checker. Similarly, the representation of the
\kl{global environment} as a list of definitions is too naive to allow for efficient lookups –
\kl{Coq} uses hash maps instead.
While we hope to replace that naive implementation with a more efficient but still certified one,
for the moment it is convenient to be able to plug an uncertified but efficient implementation
into the extracted type-checker. To do so, we rely on abstract interfaces for the
\kl{global environment}, containing all the possible queries we need to perform,
presented in \safefile{WfEnv}. The naive implementation is shown to be a valid implementation of
that interface in \safefile{WfEnvImpl}.

\subsection{Cumulativity checking}

The most important sub-routine of the type-checker is the test of \kl{cumulativity} between two terms.
The naive way to perform this, since we assume \kl{normalization}, would be to brutally normalize terms,
and compare normal forms up to \coqe{leq_term}.%
\sidenote{The extension of \kl{α-equality} to handle \kl{cumulativity}.}
But this strategy does not scale as soon as definitions are present, because it
eagerly unfolds all of these, resulting in a very inefficient test.
\kl{MetaCoq} implements a more practical strategy, which coarsely does the following:
\begin{enumerate}
  \item reduce both terms being compared to \kl(red){weak-head} normal form
    \emph{without unfolding any definition};
  \item if the two heads match, recursively compare sub-terms;
  \item if the two heads do not match, or if the recursive sub-term comparison failed, check if
    an unfolding is possible which would unblock one of the terms, and if yes, unfold it and
    go back to the first step.
\end{enumerate}
This means that the \kl{cumulativity} test must itself resort to a \kl{weak-head reduction}
function.

The difficulty with those functions is that they do not operate by a simple structural induction on
terms. Rather, they are defined using a complex abstract machine, operating on terms decomposed into
a sub-term and a stack. The termination of that abstract machine is shown using a
dependent lexicographic pre-order, which handles both the well-founded reduction order given by
\kl{normalization}, a structural order on sub-terms and stacks corresponding to a given term, and
the different phases of the algorithm.
A detailed description of this algorithm and its formalization%
\sidenote{In files \safefile{SafeReduce} for the implementation of \kl{weak-head reduction}, and
\safefile{SafeConversion} for the cumulativity checker.} is given by Théo Winterhalter,
who implemented it,%
\sidenote{It was only proven sound at the time. Jakob Botch Nielsen
wrote most of the completeness proof.
}
in his PhD thesis \sidecite[][Chapters 21-24]{Winterhalter2020}.

An interesting point that the test of \kl{cumulativity} and that of typing have in
common, is the way they handle their propositional content. First, because
we want to avoid issues linked with proof-irrelevance,
in most of the formalization definitions are in \coqe{Type},
including the reduction, conversion and typing relation.
But in the verified \kl{kernel} we want to enforce the
separation between propositional and relevant content. Thus, we use explicit squashing —
written $\| T \|$ – to cast a type into a proposition.%
\sidenote{Technically,
  $\| T \|$ is defined as a record of type $\Prop$ with a single field of type $T$.}
The main elimination of propositional content into the relevant world we rely on that
of accessibility, so that we can define \kl{reduction} and \kl{cumulativity} by well-founded induction. As customary in dependently typed code, we also use elimination of falsity in
inaccessible branches.

Second, we write code using the \kl{Equations} plugin, which lets us write the
relevant part of the definition in direct style, but to leave proofs to be filled-in using the
proof-mode. The definitions are given in monadic style, relying on what looks like the error monad:%
\sidenote{Given a type of errors $E$, the functor associated with that monad is $T \mapsto T + E$,
its unit is the left injection, and its bind $x >>= f$ either propagates $x$ if it is an error, or applies $f$ otherwise.}
the \kl{cumulativity}- and type-checker return a valid output, or an error message. However, since
we wish the functions to be correct by constructions, they must also return a proof, either a
witness for the positive answer, or a proof of impossibility in the negative case.
This means that the bind of the monad must actually perform a proof when re-raising the error,
in order to propagate the impossibility witness.%
\sidenote{For instance, if we are typing some \coqe{tProd na A B} and the call to typing for \coqe{A} 
fails, we must transform a proof that \coqe{A}
cannot be well-typed into a proof that \coqe{tProd na A B}
as a whole cannot either.}
At function definition, this is hidden by notations, so it feels like we are actually using a
monad, but under the hood proof obligations are generated each time we use a bind.

\section{Sound and Complete Inference}
\label{sec:kernel-typing}

Given the work already done in \cref{sec:kernel-bidir}, the definition of a
type checking algorithm \safefile{TypeChecker}
itself is rather straightforward: it follows closely
the structure laid out by the mutually defined bidirectional judgments, and
poses no termination issue as \kl{cumulativity} does, since it
operates by induction on the structure of the term.
Actually, rather than a type-checker, the main function we define is
\safeline{SafeChecker}{infer}{1323}, which performs type inference,%
\sidenote{
  Which is decidable thanks to our \kl{Church-style} syntax.}
from which we can easily define type-checking.

In more details, the function takes as inputs:
\begin{enumerate}
  \item an abstract environment implementation;
  \item a \kl{global environment} implemented using that implementation;
  \item a context, and a squashed proof that it is well-formed;
  \item a term
\end{enumerate}
and it returns either a type and a (squashed) proof that the term infers that type, or
an error and a proof that the term cannot infer any type, using the inductive type presented in
\cref{fig:meta-error-mon}.
\begin{figure}
  \coqfile[firstline=1,lastline=4]{./code/PCUICSafeTyping.v}
  \caption{The “error monad” used for \safeline{SafeChecker}{infer}{1323}’s return type}
  \label{fig:meta-error-mon}
\end{figure}
Thus, the function is sound and complete by construction.
In fact, we cannot separate
the definition from the soundness proof, since the conversion checker
expects a well-typed term as input in order to be terminating when it is called.

\begin{figure*}
  \coqfile[firstline=6]{./code/PCUICSafeTyping.v}
  \caption{Definition of \safeline{SafeChecker}{infer}{1323} (excerpt)}
  \label{fig:meta-infer}
\end{figure*}

\cref{fig:meta-infer} gives – an excerpt of – the algorithm.
%
For a variable \coqe{tRel n}, it checks that the variable is bound in
\coqe{Γ}, returns its type when it is, and fails otherwise.
%
In the case of a sort \coqe{tSort u}, it checks that the universe is well-formed in the current
environment, and returns a sort at the next level when it is.
%
In that of a dependent function type \coqe{tProd na A B}, it computes the sort of
\coqe{A} and \coqe{B} – in the context extended by
\coqe{na:A} –  using the \coqe{infer_type} sub-routine,%
\sidenote{Defined beforehand in an “open recursion” flavour, \eg as a function taking \coqe{infer}
as argument.}
and builds from those the sort of the product using \coqe{sort_of_product}.
Functions are similar.
%
The cases of \coqe{tLetIn} and \coqe{tApp} clearly show the bidirectional
structure. For instance, in
\coqe{tApp t u}, one needs to infer the type \coqe{ty} of \coqe{t},
then reduce it to some \coqe{tProd na A B} using the
\coqe{reduce_to_prod} function, and finally check that \coqe{u} has
type \coqe{A}.
%
All underscores \coqe{_} in the terms denote proof obligations, that are filled later on
in tactic mode. Although they are hidden, the monadic notations \coqe{;;}
and \coqe{raise} also contain underscores for the propagation of completeness
information.%
\sidenote{For instance, \coqe{raise e} is syntactic sugar for \coqe{TypeError_comp e _}.}

Interestingly, the proofs of \kl(bidir){completeness} use uniqueness of inferred types a lot.
To see why, consider \eg the case of an application $t\ u$ where the recursive call succeeds on $t$
– say it infers a product type $\P x : A.\ B$ – but the one on $u$ fails – giving us a proof $p$
that $u \cty A$ is absurd. We want to raise an error, and thus need to prove
that $t\ u \ity T$ for any $T$ is absurd.
An inversion on that last hypothesis gives some $A'$ and $B'$ such that
$t \pity{\P} \P x : A'.\ B'$ and $u \cty A'$.
But this second property cannot be directly fed $p$, because the type against which $u$ checks is
different! We thus need to use the two inference judgments and uniqueness to conclude that in
fact $A \conv A'$, and thus that $u \cty A$,
which this time we can the use to derive a contradiction from $p$.

\section{Beyond Typing: Environment Checking and Re-Typing}
\label{sec:kernel-beyond-typing}

There are two more functions defined in \kl{MetaCoq} that are very close to the type-checker.

\subsection{Re-Typing}

The first, which is defined in \safefile{SafeRetyping}, aims at computing a type for a term
which is known to be well-typed. While this seems tautological, it is not: the aim is
to extract relevant content out of propositional one. This is useful in practice in \eg
the extraction procedure, which maintains the invariant that it operates on well-typed terms,
but at times needs to actually compute types to decide whether terms should be erased.

This is also different from standard inference,
because knowing \textit{a priori} that the term under consideration is well-typed allows to
skip a lot of checks. For instance, to re-type an application $t\ u$, it suffices to infer a
product type $\P x : A.\ B$ for $t$, and to return $\subs{B}{x}{u}$, since we know that $u$
has type $A$.

In order to be useful, this re-typing procedure needs to compute a \kl{principal type}, and
thus its definition was quite complex prior to the formalization of bidirectional typing,
effectively inlining a proof of uniqueness of types. Instead, bidirectional typing
simplifies greatly both the definition of re-typing and its proof of correctness, by
clarifying its specification:
instead of computing a principal type out of any type,
the function should compute an unsquashed inferred type out of a squashed one.

\subsection{Environment Checking}

The second thing we need to handle is the verification that a whole \kl{global environment}
is well-formed. While the main thing to check is that all definitions are well-typed,
there are quite a few more things to be done: checking universes constraints, that inductive
definitions are strictly positive, that the variance information used by universe polymorphism is
valid… All these are covered in \safefile{SafeChecker}.

\pagelayout{wide} % No margins
\addpart{Bidirectional Elaboration for Gradual Typing}
\label{part:gradual}
\pagelayout{margin} % Restore margins

We have already seen in \arefpart{metacoq} how the structure of bidirectional typing can
help with proofs on \kl{CIC}/\kl{PCUIC}. But this is far from being the only advantage of
the approach. Indeed, the extra control provided on the conversion rule can be instrumental.
In this part, we go over one situation where this is the case: the extension of \kl{CIC} to
incorporate \kl(typ){gradual} features.

\kl{Gradual typing} arose as an approach to selectively and soundly relax static type
checking by endowing programmers with imprecise static
types \sidecite{Siek2006,Siek2015}. Optimistically well-typed
programs are safeguarded by runtime checks that detect violations of
statically-expressed assumptions. A gradual version of the simply-typed lambda
calculus enjoys such flexibility that it can embed the untyped lambda
calculus \cite{Siek2015}.
This means that gradually-typed languages tend to accommodate at least
two kinds of effects, non-termination and runtime errors.
% The smoothness of the static-to-dynamic checking spectrum afforded by gradual languages
% is usually captured by (static and dynamic) gradual
% guarantees
% which stipulate that typing and reduction are monotone with respect
% to precision~\cite{siekAl:snapl2015}.

Originally formulated in terms of simple types, the extension of \kl{gradual typing}
to a wide variety of typing disciplines has been an extremely active topic of
research, both in theory and in practice. As part of this quest towards more
sophisticated type disciplines, gradual typing was bound to meet with full-blown
dependent types. This encounter saw various premises in a variety of approaches
to integrate (some form of) dynamic checking with (some form of) dependent
types \sidecite{Ou2004,Wadler2009,Knowles2010,Tanter2015,Lehmann2017,Dagand2018}.
Naturally, the highly-expressive setting of dependent types, in which terms and
types are not distinct and computation happens as part of typing, raises a lot
of subtle challenges for gradualization.

Of those challenges, one of the first is the place of computation.
In the gradual setting, in order to optimistically compare types,
\kl{conversion} is replaced by \kl(grad){consistency}, a relation
akin to unification. This relation is naturally
non-transitive, meaning that the usual, \kl(typ){undirected} setting is not adapted to
gradualization.%
\sidenote{
  This is because \ruleref{rule:cic-conv} can be applied any number of times, which is sensible
  only if these successive application amount to just one.
}
Moreover, the semantics of \kl(typ){gradual} languages is usually explained
through an elaboration phase to a second language, responsible for the runtime checks ensuring
safety of evaluation. This elaboration is naturally described in a bidirectional
system, which moreover provides enough constraints on the typing derivation so that replacing
\kl{conversion} with \kl(grad){consistency} is reasonable. Finally, the identification of
the role of \kl{reduction} for \kl{constrained inference} clarifies
how the latter should be extended to incorporate imprecise types.
In fact, I told the story upside down: it is the need for bidirectional typing in the context
of \kl{gradual typing} that led me to its investigation!

In this part, we go over a collaboration with Kenji Maillard, Éric Tanter and Nicolas
Tabareau to address the challenge of gradualizing a full-blown dependently-typed language:
\kl{CIC} \sidetextcite{LennonBertrand2022}.
\cref{chap:gradual-dependent} gives a general overview of the challenges and trade-offs
involved in \kl(typ){gradual} \kl(typ){dependent} types, culminating with the
\kl{Fire Triangle of Graduality}, which identifies an irreconcilable tension between
the properties one should demand of such a type system. It ends with a broad picture
of our proposed solution to those difficulties,
the \reintro{Gradual Calculus of Inductive Constructions} (\kl{GCIC}).
Next, \cref{chap:bidir-gradual-elab} describes precisely this \kl{GCIC},
via a relation representing type-based, bidirectional elaboration,
which represents my main technical contribution to \textcite{LennonBertrand2022}.
Finally, \cref{chap:beyond-gcic} gives an overview of the rest of our work in the area:
models used to establish properties of the target language of the elaboration procedure,
and the thorny question of indexed inductive types and consistent reasoning about gradual
programs.
Due to their absence of direct relation to bidirectional typing and my lower
involvement in their technical development, we do not go into full details.
The interested reader can of course consult the appropriate publications for those –
the former appears in \textcite{LennonBertrand2022}, while the latter is the subject
of \textcite{Maillard2022}.
\chapter{Gradual Typing Meet Dependent Types}
\label{chap:gradual-dependent}

Take inspiration from first sections of TOPLAS paper here.
\chapter{From GCIC to CastCIC: Bidirectional Elaboration}
\label{chap:bidir-gradual-elab}

Again, should be an adaptation from TOPLAS. Maybe I can do a bit more, especially if
I concentrate on the bidirectional way of things. For instance, I could try and give a better algorithm for deciding conversion than the crappy one from the paper.
\chapter{Beyond \kl(tit){CastCIC}}
\label{chap:beyond-gcic}

\section{Realizing \kl(tit){CastCIC}}
\label{sec:realizing-cast-calculus}


\section{A Reasonably Gradual Type Theory}
\label{sec:ReTT}

% \appendix % From here onwards, chapters are numbered with letters, as is the appendix convention

% \pagelayout{wide} % No margins
% \addpart{Appendix}
% \pagelayout{margin} % Restore margins

% \chapter{Some more blindtext}

% \blindtext

%----------------------------------------------------------------------------------------

\backmatter % Denotes the end of the main document content
\setchapterstyle{plain} % Output plain chapters from this point onwards

%----------------------------------------------------------------------------------------
%	BIBLIOGRAPHY
%----------------------------------------------------------------------------------------

% The bibliography needs to be compiled with biber using your LaTeX editor, or on the command line with 'biber main' from the template directory

\defbibnote{bibnote}{Here are the references in citation order.\par\bigskip} % Prepend this text to the bibliography
\printbibliography[heading=bibintoc, title=Bibliography, prenote=bibnote] % Add the bibliography heading to the ToC, set the title of the bibliography and output the bibliography note

%----------------------------------------------------------------------------------------
%	INDEX
%----------------------------------------------------------------------------------------

% The index needs to be compiled on the command line with 'makeindex main' from the template directory

% \printindex % Output the index

\newpage \mbox{}
\cleardoubleevenemptypage
\includepdf[pages=2]{../MathSTICTemplate/main.pdf}

\end{document}
