\chapter{Introduction (Français)}
\label{ch:intro-fr}

\selectlanguage{english}
\begin{kaobox}[backgroundcolor=Black!10!White,frametitlebackgroundcolor=Black!10!White]
  This section is an introduction intended for French-speaking readers.
  If your English is better than your French,
  you should instead read \cref{ch:intro-en}, its translation in English.
\end{kaobox}
\selectlanguage{french}

Cette thèse se situe dans le domaine de la \kl{théorie des
types (dépendants)}, lui-même au croisement entre informatique et logique mathématique.
L’objectif principal est de participer aux fondements théoriques et pratiques des outils
que l’on appelle \kl{assistants à la preuve}, des logiciels qui, comme leur nom
l’indique, ont pour but d’assister des êtres humains dans la construction
et la vérification de preuves – au sens mathématique du terme. Il sera dans cette thèse
en particulier beaucoup question de l’assistant à la preuve \kl{Coq}, qui est celui
sur lequel mon travail s’est principalement concentré.

Pour replacer ce travail dans son contexte large, je propose dans cette introduction une histoire très parcellaire et orientée de la logique mathématique
(\cref{sec:logique-histoire}), puis une courte présentation des assistants à la preuve,
notamment ceux qui, comme \kl{Coq}, se basent sur la théorie des types (\cref{sec:assistants-preuve}). Enfin je finis par une présentation de mes contributions
personnelles pendant la durée de cette thèse (\cref{sec:cette-these}).

\section{Une très courte histoire de la logique}
\label{sec:logique-histoire}

\paragraph{Les syllogismes.}
Dans la tradition occidentale, on peut faire remonter l’étude de
la logique à Aristote, dans son \textit{Organon}.\todo{Citer Aristote correctement}
L’un des apports de ce travail est d’introduire les syllogismes.\sidenote{
  Le syllogisme le plus connu est probablement Barbara, dont un exemple est :
  \emph{tous les humains sont mortels ; Socrate est humain ; donc Socrate est mortel.}
}
Il s’agit de raisonnements dont la validité tient seulement au fait qu’ils
suivent une structure générale, et non à son contenu particulier.
Si un raisonnement est construit en assemblant ces syllogismes,
le raisonnement dans son entier doit nécessairement l’être également, puisque
chaque pas de raisonnement est valide.
L’idée importante ici est celle de décomposition en composantes élémentaires. À
partir d’un système de règles de raisonnement qu’on a identifiées comme valides 
\textit{a priori},\sidenote{
  Il peut s’agir de syllogismes, mais de bien d’autres systèmes… On en rencontra
  un certain nombre dans cette thèse !
}
on a un moyen de s’assurer de la validité de raisonnements potentiellement
très complexes.
Il suffit de vérifier qu’ils peuvent être décomposés à partir
des règles de base – et, bien entendu, que celles-ci soient correctes.

\paragraph{Les débuts de la logique mathématique.}
À la suite d’Aristote, les mathématicien·ne·s se sont emparé·e·s de la question
de la logique, en cherchant comment il était possible de fonder les mathématiques
rigoureusement. Bien qu’il s’agisse d’une question ancienne, de véritables
progrès concrets sur sa résolution ont commencé à voir le jour dans la deuxième
moitié du 19\textsuperscript{e} siècle, sur deux fronts principaux.

Le premier a consisté à se dégager du langage dit
naturel\sidenote{
  Par opposition aux langages formels qui apparaissent
  en mathématiques, informatique, etc.
}, inadapté pour la description formellement précise de la déduction, et à
concevoir à la place une nouvelle forme de langage spécifique qui puisse servir de
base à un système de raisonnement. Une étape importante
de cette ligne de recherche est
probablement \sidecite{Begriffsschrift}, qui introduit un certain nombre de
caractéristiques des langages dont il sera question dans la suite de cette thèse,
en particulier la notion de quantificateur.

Le second a pour but de montrer que les mathématiques dans leur entier peuvent
effectivement être reconstruites à partir de briques élémentaires. Une étape
importante ici a été la réduction de l’analyse à un petit nombre de propriétés
des nombres réels, puis les constructions de ces nombres réels à partir
de l’arithmétique, données simultanément par plusieurs auteurs\sidenote{
  Cantor, Méray, Dedekind, Bertrand, Weierstraß
}\todo{citation?} autour de 1870. De son côté Peano \cite{??} propose
l’axiomatisation des nombres entier qui est nommée en son honneur.
Enfin Cantor \cite{??}
propose la théorie des ensembles comme un formalisme permettant
de décrire la totalité des objets mathématiques sous la forme d’ensemble
d’éléments.

\begin{itemize}
  \item les soucis du langage naturel
  \item crise des fondements : quel système choisir ? -> Gödel : saut de la foi + vérité vs prouvabilité
\end{itemize}
Conclusion : les mathématicien·ne·s sont globalement content·e·s

\section{Les ordinateurs entrent en scène}
\label{sec:assistants-preuve}

\subsection{Pourquoi les ordinateurs ?}

\begin{itemize}
  \item Rend la vérification de preuve faisable quoique difficile, plutôt que parfaitement impossible.
  \item Automatisation et calcul sous la main, très pratique pour plein de situations mathématiques.
  \item Preuves et programmes au même endroit, parfait pour… la preuve de programmes.
\end{itemize}

\subsection{Théorie des types et langages de programmation}

Curry-Howard, et la métaphore de la grammaire.

\subsection{Une courte histoire des assistants à la preuve}

À garder ?

\section{Et cette thèse, alors ?}
\label{sec:cette-these}

\subsection{Le typage bidirectionnel}

\subsection{MetaCoq}

\subsection{Élaboration bidirectionnelle pour le typage graduel}
