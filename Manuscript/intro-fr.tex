\chapter{Résumé}
\label{ch:intro-fr}

\selectlanguage{english}
\begin{kaobox}[frametitle=Warning for the English-speaking readers, backgroundcolor=Black!10!White,frametitlebackgroundcolor=Black!10!White]
  This section is an introduction intended for French-speaking readers. If you do not
  read French, you should read \cref{ch:intro-en}, which is its translation in English.
\end{kaobox}
\selectlanguage{french}

Où j’introduis le sujet de manière générale, en français (voir \cref{ch:intro-en} pour l’anglais).

\section{Une très courte histoire de la logique}

\begin{itemize}
  \item logique grecque (sillogismes, petites composantes de base irréfutables)
  \item les soucis du langage naturel
  \item crise des fondements : quel système choisir ? -> Gödel : saut de la foi + vérité vs prouvabilité
\end{itemize}
Conclusion : les mathématicien·ne·s sont globalement content·e·s

\section{Les ordinateurs entrent en scène}

\subsection{Pourquoi les ordinateurs ?}

\begin{itemize}
  \item Rend la vérification de preuve faisable quoique difficile, plutôt que parfaitement impossible.
  \item Automatisation et calcul sous la main, très pratique pour plein de situations mathématiques.
  \item Preuves et programmes au même endroit, parfait pour… la preuve de programmes.
\end{itemize}

\subsection{Théorie des types et langages de programmation}

Curry-Howard, et la métaphore de la grammaire.

\subsection{Une courte histoire des assistants à la preuve}

À garder ?

\section{Et cette thèse, alors ?}

\subsection{Le typage bidirectionnel}

\subsection{MetaCoq}

\subsection{Élaboration bidirectionnelle pour le typage graduel}
