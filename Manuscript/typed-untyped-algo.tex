\chapter{Bidirectional conversion}
\label{chap:bidir-conv}

In \sidetextcite{Abel2017}, decidability of conversion is shown by introducing a notion
of “algorithmic conversion”, a relation which directly corresponds to an algorithm to
check convertibility of two terms. This is very similar to how we show decidability of typing
in \arefpart{metacoq} by going through bidirectional typing as an intermediate, more structured
representation.
Even better: this conversion is type-based,%
\sidenote{It uses type information to trigger η-expansion when comparing inhabitants of
a Π-type.}
and it is in fact bidirectional! Indeed, while regular conversion-checking uses a type as
input, it is mutually defined with conversion of neutrals, which \emph{infers} a type
while converting neutrals.

This section is devoted to the re-casting of the ideas of \sidetextcite{Abel2017} in our
bidirectional setting, and to the proof, made relatively easy by the bidirectional structure,
that this typed conversion relation agrees with an untyped one, close to the conversion algorithm
implemented in \kl{Coq}.