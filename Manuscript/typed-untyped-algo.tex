\chapter{Bidirectional Conversion}
\label{chap:bidir-conv}

In \cref{chap:bidir-ccw,chap:bidir-pcuic}, we considered typing, and saw how it could be
turned into a bidirectional relation. However, we did not consider \kl{conversion}.
Indeed, since we chose to use an \kl(conv){untyped} notion of conversion, a bidirectional
approach would not have made sense, as there was no type around in conversion.

However, the \kl(conv){typed} presentation of conversion is also a popular one, and in that
setting the question of giving a bidirectional presentation \emph{is} sensible.
Luckily, such a presentation is already available if we go through the literature with the
right glasses on. Indeed, in \sidetextcite{Abel2017},
decidability of conversion is shown by introducing a “conversion algorithm”,
a relation presented via inference rules, but which directly corresponds to
an implementable convertibility check.
This is somewhat similar to how we show decidability of typing
in \arefpart{metacoq} by going through bidirectional typing as an intermediate,
more structured representation.
But the interesting point is that this \kl(conv){typed}%
\sidenote{Type information is used to trigger η-expansion when comparing inhabitants of
a Π-type.},
algorithmic conversion, and it is, in fact, bidirectional!
Indeed, while regular conversion-checking uses the type as
\kl{input}, it is mutually defined with a specific relation to compare \kl{neutrals},
which \kl[inference]{\emph{infers}} a type while checking that the neutrals are convertible.
In this chapter, we re-cast the ideas of \textcite{Abel2017} in our setting,
clearly delineating their bidirectional nature.

Moreover, we can use that bidirectional structure
to show that this typed algorithmic conversion agrees with an untyped one,
close to the conversion algorithm implemented in \kl{Coq}.
This is interesting, because currently \kl{PCUIC} as presented in \kl{MetaCoq}
is not able to handle extensionality rules such as the η-rule for functions.
This is not because we do not know how to handle them in the kernel%
\sidenote{\kl{Coq}’s kernel has an implementation that takes care of extensionality rules
in a term-directed fashion.}
but rather because it is difficult to give a good specification of them in the
\kl(conv){untyped} setting chosen for \kl{MetaCoq}’s conversion.%
\sidenote{I gave a workshop presentation \cite{LennonBertrand2022a} on this issue.}%
\margincite{LennonBertrand2022a}
Thus, this could be a first step towards a specification of
\kl{MetaCoq} using \kl{typed conversion}, which would facilitate the incorporation of
extensionality rules that are currently direly missing to the project.

The chapter is organized as follows: \cref{sec:bidir-conv} introduces the main relation we will
be interested in, namely the bidirectional conversion inspired by \textcite{Abel2017};
\cref{sec:bidir-conv-meta} discusses the difficulties encountered when establishing the
meta-theory of such a system; finally, \cref{sec:unty-conv-equiv} presents the equivalence
between this bidirectional conversion and an untyped one, very close to how \kl{Coq}’s kernel
handles η-rules.

\section{Bidirectional Conversion}
\label{sec:bidir-conv}

\subsection{Extensionality and η-rules}
\label{sec:eta-rules}

Before we can get to bidirectional conversion, let us first go over why using typed conversion
is interesting. Typed conversion is as old as type theory itself \cite{MartinLoef1972},
and there are two main reasons that make it a better choice over untyped conversion.
The first is that it is easier to build models%
\sidenote{Or logical relations, translations…} using typed conversion,
because these can use the extra type information to interpret conversion based on the type.
But the reason that is of interest to us here, as we do not build such
models, are extensionality rules.

In general, extensionality rules allow equating two terms, not based on their shape,%
\sidenote{As is the case of all the rules introduced so far, especially β and ι.}
but on the type. The most basic one is that for functions,
which says that any function $f$ and $g$ of type $\P x : A.\ B$
should be convertible whenever $f\ x \conv g\ x \ty B$ – note that here $f$ and $g$ are
\emph{any} functions.
As their name suggest, this kind of rules constrain the
system to not be too intentional. For instance, in the case of functions, $f$ and $g$ cannot
contain any “hidden” information that disappear when observing their behaviour using
application, because that information would disappear when applying the extensionality rule.
In \kl{Coq}, similar extensionality rules exist for dependent pair types%
\sidenote{Or more generally for record types, which are a generalized version of these,
see \cref{sec:pcuic-ind}.}
– saying that $p$ and $q$ of type $\Sb x : A.\ B$, $p$ and $q$ are convertible whenever
their two components are –
and for strict propositions – saying that whenever $P \ty \SProp$, and $p \ty P$, $q \ty P$,
$p$ and $q$ are convertible.

In the case of functions,%
\sidenote{Something similar happens for record types.}
the extensionality rule is inter-derivable with what is called the η-rule, which equates
$f$ and $\l x : A. f~x$. While less useful than β-rules, η-rules are still valuable.
For instance, in the setting of homotopy type theory, they are needed to deduce function
extensionality from univalence \sidecite{UniFoundationsProgram2013}.

\subsection{Conversion checks, neutral comparison infers}

If we wish to describe such type-based rules, it is natural to wish for a typing relation
that maintains that type, in order to use it to trigger extensionality rules.
This what happens for instance in \kl{Agda} \sidecite{Norell2007}.%
\sidenote{Although in the specific case of functions,
the \kl{Agda} implementation actually uses the
same term-directed technique as \kl{Coq}, as presented in \cref{sec:unty-conv-equiv}.}
Because the type is maintained, we need the conversion rules to obey McBride’s discipline too.
A nice presentation of this is given by the “algorithmic conversion” of \sidetextcite{Abel2017},
from which we take inspiration here to describe a bidirectional conversion relation for
\kl{CCω}.

The important intuition about this relation is that it actually decomposes conversion in two
relations. On one side, \intro{generic conversion}, that we will continue writing $\bconvop$,
which takes a type as \kl{input} – \ie, it \emph{checks}. On the other side,
\intro{neutral comparison}, written $\nconvop$, which takes a type as \kl{output} – it \emph{infers}.
There are two reasons for this. First, applying extensionality rules on neutrals is
useless, as it will simply create a blocked redex. For instance, if $n$ and $n'$ are neutral
functions, $n\ x$ and $n'\ x$ are convertible exactly when $n$ and $n'$ are. But more
importantly, the inferred information is used to know at which type the recursive appeals to
conversion need to be done. In the case of applications for instance, comparing $n\ t$ with
$n'\ t'$, we need to infer a type $\P x : A.\ B$ while comparing $n$ with $n'$ to recursively
compare $t$ to $t'$ at type $A$.

\begin{figure*}[h]
  \ContinuedFloat*
  \begin{mathpar}
  \inferdef{Check}{\inferty{\Gamma}{t}{T} \\ \bcum{\Gamma}{T}{T'}}{\checkty{\Gamma}{t}{T'}}
    \label{rule:bd-cum-check} \and
  \inferdef{RedCum}{T \hred T' \\ U \hred U' \\ \cumh{\Gamma}{T'}{U'}}{\bcum{\Gamma}{T}{U}}
    \label{rule:bd-red-cum}
  \end{mathpar}
  \caption{\kl{Generic cumulativity}}
  \label{fig:gene-cum}
\end{figure*}

We wish to extend \kl{CCω}, so the rules we present here are meant to complement
the rules of \cref{fig:ccw-bidir-infer,fig:bidir-ccw-other}, replacing \ruleref{rule:bd-check}
of \cref{fig:bidir-ccw-other} by \ruleref{rule:bd-cum-check} of \cref{fig:gene-cum}.
We cannot define a system based purely on \kl{conversion},%
\sidenote{This is due to the product rule, to which we will get soon.}
so we use \kl{generic cumulativity} $\mathord{\bcumop}$ instead.
Note also that there is no known level at which the two types should be compared,
hence \kl{generic cumulativity} “checks”, but against the mere
fact of being a type, rather than against a precise type. This is akin to the relation often
written $\Gamma \vdash T \conv T'$ or $\Gamma \vdash T \conv T'\ type$ often used in the
setting of Martin-Löf type theory.
To deduce \kl{generic cumulativity},
there is only one rule that applies, \ruleref{rule:bd-red-cum}:
both arguments are reduced to \kl{weak-head} normal forms, before being compared
by the auxiliary relation $\cumhop$.

\begin{figure*}[h]
  \ContinuedFloat
  \begin{mathpar}
    \inferdef{BdNeuCum}{\nconv{\Gamma}{N}{N'}{S}}{\cumh{\Gamma}{N}{N'}}
      \label{rule:bd-neu-cum} \and
    \inferdef{BdUniCum}{i \leq j}{\cumh{\Gamma}{\uni[i]}{\uni[j]}}
      \label{rule:bd-uni-cum} \and
    \inferdef{BdΠCum}{\bconv{\Gamma}{A}{A'} \\ \bcum{\Gamma, x : A}{B}{B'}}
      {\cumh{\Gamma}{\P x : A.\ B}{\P x : A'.\ B'}}
      \label{rule:bd-prod-cum}
  \end{mathpar}
  \caption{\kl{Generic cumulativity} between reduced types}
  \label{fig:gene-cumh}
\end{figure*}

This auxiliary relation, in turn, is defined by the rules of \cref{fig:gene-cumh}, which
either apply congruence rules if both types being compared are \kl{canonical} forms –
Rules \nameref{rule:bd-uni-cum} and \nameref{rule:bd-prod-cum} –, or call
\kl{neutral comparison} otherwise – \ruleref{rule:bd-neu-cum}.

\begin{figure*}[h]
  \ContinuedFloat
  \begin{mathpar}
    \inferdef{RedConvTy}{T \hred T' \\ U \hred U' \\ \convh{\Gamma}{T'}{U'}}
      {\bconv{\Gamma}{T}{U}}\label{rule:bd-red-conv-ty} \and
    \inferdef{BdNeuConvTy}{\nconv{\Gamma}{N}{N'}{S}}{\convh{\Gamma}{N}{N'}}
      \label{rule:bd-neu-conv-ty} \and
    \inferdef{BdUniConvTy}{i = j}{\convh{\Gamma}{\uni[i]}{\uni[j]}}
      \label{rule:bd-uni-conv-ty} \and
    \inferdef{BdΠConvTy}{\bconv{\Gamma}{A}{A'} \\ \bconv{\Gamma, x : A}{B}{B'}}
      {\convh{\Gamma}{\P x : A.\ B}{\P x : A'.\ B'}}
      \label{rule:bd-prod-conv-ty} \and
  \end{mathpar}
  \caption{\kl{Generic conversion} between types}
  \label{fig:gene-conv-ty}
\end{figure*}

\kl{Generic conversion} is defined in \cref{fig:gene-conv-ty}, in a way very similar to
\kl{generic cumulativity}.

\begin{figure*}[h]
  \ContinuedFloat
  \begin{mathpar}
    \inferdef{VarComp}{(x : A) \in \Gamma}{\nconv{\Gamma}{x}{x}{A}}
      \label{rule:neu-comp-var}\and
    \inferdef{AppComp}{\pnconv{\P}{\Gamma}{n}{n'}{\P x : A.\ B} \\ \bconv{\Gamma}{t}{t'}[A]}
      {\nconv{\Gamma}{n\ t}{n\ t'}{\subs{B}{x}{t}}}
      \label{rule:neu-comp-app} \and
    \inferdef{RedComp}{\nconv{\Gamma}{n}{n'}{T} \\ T \hred \P x : A.\ B}{\pnconv{\P}{\Gamma}{n}{n'}{\P x : A.\ B}}
      \label{rule:neu-comp-red}
  \end{mathpar}
  \caption{\kl{Neutral comparison}}
  \label{fig:neu-comp}
\end{figure*}

Next, we get to \kl{neutral comparison}, in \cref{fig:neu-comp}. Neutrals are
related exactly when they are the same variable, applied to two lists of recursively convertible arguments. The interesting rule is \ruleref{rule:neu-comp-app}, where we see
the behaviour described earlier: the domain of the inferred type for the neutral is used to
compare the arguments.

\begin{figure*}[h]
  \ContinuedFloat
  \begin{mathpar}
    \inferdef{RedConvTm}{t \hred t' \\ u \hred u' \\ A \hred A' \\ \convh{\Gamma}{t'}{u'}[A']}
      {\bconv{\Gamma}{t}{u}[A]}\label{rule:bd-red-conv-tm} \and
    \inferdef{BdNeuConvTm}{\nconv{\Gamma}{n}{n'}{S}}{\convh{\Gamma}{n}{n'}[T]}
      \label{rule:bd-neu-conv-tm} \and
    \inferdef{BdUniConvTm}{i = j}{\convh{\Gamma}{\uni[i]}{\uni[j]}[\uni[k]]}
      \label{rule:bd-uni-conv-tm} \and
    \inferdef{BdΠConvTm}{\bconv{\Gamma}{A}{A'}[\uni[i]] \\
      \bconv{\Gamma, x : A}{B}{B'}[\uni[i]]}
      {\convh{\Gamma}{\P x : A.\ B}{\P x : A'.\ B'}[\uni[i]]}
      \label{rule:bd-prod-conv-tm}
  \end{mathpar}
  \caption{\kl{Generic conversion} between terms at the universe}
  \label{fig:gene-conv-tm}
\end{figure*}

Finally, we get to \kl{generic conversion} between terms, which is called recursively by
\kl{neutral comparison}.
The first set of rules, given in \cref{fig:gene-conv-tm} is very similar to the one for types.
First, the two terms are reduced, and the type at which they are compared too,
and the terms are then compared using relation $\convhop$
(\ruleref{rule:bd-red-conv-tm}). If the terms are neutrals, \kl{neutral comparison} is used
(\ruleref{rule:bd-neu-conv-tm}).

Otherwise, congruence rules must be used. In case the comparison happens at the universe, 
these are very similar to that for types (Rules \nameref{rule:bd-uni-conv-tm} and
\nameref{rule:bd-prod-conv-tm}).
Note however, that to maintain the well-formedness invariant mandated by McBride’s discipline,
we should always call $\bconv{\Gamma}{t}{t'}{A}$ when we know that both $t$ and $t'$
check against $A$. But in \ruleref{rule:bd-prod-conv-tm}, the domains and codomains might be at
a universe level lower that $i$ even if the whole product is.%
\sidenote{For instance, $A$ might be $\uni[0]$ and $B$ might be $\uni[1]$.}
Thus, in order to recursively compare $A$ to $A'$ and $B$ to $B'$, we must know that they still
check against $\uni[i]$, which requires \kl{cumulativity}.

\begin{figure}[h]
  \ContinuedFloat
  \begin{mathpar}
    \inferdef{BdFunConv}{
      \bconv{\Gamma, x : A}{f\ x}{f'\ x}[B]}
      {\convh{\Gamma}{f}{f'}[\P x : A.\ B]}
      \label{rule:bd-fun-conv}
  \end{mathpar}
  \caption{\kl{Generic conversion} between functions}
  \label{fig:gene-conv-fun}
\end{figure}

The last rule is that for comparing two functions (\ruleref{rule:bd-fun-conv}).
In that case, an extensionality rule is directly applied without even looking
at the two terms. There is thus no congruence rule for λ-abstractions, but it is actually
derivable, because $(\l x : A.\ t)\ x \hred t$, and so in case both $f$ and $f'$ are
abstractions, the recursive calls amount to comparing their bodies.

The rules as given directly translate to an algorithm, as they nicely term/type directed,
\ie there is always at most one rule that applies to derive a judgment. Moreover,
if in \kl{generic cumulativity} and \kl{generic conversion} we view all objects as \kl{inputs}%
\sidenote{The subject is the “computational content” of the judgment, \ie whether the
conversion/cumulativity holds. This is similar to the conversion judgments of
\textcite{Bauer2020}.}
\margincite{Bauer2020}
in \kl{neutral comparison} the type is an \kl{output} and all other objects are inputs,
and in \kl{reduction} $t \hred t'$, $t$ is an \kl{input} and $t'$ is an \kl{output}, then
all rules respect McBride’s discipline.

\section{Untyped Presentation}
\label{sec:unty-conv-equiv}

In the presentation of \cref{sec:bidir-conv}, types are carried around,
but almost never used. Indeed,
only \ruleref{rule:bd-fun-conv} really needs the type information to be applied.
However, there is an alternative approach, which is the one used by the kernels of \kl{Coq}
and \kl{Agda}, which avoids looking at types at all, by replacing \ruleref{rule:bd-fun-conv}
by term-directed rather than type-directed ones.

\subsection{The rules}

As we do not want to maintain types, there is also no point in maintaining the context either.
Thus, the \kl{conversion} we define is of the following form: $\buconv{\gamma}{t}{t'}$.

\begin{figure*}[h]
  \ContinuedFloat*
  \begin{mathpar}
  \inferdef{CheckUty}{\inferty{\Gamma}{t}{T} \\ \bucum{\left| \Gamma \right|}{T}{T'}}{\checkty{\Gamma}{t}{T'}}
    \label{rule:bd-ucum-check} \and
  \inferdef{RedCumUty}{t \hred t' \\ u \hred u' \\ \ucumh{\gamma}{t'}{u'}}
    {\bucum{\gamma}{t}{u}} \label{rule:bd-red-ucum} \and
  \inferdef{RedConvUty}{t \hred t' \\ u \hred u' \\ \uconvh{\gamma}{t'}{u'}}
    {\buconv{\gamma}{t}{u}} \label{rule:bd-red-uconv} \and
  \inferdef{BdNeuCumUty}{\nuconv{\gamma}{n}{n'}}{\ucumh{\gamma}{n}{n'}}
    \label{rule:bd-neu-ucum} \and
  \inferdef{BdNeuConvUty}{\nuconv{\gamma}{n}{n'}}{\uconvh{\gamma}{n}{n'}}
    \label{rule:bd-neu-uconv} \and
  \end{mathpar}
  \caption{Untyped cumulativity and conversion}
  \label{fig:gene-ucum}
\end{figure*}

The first rules of \cref{fig:gene-ucum} are similar to those for the typed variants: cumulativity can be used
in checking, and terms are compared by first reducing them to weak-head normal form, and if they are neutrals
the special \kl{neutral comparison} is called.

\begin{figure}[h]
  \ContinuedFloat
  \begin{mathpar}
  \inferdef{BdUniCumUty}{i \leq j}{\ucumh{\gamma}{\uni[i]}{\uni[j]}}
    \label{rule:bd-uni-ucum} \and
  \inferdef{BdΠCumUty}{\buconv{\gamma}{A}{A'} \\
    \bucum{\gamma, x}{B}{B'}}
    {\ucumh{\gamma}{\P x : A.\ B}{\P x : A'.\ B'}}
    \label{rule:bd-prod-ucum} \\
  \inferdef{BdUniConvUty}{i = j}{\uconvh{\gamma}{\uni[i]}{\uni[j]}}
    \label{rule:bd-uni-uconv} \and
  \inferdef{BdΠConvUty}{\buconv{\gamma}{A}{A'} \\
    \buconv{\gamma, x}{B}{B'}}
    {\uconvh{\gamma}{\P x : A.\ B}{\P x : A'.\ B'}}
    \label{rule:bd-prod-uconv}
  \end{mathpar}
  \caption{Untyped bidirectional conversion for types}
  \label{fig:bd-cong-univ}
\end{figure}

The rules for the comparison of types are given in \cref{fig:bd-uconv}, and are again
very similar to those for the typed variant: there is a congruence rule for Π-types,
and universes are convertible when their levels are in the right relation.

\begin{figure}[h]
  \ContinuedFloat
  \begin{mathpar}
    \inferdef{VarCompUty}{(x : A) \in \gamma}{\nuconv{\gamma}{x}{x}}
      \label{rule:neu-ucomp-var}\and
    \inferdef{AppCompUty}{\nuconv{\gamma}{n}{n'} \\ \buconv{\gamma}{t}{t'}}
      {\nuconv{\gamma}{n\ t}{n\ t'}}
      \label{rule:neu-ucomp-app}
  \end{mathpar}
  \caption{Untyped \kl{Neutral comparison}}
  \label{fig:neu-ucomp}
\end{figure}

In the case of \kl{neutral comparison}, the rules (\cref{fig:neu-ucomp}) are even simpler
than in the typed case, because there is no need for a special rule to reduce the type,
as they are not maintained. Thus, there are only two rules, one for application and one
base case for variables.

\begin{figure}[h]
  \ContinuedFloat
  \begin{mathpar}
  \inferdef{BdAbsCong}{\buconv{\gamma,x}{t}{t'}}
    {\uconvh{\gamma}{\l x : A.\ t}{\l x : A'.\ t'}} \label{rule:bd-abs-uconv} \\
  \inferdef{BdAbsNeu}{\buconv{\gamma,x}{t}{n'\ x} \\ \ne{n'}}{\uconvh{\gamma}{\l x : A.\ t}{n'}}
    \label{rule:bd-abs-neu} \and
  \inferdef{BdNeuAbs}{\buconv{\gamma,x}{n\ x}{t'} \\ \ne{n}}{\uconvh{\gamma}{n}{\l x : A'.\ t'}}
    \label{rule:bd-neu-abs}
  \end{mathpar}
  \caption{Untyped, bidirectional conversion for functions}
  \label{fig:bd-uconv-fun}
\end{figure}

Finally, the interesting difference appears in \cref{fig:bd-uconv-fun}. Here what was done
using only one generic rule (\ruleref{rule:bd-fun-conv}) is decomposed into four of them,
depending on whether each function in weak-head normal form is a neutral or an abstraction.
In case both are abstractions, the extensionality rules amounts to a congruence, \ie
\ruleref{rule:bd-abs-uconv}.%
\sidenote{If we maintain the invariant that both terms that are compared have a common type,
then there is no need to compare the domains of the abstractions because they are always
convertible.} 
In case both are neutrals, the extensionality rule only inserts
a useless application to a variable, but \kl{neutral comparison} can be directly used
instead. The only setting where the extensionality rule is useful is when comparing a neutral
to an abstraction. But in those cases, the information that the comparison happens at a function
type and that the neutral needs to be η-expanded can be obtained from the abstraction.
This is what the symmetric Rules \nameref{rule:bd-abs-neu} and \nameref{rule:bd-neu-abs} do.

\subsection{Equivalences between the systems}

The presentations are so similar they ought to be equivalent. We can show this, assuming
for a moment that the system with \kl{typed conversion} has the good properties we now list.

%to fill

\paragraph{Typed to untyped}

Let us first tackle the typed-to-untyped direction, which has to erase the typing information.
To avoid cluttering the statements with a lot of assumptions, we will simply say “inputs are
well-formed” with the following meaning.

\begin{definition}[Inputs well-formation]
  We say that “inputs are well-formed” for a relation to mean the following:
  \begin{itemize}
    \item in the case of $\bconv{\Gamma}{t}{t'}[T]$ and of $\convh{\Gamma}{t}{t'}[T]$,
      that $\vdash \Gamma$,
      that there exists $i$ such that $\pinferty{\uni}{\Gamma}{T}{\uni[i]}$,
      and that $\checkty{\Gamma}{t}{T}$ and $\checkty{\Gamma}{t'}{T}$;
    \item in the case of $\bconv{\Gamma}{T}{T'}$, $\convh{\Gamma}{T}{T'}$,
      $\bcum{\Gamma}{T}{T'}$ and $\cumh{\Gamma}{T}{T'}$, that $\vdash \Gamma$,
      and that there exist $i$ and $j$ such that $\pinferty{\uni}{\Gamma}{T}{\uni[i]}$
      and $\pinferty{\uni}{\Gamma}{T'}{\uni[j]}$;
    \item in the case of $\nconv{\Gamma}{n}{n'}{T}$ and of $\pnconv{\P}{\Gamma}{n}{n'}{T}$,
      that $\vdash \Gamma$,
      and that there exists $T'$ such that $\inferty{\Gamma}{n'}{T'}$.%
      \sidenote{We do not demand that $\inferty{\Gamma}{n}{T''}$ for some $T''$, as that
      already follows from the relation holding that $\inferty{\Gamma}{n}{T}$.}
  \end{itemize}
\end{definition}

Because McBride’s discipline is respected in the structure of the rules, inputs well-formation is preserved, in
the following sense.

\begin{lemma}[Preservation of well-formation]
  Whenever one of the typed conversion relation holds and its inputs are well-typed, then inputs are well-typed
  in all sub-derivations.
\end{lemma}

\begin{proof}
  The proof is by mutual induction. It requires stability of typing by context/type cumulativity to handle the fact that our
  rules are left-leaning — \eg context extension in \ruleref{rule:bd-prod-conv-ty} is done using the domain of the left Π-type –,
  and to deduce that the η-expansions of \ruleref{rule:bd-fun-conv} are well-formed.
  Subject reduction is needed to know that weak-head reduction preserves well-formation.
  Finally, validity (that can be alternatively seen as well-formation of outputs), is necessary in \ruleref{rule:neu-comp-app}
  to ensure that $t'$ indeed checks against $A$.  
\end{proof}

The only rule that needs looking at is, of course, that which differs between the two systems, \ie \ruleref{rule:bd-fun-conv}.
But we have the following lemma.

\begin{lemma}[Injectivity of η-expansion]
  If $\convh{\Gamma}{f}{f'}{B}$ holds, its inputs are well-formed, and $\buconv{| \Gamma |,x}{f\ x}{f'\ x}$ holds too,
  then $\uconvh{| \Gamma |}{f}{f'}$.
\end{lemma}

\begin{proof}
  By inversion on $\convh{\Gamma}{f}{f'}{\P x : A.\ B}$, we know that $f\ x$ and $f'\ x$ reduce to weak-head normal forms,
  say $f\ x \hred v$, $f'\ x \hred v'$ and by
  inversion on these derivations, we get that also $f$ and $f'$ reduce to weak-head normal forms, say
  $f \hred w$ and $f' \hred w'$.
  By inversion on $\buconv{| \Gamma |,x}{f\ x}{f'\ x}$ and because weak-head normal forms are unique,
  we also get that and $\uconvh{| \Gamma |,x}{v}{v'}$.
  Moreover, because of input well-formation and \kl{subject reduction}, we know that both $w$ and $w'$
  check against $\P x : A.\ B$. Since they are weak-head normal forms, they must thus be either λ-abstractions, or neutrals.
  We thus have four cases to consider.

  In case both $w$ and $w'$ are λ-abstractions, say respectively $\l x : A.\ t$ and $\l x : A'.\ t'$, we have that
  $f\ x \hred w\ x \hored t$, and similarly $f'\ x \hred t'$. Because weak-head reduction is deterministic,
  we must have $t \hred v$ and $t' \hred v'$, but then since $\uconvh{| \Gamma |,x}{v}{v'}$ we also have
  $\buconv{|\Gamma |,x}{t}{t'}$. Thus, we can apply \ruleref{rule:bd-abs-uconv} and conclude.

  In case $w$ is a λ-abstraction, say $\l x : A.\ t$ and $w'$ is a neutral $n'$, then $v'$ must be equal to $n'\ x$.
  Then we have $f\ x \hred w\ x \hored t \hred v$, and thus $\buconv{|\Gamma|,x}{t}{n'\ x}$ since $\uconvh{|\Gamma|,x}{v}{n'\ x}$.
  Thus, \ruleref{rule:bd-abs-neu} applies to conclude. The reasoning in the symmetric case where $w'$ is an abstraction
  and $w$ is not is similar.

  In the last case, both $w$ and $w'$ are neutrals, say $n$ and $n'$. Then $v$ and $v'$ are respectively $n\ x$ and $n'\ x$.
  Since $\uconvh{|\Gamma|,x}{n\ x}{n'\ x}$, we must have also $\nuconv{|\Gamma|,x}{n\ x}{n'\ x}$ because all rules but
  \ruleref{rule:bd-neu-uconv} equate canonical forms. But then the last rule that applies must have been
  \ruleref{rule:neu-comp-app}, and thus we have $\nuconv{|\Gamma|,x}{n}{n'}$. From this, we can get $\uconvh{|\Gamma|,x}{n}{n'}$
  and since $f \hred n$ and $f' \hred n'$, we finally obtain $\buconv{|\Gamma|,x}{f}{f'}$, as expected.
\end{proof}


\begin{theorem}[Typed to untyped bidirectional conversion]
  The following implications hold whenever inputs are well-formed:
  \begin{itemize}
    \item if $\bconv{\Gamma}{t}{t'}{T}$ or $\bconv{\Gamma}{t}{t'}$, then $\buconv{| \Gamma |}{t}{t'}$;
    \item if $\bcum{\Gamma}{T}{T'}$, then $\bucum{|\Gamma|}{T}{T'}$;
    \item if $\convh{\Gamma}{t}{t'}{T}$ or $\convh{\Gamma}{t}{t'}$ then $\uconvh{| \Gamma |}{t}{t'}$;
    \item if $\nconv{\Gamma}{n}{n'}{T}$ or $\pnconv{\P}{\Gamma}{n}{n'}{T}$ then $\nuconv{| \Gamma |}{n}{n'}$.
  \end{itemize}
\end{theorem}



\section{Meta-Theory of the Bidirectional System}
\label{sec:bidir-conv-meta}

\subsection{The Properties We Can Prove}

\subsection{The Properties We Can’t Prove}
